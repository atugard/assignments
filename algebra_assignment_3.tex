\documentclass[12pt]{extarticle}
%Some packages I commonly use.
\usepackage[english]{babel}
\usepackage{graphicx}
\usepackage{framed}
\usepackage[normalem]{ulem}
\usepackage{amsmath}
\usepackage{amsthm}
\usepackage{amssymb}
\usepackage{amsfonts}
\usepackage{enumerate}
\usepackage[utf8]{inputenc}
\usepackage[top=1 in,bottom=1in, left=1 in, right=1 in]{geometry}

%A bunch of definitions that make my life easier
\newcommand{\matlab}{{\sc Matlab} }
\newcommand{\abs}[1]{|#1|}
\newcommand{\set}[1]{\{#1\}}
\newcommand{\cvec}[1]{{\mathbf #1}}
\newcommand{\rvec}[1]{\vec{\mathbf #1}}
\newcommand{\ihat}{\hat{\textbf{\i}}}
\newcommand{\im}{\text{im}}
\newcommand{\jhat}{\hat{\textbf{\j}}}
\newcommand{\khat}{\hat{\textbf{k}}}
\newcommand{\minor}{{\rm minor}}
\newcommand{\trace}{{\rm trace}}
\newcommand{\spn}{{\rm Span}}
\newcommand{\rem}{{\rm rem}}
\newcommand{\ran}{{\rm range}}
\newcommand{\range}{{\rm range}}
\newcommand{\mdiv}{{\rm div}}
\newcommand{\proj}{{\rm proj}}
\newcommand{\R}{\mathbb{R}}
\newcommand{\C}{\mathbb{C}}
\newcommand{\F}{\mathbb{F}}
\newcommand{\N}{\mathbb{N}}
\newcommand{\Q}{\mathbb{Q}}
\newcommand{\Z}{\mathbb{Z}}
\newcommand{\<}{\langle}
\newcommand{\ideal}{\triangleleft}
\renewcommand{\>}{\rangle}
\renewcommand{\emptyset}{\varnothing}
\newcommand{\attn}[1]{\textbf{#1}}
\theoremstyle{definition}
\newtheorem{theorem}{Theorem}
\newtheorem{prob}{Problem}
\newtheorem{corollary}{Corollary}
\newtheorem*{definition}{Definition}
\newtheorem*{example}{Example}
\newtheorem*{note}{Note}
\newtheorem{exercise}{Exercise}
\newcommand{\bproof}{\bigskip {\bf Proof. }}
\newcommand{\eproof}{\hfill\qedsymbol}
\newcommand{\Disp}{\displaystyle}
\newcommand{\qe}{\hfill\(\bigtriangledown\)}
\setlength{\columnseprule}{1 pt}


\title{ Math 5107 -- Algebra Assignment 3}
\author{David Draguta}
\date{2021-11-11}

\begin{document}

\maketitle

\begin{exercise}
  Let $R = \mathbb{F}_5[x,y]$. Compute the homology of the following complex, i.e. find presentations for all the homology modules.
  \begin{align*}
    0 \xrightarrow{0_1} R^2 \xrightarrow{d_2} R^3 \xrightarrow{d_3} R^1 \xrightarrow{0_2} 0 
  \end{align*}
  where
  \begin{align*}
    d_2 = 
    \begin{pmatrix}
      x^2  & -2xy+y^2 \\
      xy   & x^2 - y^2 \\
      -y^2 & xy - 2y^2 
    \end{pmatrix}
  \end{align*}
  and
    \begin{align*}
    d_1 = 
    \begin{pmatrix}
      -2x^2y^2+2xy^3+y^4 & x^3y-2x^2y^2-2xy^3+y^4 & -x^4-x^2y^2+xy^3
    \end{pmatrix}
  \end{align*}
\end{exercise}
\begin{proof}
  We get the following homology modules
  \begin{align*}
    M_1 = \ker(d_2)/\im(0_1), \quad M_2 = \ker(d_3)/\im(d_2), \quad M_3 = \ker(0_2)/\im(d_3)
  \end{align*}
\end{proof}


\end{document}
