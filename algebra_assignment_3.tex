\documentclass[12pt]{extarticle}
%Some packages I commonly use.
\usepackage[english]{babel}
\usepackage{graphicx}
\usepackage{framed}
\usepackage[normalem]{ulem}
\usepackage{amsmath}
\usepackage{amsthm}
\usepackage{amssymb}
\usepackage{amsfonts}
\usepackage{enumerate}
\usepackage[utf8]{inputenc}
\usepackage[top=1 in,bottom=1in, left=1 in, right=1 in]{geometry}

%A bunch of definitions that make my life easier
\newcommand{\matlab}{{\sc Matlab} }
\newcommand{\abs}[1]{|#1|}
\newcommand{\set}[1]{\{#1\}}
\newcommand{\cvec}[1]{{\mathbf #1}}
\newcommand{\rvec}[1]{\vec{\mathbf #1}}
\newcommand{\ihat}{\hat{\textbf{\i}}}
\newcommand{\im}{\text{im}}
\newcommand{\jhat}{\hat{\textbf{\j}}}
\newcommand{\khat}{\hat{\textbf{k}}}
\newcommand{\minor}{{\rm minor}}
\newcommand{\trace}{{\rm trace}}
\newcommand{\spn}{{\rm Span}}
\newcommand{\rem}{{\rm rem}}
\newcommand{\ran}{{\rm range}}
\newcommand{\range}{{\rm range}}
\newcommand{\mdiv}{{\rm div}}
\newcommand{\proj}{{\rm proj}}
\newcommand{\R}{\mathbb{R}}
\newcommand{\C}{\mathbb{C}}
\newcommand{\F}{\mathbb{F}}
\newcommand{\N}{\mathbb{N}}
\newcommand{\Q}{\mathbb{Q}}
\newcommand{\Z}{\mathbb{Z}}
\newcommand{\<}{\langle}
\newcommand{\ideal}{\triangleleft}
\renewcommand{\>}{\rangle}
\renewcommand{\emptyset}{\varnothing}
\newcommand{\attn}[1]{\textbf{#1}}
\theoremstyle{definition}
\newtheorem{theorem}{Theorem}
\newtheorem{prob}{Problem}
\newtheorem{corollary}{Corollary}
\newtheorem*{definition}{Definition}
\newtheorem*{example}{Example}
\newtheorem*{note}{Note}
\newtheorem{exercise}{Exercise}
\newcommand{\bproof}{\bigskip {\bf Proof. }}
\newcommand{\eproof}{\hfill\qedsymbol}
\newcommand{\Disp}{\displaystyle}
\newcommand{\qe}{\hfill\(\bigtriangledown\)}
\setlength{\columnseprule}{1 pt}


\title{ Math 5107 -- Algebra Assignment 3}
\author{David Draguta}
\date{2021-11-11}

\begin{document}

\maketitle

\begin{exercise}
  Let $R = \mathbb{F}_5[x,y]$. Compute the homology of the following complex, i.e. find presentations for all the homology modules.
  \begin{align*}
    0 \xrightarrow{0_1} R^2 \xrightarrow{d_1} R^3 \xrightarrow{d_2} R^1 \xrightarrow{0_2} 0 
  \end{align*}
  where
  \begin{align*}
    d_1 = 
    \begin{pmatrix}
      x^2  & -2xy+y^2 \\
      xy   & x^2 - y^2 \\
      -y^2 & xy - 2y^2 
    \end{pmatrix}
  \end{align*}
  and
    \begin{align*}
    d_2 = 
    \begin{pmatrix}
      -2x^2y^2+2xy^3+y^4 & x^3y-2x^2y^2-2xy^3+y^4 & -x^4-x^2y^2+xy^3
    \end{pmatrix}
  \end{align*}
\end{exercise}
\begin{proof}
  We get the following homology modules
  \begin{align*}
    M_1 = \ker(d_2)/\im(0_1), \quad M_2 = \ker(d_3)/\im(d_2), \quad M_3 = \ker(0_2)/\im(d_3).
  \end{align*}

  Using the Macaulay2 calculator, one finds that $ker(d_3) = im(d_2)$, so that $M_2 = 0$. Also, $M_3 = 0$ too as $ker(0_2)=R^1$ and $im(d_3) = R^1$.

  Thus, this homology is trivial, i.e. all of the homology modules are zero. 

\end{proof}

\begin{exercise}
  Let $R= \Q[x,y]$ and $M$ be the $R$-module with presentation given by the matrix
  \begin{align*}
    A =
    \begin{pmatrix}
      x^2     & xy          & y^2 \\
      xy+4y^4 & 6x^2 + 3y^2 & 3xy + 2y^2  
    \end{pmatrix}
  \end{align*}
  Compute a free resolution of $M$.
\end{exercise}
\begin{proof}
  We have the free resolution
  \begin{align*}
    R^2 \xrightarrow{A} R^2 \xrightarrow{\pi_{AR^2}} M \xrightarrow{0} 0 
  \end{align*}
\end{proof}
\begin{exercise}
  Suppose that we have a finite chain complex $C$ of finite dimensional $k$ vector spaces with boundary map $\delta_i: C_i \to C_{i-1}$
  \begin{align*}
    0 \to C_n \to \dots C_0 \to 0
  \end{align*}
  Let $r_i$ be the rank of $\delta_i$ and $d_i = dim C_i $.
    \begin{enumerate}
    \item
      Write $\dim(H_i(C))$ in terms of $r_j$ and $d_j$.
    \item
      Define the Euler characteristic
      \begin{align*}
        \chi(C) = \sum\limits_{i} (-1)^i dim H_i(C)
      \end{align*}
      Show that
      \begin{align*}
        \chi(C) = \sum\limits_{i} (-1)^i dim (C_i).
      \end{align*}
    \end{enumerate}
\end{exercise}
\begin{proof}
  \begin{enumerate}
  \item
    We have 
    \begin{align*}
      h_i
      &= \dim(H_i) = \dim (\ker(\delta_i)/\im(\delta_{i+1})) \\
      &= \dim (\ker(\delta_i)/\im(\delta_{i+1})) \\
      &= \dim (\ker(\delta_i)) + \dim(\im(\delta_i)) - \dim(\im(\delta_i)) - \dim(\im(\delta_{i+1}))  \\
      &=  c_i - r_i - r_{i+1}
    \end{align*}
  \item
    \begin{align*}
      \sum\limits_i (-1)^i h_i &= \sum\limits_i (-1)^i (c_i - r_i - r_{i+1})  \\
      &= (c_n - r_n - r_{n+1}) - (c_{n-1} - r_{n-1} - r_n) + \cdots + (-1)^n(c_0 - r_0 - r_1) \\
      &= \sum \limits_i (-1)^i c_i - r_{n+1} + (-1)^{n+1} r_0 \\
      &= \sum \limits_i (-1)^i c_i
    \end{align*}
  \end{enumerate}
\end{proof}
\end{document}
