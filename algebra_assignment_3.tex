\documentclass[12pt]{extarticle}
%Some packages I commonly use.
\usepackage[english]{babel}
\usepackage{graphicx}
\usepackage{framed}
\usepackage[normalem]{ulem}
\usepackage{amsmath}
\usepackage{amsthm}
\usepackage{amssymb}
\usepackage{amsfonts}
\usepackage{enumerate}
\usepackage[utf8]{inputenc}
\usepackage[top=1 in,bottom=1in, left=1 in, right=1 in]{geometry}

%A bunch of definitions that make my life easier
\newcommand{\matlab}{{\sc Matlab} }
\newcommand{\abs}[1]{|#1|}
\newcommand{\set}[1]{\{#1\}}
\newcommand\restr[2]{{% we make the whole thing an ordinary symbol
  \left.\kern-\nulldelimiterspace % automatically resize the bar with \right
  #1 % the function
  \vphantom{\big|} % pretend it's a little taller at normal size
  \right|_{#2} % this is the delimiter
  }}
\newcommand{\cvec}[1]{{\mathbf #1}}
\newcommand{\rvec}[1]{\vec{\mathbf #1}}
\newcommand{\ihat}{\hat{\textbf{\i}}}
\newcommand{\im}{\text{im}}
\newcommand{\cok}{\text{cok}}
\newcommand{\jhat}{\hat{\textbf{\j}}}
\newcommand{\khat}{\hat{\textbf{k}}}
\newcommand{\minor}{{\rm minor}}
\newcommand{\trace}{{\rm trace}}
\newcommand{\spn}{{\rm Span}}
\newcommand{\rem}{{\rm rem}}
\newcommand{\ran}{{\rm range}}
\newcommand{\range}{{\rm range}}
\newcommand{\mdiv}{{\rm div}}
\newcommand{\proj}{{\rm proj}}
\newcommand{\R}{\mathbb{R}}
\newcommand{\C}{\mathbb{C}}
\newcommand{\F}{\mathbb{F}}
\newcommand{\N}{\mathbb{N}}
\newcommand{\Q}{\mathbb{Q}}
\newcommand{\Z}{\mathbb{Z}}
\newcommand{\<}{\langle}
\newcommand{\ideal}{\triangleleft}
\renewcommand{\>}{\rangle}
\renewcommand{\emptyset}{\varnothing}
\newcommand{\attn}[1]{\textbf{#1}}
\theoremstyle{definition}
\newtheorem{theorem}{Theorem}
\newtheorem{prob}{Problem}
\newtheorem{corollary}{Corollary}
\newtheorem*{definition}{Definition}
\newtheorem*{example}{Example}
\newtheorem*{note}{Note}
\newtheorem{exercise}{Exercise}
\newcommand{\bproof}{\bigskip {\bf Proof. }}
\newcommand{\eproof}{\hfill\qedsymbol}
\newcommand{\Disp}{\displaystyle}
\newcommand{\qe}{\hfill\(\bigtriangledown\)}
\setlength{\columnseprule}{1 pt}


\title{ Math 5107 -- Algebra Assignment 3}
\author{David Draguta}
\date{2021-11-11}

\begin{document}

\maketitle

\begin{exercise}
  Let $R = \mathbb{F}_5[x,y]$. Compute the homology of the following complex, i.e. find presentations for all the homology modules.
  \begin{align*}
    0 \xrightarrow{0} R^2 \xrightarrow{d_1} R^3 \xrightarrow{d_2} R^1 \xrightarrow{0} 0 
  \end{align*}
  where
  \begin{align*}
    d_1 = 
    \begin{pmatrix}
      x^2  & -2xy+y^2 \\
      xy   & x^2 - y^2 \\
      -y^2 & xy - 2y^2 
    \end{pmatrix}
  \end{align*}
  and
    \begin{align*}
    d_2 = 
    \begin{pmatrix}
      -2x^2y^2+2xy^3+y^4 & x^3y-2x^2y^2-2xy^3+y^4 & -x^4-x^2y^2+xy^3
    \end{pmatrix}
  \end{align*}
\end{exercise}
\begin{proof}
  Using the Macaulay2 calculator one finds
  \begin{align*}
    \ker(d_1) &= 0, \\
    \ker(d_2) &= \im(d_1)
  \end{align*}
  so that
  \begin{align*}
    H_0 &= \ker(d_1)/\im(0) \\
    &= 0,
  \end{align*}
  \begin{align*}
    H_1 &= \ker(d_2)/\im(d_1)  \\
    &= 0
  \end{align*}
  and
  \begin{align*}
    H_2
    &= \ker(0)/\im(d_2) \\
    &= R/d_2R^3 \\
    &= cok(d_2).
  \end{align*}
  
\end{proof}

\begin{exercise}
  Let $R= \Q[x,y]$ and $M$ be the $R$-module with presentation given by the matrix
  \begin{align*}
    A =
    \begin{pmatrix}
      x^2     & xy          & y^2 \\
      xy+4y^4 & 6x^2 + 3y^2 & 3xy + 2y^2  
    \end{pmatrix}
  \end{align*}
  Compute a free resolution of $M$.
\end{exercise}
\begin{proof}
  We have that
  \begin{align*}
    M \cong cok(A) = R/AR,
  \end{align*}
  which yields the following free resolution 
  \begin{align*}
    R^3 \xrightarrow{A} R^2 \xrightarrow{\pi_{AR^2}} M \xrightarrow{0} 0.
  \end{align*}
\end{proof}
\begin{exercise}
  Suppose that we have a finite chain complex $C$ of finite dimensional $k$ vector spaces with boundary map $\delta_i: C_i \to C_{i-1}$
  \begin{align*}
    0 \to C_n \to \dots C_0 \to 0
  \end{align*}
  Let $r_i$ be the rank of $\delta_i$ and $d_i = dim C_i $.
    \begin{enumerate}
    \item
      Write $\dim(H_i(C))$ in terms of $r_j$ and $d_j$.
    \item
      Define the Euler characteristic
      \begin{align*}
        \chi(C) = \sum\limits_{i} (-1)^i dim H_i(C)
      \end{align*}
      Show that
      \begin{align*}
        \chi(C) = \sum\limits_{i} (-1)^i dim (C_i).
      \end{align*}
    \end{enumerate}
\end{exercise}
\begin{proof}
  In what follows we're going to use that for a linear transformation of vector spaces $T:V \to W$, that 
  \begin{align*}
    \dim(V) = dim(ker(T)) + dim(im(T)).
  \end{align*}
  Thus, we have for $W \subseteq V$ a subspace and $\pi_W : V \to V/W$ the projection onto $W$ that 
  \begin{align*}
    \dim(V)
    &= \dim(\ker(\pi_W)) + \dim(\im(\pi_W)) \\
    &= \dim(W) + \dim(V/W),
  \end{align*}
  so that in particular
  \begin{align*}
    \dim(V/W) = \dim(V) - \dim(W)
  \end{align*}
  
  
  %% or for an $m \times n$ matrix $A$ that
  %% \begin{align*}
  %%   \dim(k^n) &= \dim(\ker(A)) + \dim(\im(A)) \\
  %%   &= \dim(nullsp(A)) + \dim(colsp(A)) \\
  %%   &= nullity(A) + rank(A)
  %% \end{align*}

  
  \begin{enumerate}
  \item
    We have 
    \begin{align*}
      h_i
      &= \dim(H_i) = \dim (\ker(\delta_i)/\im(\delta_{i+1})) \\
      &= \dim (\ker(\delta_i)/\im(\delta_{i+1})) \\
      &= \dim (\ker(\delta_i)) + \dim(\im(\delta_i)) - \dim(\im(\delta_i)) - \dim(\im(\delta_{i+1}))  \\
      &=  c_i - r_i - r_{i+1}
    \end{align*}
  \item
    \begin{align*}
      \sum\limits_i (-1)^i h_i &= \sum\limits_i (-1)^i (c_i - r_i - r_{i+1})  \\
      &= (c_n - r_n - r_{n+1}) - (c_{n-1} - r_{n-1} - r_n) + \cdots + (-1)^n(c_0 - r_0 - r_1) \\
      &= \sum \limits_i (-1)^i c_i - r_{n+1} + (-1)^{n+1} r_0 \\
      &= \sum \limits_i (-1)^i c_i,
    \end{align*}
    which holds since
    \begin{align*}
      r_0 = \dim(\im(0)) = 0,
    \end{align*}
    and
    \begin{align*}
      r_{n+1} = \dim(\im(0)) = 0.
    \end{align*}
  \end{enumerate}
\end{proof}

\begin{exercise}
  Let $A,B,C$ be abelian groups with maps $\alpha: A \to B$ and $\beta: B \to C$ . Prove that the following sequence is exact
  \begin{align*}
    0 \to \ker \alpha \to \ker \beta \alpha \to \ker \beta \to \cok \alpha \to \cok  \beta \alpha \to \cok \beta \to 0
  \end{align*}
\end{exercise}
\begin{proof}
  We have the following sequence of maps 
  \begin{align*}
    0 \xrightarrow{0} \ker \alpha \xrightarrow{id}  \ker \beta \alpha \xrightarrow{\alpha} \ker \beta \xrightarrow{\pi_{\im(\alpha)}} \cok \alpha \xrightarrow{\beta} \cok  \beta \alpha \xrightarrow{\pi_{\im(\beta)}} \cok \beta \xrightarrow{0} 0,
  \end{align*}
  where we assume the domains are adjusted to make sense...
  \begin{itemize}
  \item
    $0 \xrightarrow{0} \ker \alpha \xrightarrow{id}  \ker \beta \alpha:$

    $\im(0) = \set{0}$ and $ker(id) = \set{0}$, so this parts exact.
  \item
    $\ker \alpha \xrightarrow{id}  \ker \beta \alpha \xrightarrow{\alpha} \ker \beta$:

    $\im(\restr{\alpha}{\ker(a)}) = \ker(\alpha)$ and $\ker(\restr{\alpha}{\ker \beta \alpha}) = \ker(\alpha)$

  \item
    $\ker \beta \alpha \xrightarrow{\alpha} \ker \beta \xrightarrow{\pi_{\im(\alpha)}} \cok \alpha$:

    $\im(\restr{\alpha}{\ker \beta \alpha}) = \alpha(A) \cap \ker(\beta)$ and $\ker(\restr{\pi_{\im(\alpha)}}{\ker(\beta)}) = \im(\alpha) \cap \ker(\beta)$
  \item
    $\ker \beta \xrightarrow{\pi_{\im(\alpha)}} \cok \alpha \xrightarrow{\beta} \cok  \beta \alpha$:

    $\im(\ker(\beta \alpha)) = \im(\alpha) \cap \ker(\beta)$ and $\ker(\pi_{\im(\alpha)}) = \im(\alpha) \cap \ker(\beta)$.
  \item 
    $\cok \alpha \xrightarrow{\beta} \cok  \beta \alpha \xrightarrow{\pi_{\im(\beta)}} \cok \beta$:

    $\beta(\cok(\alpha)) = \beta(B/\alpha(B)) = \beta(B)/\beta\alpha(B)$ and $ker(\restr{\pi_{\im(\beta)}}{\cok \beta \alpha}) = \beta(B)/\beta\alpha(B)$
  \item
    $\cok  \beta \alpha \xrightarrow{\pi_{\im(\beta)}} \cok \beta \xrightarrow{0} 0$:

    $\im(\restr{\pi_{\im(\beta)}}{\cok \beta \alpha}) = \cok \beta $ and $\ker(0) = \cok \beta$.
  \end{itemize}
  Hence, the sequence is exact! 
\end{proof}
\begin{exercise}
  Let $k$ be a field and let $R = k[x]$. For all $i,m,n \geq 0$ compute
  \begin{align*}
    Ext_R^i(k[x]/(x^n), k[x]/(x^m)),
  \end{align*}
  and
  \begin{align*}
    Tor_R^i(k[x]/(x^n), k[x]/(x^m)).
  \end{align*}
\end{exercise}


\begin{proof}
  \begin{itemize}
  \item
    $Ext_{k[x]}^i(k[x]/(x^n), k[x]/(x^m))$:
    
    We have the following projective resolution of $k[x]/(x^n)$ as a $k[x]$ module:
    \begin{align*}
      0 \xrightarrow{0} k[x] \xrightarrow{x^n} k[x] \xrightarrow{\pi_{x^n}} k[x]/k[x]x^n \xrightarrow{0} 0 
    \end{align*}
    which we reverse the arrows of and apply the $hom(-, k[x]/(x^m))$ functor to and get
    \begin{align*}
      0 \xrightarrow{0_*} \hom(k[x], k[x]/k[x]x^m) \xrightarrow{x^n_*} \hom(k[x], k[x]/k[x]x^m) \xrightarrow{0_*} 0       
    \end{align*}
    which is isomorphic to
    \begin{align*}
      0 \xrightarrow{0} k[x]/k[x]x^m \xrightarrow{x^n} k[x]/k[x]x^m \xrightarrow{0} 0.
    \end{align*}
    The homology modules give us
    \begin{align*}
      Ext_{k[x]}^0
      &= ker(x^n)/im(0) = ker(x^n) \\
      &= 
      \begin{cases}
        \text{ if } n < m & k[x]x^{m-n}/k[x]x^m \\
        \text{ otherwise } n \geq m & k[x]/k[x]x^m 
      \end{cases}
    \end{align*}
    and
    \begin{align*}
      Ext_{k[x]}^1
      &= ker(0)/im(x^n) \\
      &= 
      \begin{cases}
        \text{ if } n < m & k[x]/k[x]x^n \\
        \text{ otherwise } n \geq m & k[x]/k[x]x^m 
      \end{cases}
    \end{align*}
  \item
    $Tor_{k[x]}^i(k[x]/(x^n), k[x]/(x^m)):$

    We can use the same projective resolution as before
    \begin{align*}
      0 \xrightarrow{0} k[x] \xrightarrow{x^n} k[x] \xrightarrow{\pi_{(x^n)}} k[x]/(x^n) \xrightarrow{0} 0  
    \end{align*}
    and then drop the second last term and apply $(-) \otimes k[x]/(x^m)$.
    We get
    \begin{align*}
      0 \xrightarrow{0} k[x] \otimes_{k[x]} k[x]/(x^m)  \xrightarrow{x^n_*} k[x] \otimes_{k[x]} k[x]/(x^m) \xrightarrow{0} 0  .
    \end{align*}
    We have
    \begin{align*}
      k[x] \otimes_{k[x]} k[x]/(x^m) \cong k[x]/(x^m),
    \end{align*}
    and so this sequence is equivalent to
    \begin{align*}
      0 \xrightarrow{0} k[x]/(x^m)  \xrightarrow{x^n} k[x]/(x^m) \xrightarrow{0} 0  .      
    \end{align*}
    Which is the same as in the case of the ext functor, and so we get:
    \begin{align*}
      Tor_{k[x]}^0(k[x]/(x^n), k[x]/(x^m)) = k[x]x^{m-n}/k[x]x^m 
    \end{align*}
    and
    \begin{align*}
      Tor_{k[x]}^1(k[x]/(x^n), k[x]/(x^m)) = k[x]x^{m-n}/k[x]x^n .
    \end{align*}
  \end{itemize}
\end{proof}
\begin{exercise}
  Let $p \in \Z$ be a prime number. For all $i,m,n \geq 0 $ compute
  \begin{align*}
    Ext_{\Z}^i(\Z/p^n\Z, \Z/p^m\Z)
  \end{align*}
  and
  \begin{align*}
    Tor_{\Z}^i(\Z/p^n\Z, \Z/p^m\Z)
  \end{align*}
  Explain how the above result can be combined with the fundamental theorem of finitely generated abelian groups to determine Ext and Tor for any finitely generated abelian groups.
\end{exercise}
\begin{proof}
  We start as before with the following projective resolution
  \begin{align*}
    0 \to \Z \xrightarrow{p^n} \Z \xrightarrow{\pi_{p^n}} \Z/p^n\Z \xrightarrow{0} 0 
  \end{align*}
  and then apply $\hom(-, \Z/p^m\Z)$ to get
  \begin{align*}
    0 \xrightarrow{0} \hom(\Z, \Z/p^m\Z) \xrightarrow{p^n_*} \hom(\Z, \Z/p^m\Z) \xrightarrow{0} 0
  \end{align*}
  which is the same as
  \begin{align*}
    0 \xrightarrow{0} \Z/p^m\Z \xrightarrow{p^n} \Z/p^m \Z \xrightarrow{0} 0.
  \end{align*}
  We have
  \begin{align*}
    Ext_{\Z}^0(\Z/p^n\Z, \Z/p^m\Z)
    &= \ker(p^n)/\im(0) \\
    &= 
    \begin{cases}
      \text{ if } n<m & p^{m-n}\Z/ p^m \Z \\
      \text{ otherwise } & \Z / p^m \Z
    \end{cases}
  \end{align*}
  and
  \begin{align*}
    Ext_{\Z}^1(\Z/p^n\Z, \Z/p^m\Z) &= \ker(0)/\im(p^n) \\
    &= (\Z/p^m \Z)/(p^n \Z / p^m \Z) \\
    &=
    \begin{cases}
      \text{ if } n<m & \Z/ p^n \Z \\
      \text{ otherwise } & \Z / p^m \Z
    \end{cases}
  \end{align*}

  We apply $(-) \otimes \Z/p^m\Z$ to the projective resolution of $\Z/p^n\Z$, remembering to drop $\Z/p^n \Z$ itself to get

  \begin{align*}
    0 \xrightarrow{0} \Z \otimes_{\Z} \Z/p^m\Z \xrightarrow{p^n_*} \Z \otimes_{\Z} \Z/p^m\Z \xrightarrow{0} 0 ,
  \end{align*}

  which is the same as

  \begin{align*}
    0 \xrightarrow{0} \Z/p^m\Z \xrightarrow{p^n} \Z/p^m\Z \xrightarrow{0} 0 .
  \end{align*}
  Thus, in this case we have the same result as in the previous case
  \begin{align*}
    Tor_{\Z}^0(\Z/p^n\Z, \Z/p^m\Z) =
    \begin{cases}
      \text{ if } n<m & p^{m-n}\Z/ p^m \Z \\
      \text{ otherwise } & \Z / p^m \Z
    \end{cases}
  \end{align*}
  and
  \begin{align*}
    Tor_{\Z}^1(\Z/p^n\Z, \Z/p^m\Z) =
    \begin{cases}
      \text{ if } n<m & \Z/ p^n \Z \\
      \text{ otherwise } & \Z / p^m \Z
    \end{cases}
  \end{align*}
  
  The Ext functor for finitely generated groups allows us to pull out direct sums in both indices, so we have

  \begin{align*}
    & Ext_{\Z}^i(\Z^r \oplus \bigoplus\limits_{j=1}^n \Z/p_j^{r_j}\Z, \Z^l
    \oplus \bigoplus\limits_{j=1}^n \Z/q_j^{l_j}\Z) \\
    &= Ext_{\Z}^i(\Z^r, \Z^l) \oplus Ext_{\Z}^i(\Z^r, \bigoplus \Z/q_j^{l_j}\Z) \oplus Ext_{\Z}^i(\bigoplus\limits_{j=1}^n \Z/p_j^{r_j}\Z, \Z^l) \oplus Ext_{\Z}^i(\bigoplus\limits_{j=1}^n \Z/p_j^{r_j}\Z,\bigoplus\limits_{j=1}^n \Z/q_j^{l_j}\Z)
  \end{align*}
  We have that $Ext_{\Z}^i(\Z^r, \Z^l) = 0$, since for example we have the following projective resolution of $\Z^r$:
  \begin{align*}
    0 \xrightarrow{0} \Z^r \xrightarrow{id} \Z^r \xrightarrow{id} \Z^r \xrightarrow{0} 0 
  \end{align*}
  which we turn into
  \begin{align*}
    0 \xrightarrow{0} \hom(\Z^r,\Z^l) \xrightarrow{id_*} \hom(\Z^r, \Z^l) \xrightarrow{0} 0     
  \end{align*}
  which is equivalent to 
  \begin{align*}
    0 \xrightarrow{0} \Z^{rl} \xrightarrow{id} \Z^{rl} \xrightarrow{0} 0.
  \end{align*}
  Thus 
  \begin{align*}
    Ext_{\Z}^0(\Z^r, \Z^l) = ker(id)/im(0) = 0, \quad Ext_{\Z}^1(\Z^r, \Z^l) = ker(0)/im(id) = \Z^{rl}/\Z^{rl} = 0.
  \end{align*}

  We also have that $Ext_{\Z}^i(\Z^r, \bigoplus \Z/q_j^{l_j}\Z) = 0$ for $i=0,1$, since
  \begin{align*}
    Ext_{\Z}^i(\Z^r, \bigoplus \Z/q_j^{l_j}\Z)
    &= (Ext_{\Z}^i(\Z, \bigoplus \Z/q_j^{l_j}\Z))^r \\
    &= (\bigoplus Ext_{\Z}^i(\Z,\Z/q_j^{l_j}\Z))^r.
  \end{align*}
  We have the following projective resolution of $\Z$:
  \begin{align*}
    0 \xrightarrow{0} \Z \xrightarrow{id} \Z \xrightarrow{id} \Z \xrightarrow{0} 0,
  \end{align*}
  and then we apply $\hom$ to this to get
  \begin{align*}
    0 \xrightarrow{0} \hom(\Z, \Z/q_j^{l_j}\Z) \xrightarrow{id_*} \hom(\Z, \Z/q_j^{l_j}\Z) \xrightarrow{0} 0,    
  \end{align*}
  which is the same as 
  \begin{align*}
    0 \xrightarrow{0} \Z/q_j^{l_j}\Z \xrightarrow{id} \Z/q_j^{l_j}\Z \xrightarrow{0} 0.
  \end{align*}
  Thus, we get
  \begin{align*}
    Ext_{\Z}^i(\Z,\Z/q_j^{l_j}\Z) = 0,
  \end{align*}
  and this holds for each $j$. Thus,
  \begin{align*}
    Ext_{\Z}^i(\Z^r, \bigoplus \Z/q_j^{l_j}\Z) = 0.
  \end{align*}

  Next, we have that $Ext(\bigoplus \Z/p_j^{r_j} \Z, \Z^l) =\bigoplus Ext( \Z/p_j^{r_j} \Z, \Z^l) = 0$, since for the projective resolution
  \begin{align*}
    0 \xrightarrow{0} \Z \xrightarrow{p_j^{r_j}} \Z \xrightarrow{\pi_{p_j^{r_j}}} \Z / p_j^{r_j} \Z \xrightarrow{0} 0 
  \end{align*}
  which we turn into 
  \begin{align*}
    0 \xrightarrow{0} \hom(\Z, \Z^l) \xrightarrow{p_j^{r_j}_*} \hom(\Z, \Z^l) \xrightarrow{0} 0 
  \end{align*}
  which is equivalent to
  \begin{align*}
    0 \xrightarrow{0} \Z^l \xrightarrow{p_j^{r_j}} \Z^l \xrightarrow{0} 0;
  \end{align*}
  and so
  \begin{align*}
    Ext_{\Z}^0(\Z/p_j^{r_j} \Z, \Z^l)
    &= \ker(p_j^{r_j})/\im(0) \\
    &= 0
  \end{align*}
  and
  \begin{align*}
    Ext_{\Z}^1(\Z/p_j^{r_j} \Z, \Z^l)
    &= \ker(0)/\im(p_j^{r_j}) \\
    &= \Z^l/\Z^l \\
    &= 0 
  \end{align*}

  Thus,
  \begin{align*}
    &Ext_{\Z}^i(\Z^r \oplus \bigoplus\limits_{j=1}^n \Z/p_j^{r_j}\Z, \Z^l
    \oplus \bigoplus\limits_{j=1}^n \Z/q_j^{l_j}\Z) \\
    &= Ext_{\Z}^i(\bigoplus\limits_{j=1}^n \Z/p_j^{r_j}\Z,\bigoplus\limits_{j=1}^n \Z/q_j^{l_j}\Z) \\
    &= \bigoplus\limits_{i,j}^{n.m} Ext_{\Z}^i( \Z/p_j^{r_j}\Z, \Z/q_j^{l_j}\Z).
  \end{align*}

  Now, if $p_i \neq q_j$ then
  \begin{align*}
    Ext_{\Z}^i( \Z/p_j^{r_j}\Z, \Z/q_j^{l_j}\Z) = 0,
  \end{align*}
  since for the following projective resolution of $\Z/p_j^{r_j}\Z$:
  \begin{align*}
    0 \xrightarrow{0} \Z \xrightarrow{p_i^{r_i}} \Z \xrightarrow{\pi_{p_i^{r_i}}} \Z/p_i^{r_i}\Z \xrightarrow{0} 0
  \end{align*}
  we have
  \begin{align*}
    0 \xrightarrow{0} \hom_{\Z}(\Z, \Z/q_j^s_j\Z) \xrightarrow{p_i^{r_i}} \hom_{\Z}(\Z, \Z/q_j^s_j\Z) \xrightarrow{0} 0
  \end{align*}
  which is equivalent to
  \begin{align*}
    0 \xrightarrow{0} \Z/q_j^s_j\Z \xrightarrow{p_i^{r_i}} \Z/q_j^s_j\Z \xrightarrow{0} 0 .
  \end{align*}
  Thus, we find just searching out the homology functors that since $p_i^{r_i}$ and $q_j^{s_j}$ are coprime that
  \begin{align*}
    H_0 = ker(p_i^{r_i})/im(0) = 0/0 = 0
  \end{align*}
  and
  \begin{align*}
    H_1 = ker(0)/im(0) = (\Z/q_j^{s_j}\Z)/(\Z/q_j^{s_j}) = 0.
  \end{align*}

  Hence, after a reindexing of terms we have that
  \begin{align*}
    &Ext_{\Z}^i(\Z^r \oplus \bigoplus\limits_{j=1}^n \Z/p_j^{r_j}\Z, \Z^l
    \oplus \bigoplus\limits_{j=1}^n \Z/q_j^{l_j}\Z) \\
    &= \bigoplus\limits_{j}^{n} Ext_{\Z}^i( \Z/p_j^{r_j}\Z, \Z/q_j^{l_j}\Z).
  \end{align*}
  Thus,
  \begin{align*}
    &Ext_{\Z}^0(\Z^r \oplus \bigoplus\limits_{j=1}^n \Z/p_j^{r_j}\Z, \Z^l
    \oplus \bigoplus\limits_{j=1}^n \Z/q_j^{l_j}\Z) \\
    &= \bigoplus\limits_{j}^{n}  N_j
  \end{align*}
  where
  \begin{align*}
    N_j =
    \begin{cases}
      \text{ if } s_j<r_j & p_j^{r_j-s_j}\Z/p_j^{r_j} \Z \\
      \text{ otherwise } & \Z/p_j^{r_j} \Z
    \end{cases}
  \end{align*}
  and
  \begin{align*}
    &Ext_{\Z}^1(\Z^r \oplus \bigoplus\limits_{j=1}^n \Z/p_j^{r_j}\Z, \Z^l
    \oplus \bigoplus\limits_{j=1}^n \Z/q_j^{l_j}\Z) \\
    &= \bigoplus\limits_{j}^{n} M_j,
  \end{align*}
  where
  \begin{align*}
    M_j =
    \begin{cases}
      \text { if } r_j < s_j & \Z/p_j^{r_j} \Z \\
      \text { otherwise} & \Z/p_j^{s_j} \Z 
    \end{cases}
  \end{align*}
  
  
  The Tor functor preserves direct sums in the first argument and so we get
  \begin{align*}
    Tor_{\Z}^i(\Z^r \oplus \bigoplus\limits_{j=1}^n \Z/p_j^{r_j}\Z, \Z/q^s\Z) 
    &= Tor_{\Z}^i(\Z, \Z/q^s\Z)^r \oplus \bigoplus\limits_{j=1}^n Tor_{\Z}^i(\Z/p_j^{r_j}\Z, \Z/q^s\Z)
  \end{align*}
  We have the following projective resolution of $\Z$
  \begin{align*}
    0 \xrightarrow{0} \Z \xrightarrow{id} \Z \xrightarrow{id} \Z \xrightarrow{0} 0
  \end{align*}
  to which we apply $(-) \otimes_{\Z} (\Z/q^s\Z)$ to get
  \begin{align*}
    0 \xrightarrow{0_*} \Z \otimes_{\Z} (\Z/q^s\Z) \xrightarrow{id_*} \Z \otimes_{\Z} (\Z/q^s\Z) \xrightarrow{0_*} 0    
  \end{align*}
  which is equal to 
  \begin{align*}
    0 \xrightarrow{0} \Z/q^s\Z \xrightarrow{id} \Z/q^s\Z \xrightarrow{0} 0.
  \end{align*}
  From which we get
  \begin{align*}
    Tor_{\Z}^0(\Z, \Z/q^s\Z) = \ker(id)/\im(0) = 0
  \end{align*}
  and
  \begin{align*}
    Tor_{\Z}^1(\Z, \Z/q^s\Z) = \ker(0)/\im(id) = (\Z/q^s\Z)/(\Z/q^s\Z) = 0    
  \end{align*}
  Now, if $p \neq q$, then $p^{r_j}$ and $q^s$ are coprime. We have the following projective resolution of $\Z/p_j^{r_j}\Z$:
  \begin{align*}
    0 \xrightarrow{0} \Z \xrightarrow{p_j^{r_j}} \Z \xrightarrow{\pi_{p_j^{r_j}}} \Z/p_j^{r_j}\Z \xrightarrow{0} 0,
  \end{align*}
  to which we apply $(-) \otimes_{\Z} (\Z/q^s \Z)$ (while remembering to drop the $\Z/p_j^{r_j}\Z$ term), to get
  \begin{align*}
    0 \xrightarrow{0_*} \Z \otimes_{\Z} (\Z/q^s \Z) \xrightarrow{p_j^{r_j}_*} \Z \otimes_{\Z} (\Z/q^s \Z) \xrightarrow{0_*} 0,    
  \end{align*}
  which is equivalent to
  \begin{align*}
    0 \xrightarrow{0} \Z/q^s \Z \xrightarrow{p_j^{r_j}} \Z/q^s \Z \xrightarrow{0} 0.
  \end{align*}
  Thus
  \begin{align*}
    Tor_{\Z}^0(\Z/p_j^{r_j}\Z, \Z/q^s\Z) &= \ker(p_j^{r_j})/\im(0) \\
    &= 0/0 \\
    &= 0
  \end{align*}
  and
  \begin{align*}
    Tor_{\Z}^1(\Z/p_j^{r_j}\Z, \Z/q^s\Z) &= \ker(0)/\im(p_j^{r_j}) \\
    &= (\Z/q^s\Z)/(\Z/q^s\Z) \\
    &= 0 
  \end{align*}
  \begin{align*}
    Tor_{\Z}^i(\Z/p_j^{r_j}\Z, \Z/q^s\Z) = 0
  \end{align*}
  Thus, we have
  \begin{align*}
    &Tor_{\Z}^i(\Z^r \oplus \bigoplus\limits_{j=1}^n \Z/p_j^{r_j}\Z, \Z/q^s\Z)  \\
    &= Tor_{\Z}^i(\Z, \Z/q^s\Z)^r \oplus \bigoplus\limits_{j=1}^n Tor_{\Z}^i(\Z/p_j^{r_j}\Z, \Z/q^s\Z) \\
    &= \bigoplus\limits_{j=1}^n Tor_{\Z}^i(\Z/p_j^{r_j}\Z, \Z/q^s\Z) \\
    &= 
    \begin{cases}
      \text { if } \forall j, q \neq p_j & 0 \\
      \text { otherwise } & Tor_{\Z}^i(\Z/p_j^{r_j}\Z, \Z/p_j^s\Z)
    \end{cases}
  \end{align*}
  where $p_j = q$. 
  Thus, we have
  \begin{align*}
    &Tor_{\Z}^i(\Z^r \oplus \bigoplus\limits_{j=1}^n \Z/p_j^{r_j}\Z, \Z/q^s\Z)  \\
    &=
    \begin{cases}
      \text { if } \forall j, q \neq p_j & 0 \\
      \text { otherwise } & N_i
    \end{cases}
  \end{align*}
  where
  \begin{align*}
    N_0 &=
    \begin{cases}
      \text{ if } s<r_j & p^{r_j-s}\Z/ p^{r_j} \Z \\
      \text{ otherwise } & \Z / p^{r_j} \Z
    \end{cases}
  \end{align*}
  and
  \begin{align*}
    N_1 &=
    \begin{cases}
      \text{ if } s<r_j & \Z/ p^s \Z \\
      \text{ otherwise } & \Z / p^{r_j} \Z      
    \end{cases}
  \end{align*}
\end{proof}
\end{document}
