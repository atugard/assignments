\documentclass[12pt]{extarticle}
%Some packages I commonly use.
\usepackage[english]{babel}
\usepackage{graphicx}
\usepackage{framed}
\usepackage[normalem]{ulem}
\usepackage{amsmath}
\usepackage{amsthm}
\usepackage{amssymb}
\usepackage{amsfonts}
\usepackage{enumerate}
\usepackage[utf8]{inputenc}
\usepackage[top=1 in,bottom=1in, left=1 in, right=1 in]{geometry}

%A bunch of definitions that make my life easier
\newcommand{\matlab}{{\sc Matlab} }
\newcommand{\abs}[1]{|#1|}
\newcommand{\set}[1]{\{#1\}}
\newcommand{\cvec}[1]{{\mathbf #1}}
\newcommand{\rvec}[1]{\vec{\mathbf #1}}
\newcommand{\ihat}{\hat{\textbf{\i}}}
\newcommand{\jhat}{\hat{\textbf{\j}}}
\newcommand{\khat}{\hat{\textbf{k}}}
\newcommand{\minor}{{\rm minor}}
\newcommand{\trace}{{\rm trace}}
\newcommand{\spn}{{\rm Span}}
\newcommand{\rem}{{\rm rem}}
\newcommand{\ran}{{\rm range}}
\newcommand{\range}{{\rm range}}
\newcommand{\mdiv}{{\rm div}}
\newcommand{\proj}{{\rm proj}}
\newcommand{\R}{\mathbb{R}}
\newcommand{\N}{\mathbb{N}}
\newcommand{\Q}{\mathbb{Q}}
\newcommand{\Z}{\mathbb{Z}}
\newcommand{\<}{\langle}
\renewcommand{\>}{\rangle}
\renewcommand{\emptyset}{\varnothing}
\newcommand{\attn}[1]{\textbf{#1}}
\theoremstyle{definition}
\newtheorem{theorem}{Theorem}
\newtheorem{prob}{Problem}
\newtheorem{corollary}{Corollary}
\newtheorem*{definition}{Definition}
\newtheorem*{example}{Example}
\newtheorem*{note}{Note}
\newtheorem{exercise}{Exercise}
\newcommand{\bproof}{\bigskip {\bf Proof. }}
\newcommand{\eproof}{\hfill\qedsymbol}
\newcommand{\Disp}{\displaystyle}
\newcommand{\qe}{\hfill\(\bigtriangledown\)}
\setlength{\columnseprule}{1 pt}


\title{ Math 5107 Algebra I - Assignment 1}
\author{David Draguta}

\begin{document}

\maketitle

\begin{exercise}
  Let $k$ be a field and let $R = k^{n \times n}$. When is the left $R$-module $k^{n \times r}$ a cyclic module? A free module? A simple module? 
\end{exercise}
\begin{proof}
  $k^{n \times r}$ is a cyclic module when for all $c \in k$ there exist $\mathbf{a}, \mathbf{b} \in k^n$ such that
  $c = \mathbf{a} \cdot \mathbf{b}$. The module is free if it has a basis, i.e. if there exist spanning elements $E_1, \dots, E_m \in k^{n \times r}$ such that if
  $\sum\limits_{s=1}^{m} A_s E_s = 0 \implies A_s = 0$ for all $s$. Let $E_s = (e_{ij}^s)$ and $A_s = (a_{ij}^s)$. Then this condition is equivalent to
  
  \begin{align*}
    \sum\limits_s \sum\limits_{l}a_{il}^s e_{lj}^s = 0 \text{ for all } i, j  \implies a_{il}^s = 0, \text { for all } i,j, s.
  \end{align*}

\end{proof}
\begin{exercise}
  Show that $\Q$ is not finitely generated as a $\Z$ module. Show that $_\Z\Q$ does not have a maximal submodule.
\end{exercise}

\begin{proof}
  Suppose $\Q$ is generated by $a_1/b_1, \dots, a_n/b_n$. Then for all rational numbers $q \in \Q$ we have $q(b_1 \cdots b_n) \in \Z$, which is clearly not true; you can just take $(b_1 \cdots b_n)^{-2}$ as a counter example.
\end{proof}

\begin{exercise}
  Let $\phi: R \to S$ be a ring homomorphism of commutative rings. Let $N$ be an $R$-module and $M$ an $S$-module.
  \begin{enumerate}
  \item
    Show that we can give $M$, the structure of a $R$-module that we will call $\phi_*(M)$ with scalar multiplication given by $r \cdot m := \phi(r)m$ for $r \in R$, $m \in M$.
  \item
    Show that $\phi^!(N) := Hom_R(S,N)$ has a natural structure as an $S$-module.
  \item
    Show that there is a natural isomorphism of $R$-modules $Hom_S(M, \phi^!(N)) \cong Hom_R(\phi_*(M), N)$.
  \end{enumerate}
\end{exercise}
\begin{proof}
  \begin{enumerate}
  \item
    We check the module axioms for $\phi_*(M)$:
    \begin{enumerate}
    \item
      $r \cdot (x + y) = \phi(r)(x+y) = \phi(r)x + \phi(r)y = r \cdot x + r \cdot y $
    \item
      $(r+s) \cdot x = \phi(r+s) \cdot x = (\phi(r) + \phi(s)) \cdot x = \phi(r)x + \phi(s)x = r \cdot x + s \cdot x$
    \item
      $rs \cdot x = \phi(rs)x = (\phi(r)\phi(s))x = \phi(r)(\phi(s)x) = r \cdot (s \cdot x)$
    \item
      $1_R \cdot x = \phi(1_R)x = 1_Sx = x$
    \end{enumerate}
  \item
    We define scalar multiplication in $\phi^!(N)$ by $(sT)(s') = T(ss')$
  \item
    Given $\alpha \in Hom_R(M, \phi^!(N))$, we define $\beta &: Hom_S(M, \phi^!(N)) \to Hom_R(\phi_*(M), N)$ by 
    \begin{align*}
      \beta(\alpha)(\sum\limits_{i=1}^n r_i \cdot m_i) = \sum\limits_{i=1}^n \alpha(m_i)(\phi(r_i)).
    \end{align*}
  \end{enumerate}
  We check that it's $R$-linear:
  \begin{align*}
    \beta(r_1'\alpha_1 + r_2' \alpha_2)(\sum\limits_{i=1}^n r_i \cdot m_i)
    &= \sum\limits_{i=1}^n (r_1'\alpha_1 + r_2' \alpha_2)(m_i)(\phi(r_i)) \\
    &= \sum\limits_{i=1}^n (r_1'\alpha_1)(m_i)(\phi(r_i)) + \sum\limits_{i=1}^n (r_2'\alpha_2)(m_i)(\phi(r_i)) \\
    &= r_1'\sum\limits_{i=1}^n (\alpha_1)(m_i)(\phi(r_i)) + r_2'\sum\limits_{i=1}^n (\alpha_2)(m_i)(\phi(r_i)) \\
    &= r_1' \beta(\alpha_1)(\sum\limits_{i=1}^n r_i \cdot m_i) + r_2'\beta(\alpha_2)(\sum\limits_{i=1}^n r_i \cdot m_i)   \\
    &= (r_1' \beta(\alpha_1) + r_2'\beta(\alpha_2))(\sum\limits_{i=1}^n r_i \cdot m_i),
  \end{align*}
  and so
  \begin{align*}
    \beta(r_1'\alpha_1 + r_2' \alpha_2) = r_1' \beta(\alpha_1) + r_2'\beta(\alpha_2).
  \end{align*}
\end{proof}
\begin{exercise}
  Let $R$ be a commutative ring. Let $M, N$ be $R$-modules with finite presentations
  \begin{align*}
    R^{m_1} \overset{A} \longrightarrow R^{m_0} \longrightarrow M \longrightarrow 0  \\
    R^{n_1} \overset{B} \longrightarrow R^{n_0} \longrightarrow N \longrightarrow 0,
  \end{align*}
  with $A \in R^{m_0 \times m_1}$ and $B \in R^{n_0 \times n_1}$. Show that there's a map of $R$-modules
  \begin{align*}
    \set{(C,D) \in R^{n_1 \times m_1} \oplus R^{n_0 \times m_0} | DA=BC} \to Hom_R(M,N).
  \end{align*}
  Is this map surjective? 
\end{exercise}
\begin{proof}
  Let $\pi: R^{m_0} \to R^{m_0}/AR^{m_1} = M$ and $\pi': R^{n_0} \to R^{n_0}/BR^{n_1} = N$. We define our map $\varphi: M \to N$ as follows. Let $m \in M$.
  Then as $\pi$ is surjective, there exists some $x \in R^{m_0}$ such that $\pi(x) = m $. We define $\phi(m) = \pi'(Dx)$, and check that this is
  a well-defined $R$-module morphism.
  \begin{itemize}
  \item
    \underline{Well-defined}: Suppose we have lifts $x, x' \in R^{m_0}$ such that $\pi(x) = \pi(x')$. Then $x-x' \in ker(\pi)=AR^{m_1} \subseteq D^{-1}BCR^{m_1} \subseteq D^{-1} B R^{n_1}$, i.e. $Dx-Dx' \in BR^{n_1}$, where $D^{-1}$ is the preimage, and not the inverse, since the inverse doesn't generally exist.
    Now $ \pi'(Dx) = \pi'(Dx') \iff Dx - Dx' \in ker(\pi') = BR^{n_1}$, which as we've just shown holds, so $\phi$ is well defined.
  \item{R-linear}: Let $m_1, m_2 \in R$ with lifts $x_1, x_2 \in R^{m_0}$ and $r_1, r_2 \in R$, then
    \begin{align*}
      \phi(r_1 \cdot m_1 + r_2 \cdot m_2) &= \pi'D(r_1 \cdot x_1 + r_2 \cdot x_2) = \pi'(r_1D(x_1) + r_2D(x_2)) \\
      &= r_1 \pi'D(x_1) + r_2 \pi'D(x_2)  \\
      &= r_1 \phi(m_1) + r_2 \phi(m_2).
    \end{align*}
  \end{itemize}
  This mapping is not surjective, since for any
\end{proof}
\begin{exercise}
  Use Smith Normal Form to simplify the description of these modules given by the presentations
  \begin{enumerate}
  \item
    Over $\Z$ with presentation
    \begin{center}
      \begin{pmatrix}
        2 & -1 & 0 \\
        -1 & 2 & -1 \\
        0 & -1 & 2
      \end{pmatrix}
    \end{center}
  \item
    Over $\Q[x]$ with presentation
    \begin{center}
      \begin{pmatrix}
        x-1 & x  \\
        x^2+1 & x^2 
      \end{pmatrix}
    \end{center}
  \end{enumerate}
\end{exercise}
\begin{proof}
  \begin{enumerate}
  \item
    \begin{align*}
      \begin{pmatrix}
        2 & -1 & 0 \\
        -1 & 2 & -1 \\
        0 & -1 & 2
      \end{pmatrix} \sim
      \begin{pmatrix}
        1 & 0 & 0 \\
        0 & 1 & 0 \\
        0 & 0 & 4
      \end{pmatrix}
    \end{align*}
    so $M \cong \Z / 4 \Z$.
  \item
    \begin{align*}
      \begin{pmatrix}
        x-1 & x  \\
        x^2+1 & x^2 
      \end{pmatrix} \sim
      \begin{pmatrix}
        1 & 0  \\n
        0 & x^2+x 
      \end{pmatrix}
    \end{align*}
    so $M \cong \Q[x]/(x^2+x)$.
  \end{enumerate}
\end{proof}
\begin{exercise}
  Consider the module $M = \Q^4$ over $\Q[x]$ described by
  \begin{center}
    \begin{align*}
      \begin{pmatrix}
        1 & -1 & 1 & 0 \\
        1 & 1 & 0 & 1 \\
        0 & 0 & 1 & -1 \\
        0 & 0 & 1 & 1
      \end{pmatrix}
    \end{align*}
  \end{center}
  Find a simple representation for $M$.
\end{exercise}
\begin{proof}
  \begin{align*}
    \begin{pmatrix}
      1 & -1 & 1 & 0 \\
      1 & 1 & 0 & 1 \\
      0 & 0 & 1 & -1 \\
      0 & 0 & 1 & 1
    \end{pmatrix}
    \sim
    \begin{pmatrix}
      1 & -1 & 1 & 0 \\
      1 & 1 & 0 & 1 \\
      0 & 0 & 1 & -1 \\
      0 & 0 & 1 & 1
    \end{pmatrix}
  \end{align*}
\end{proof}
\end{document}
