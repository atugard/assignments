\documentclass[12pt]{extarticle}
%Some packages I commonly use.
\usepackage[english]{babel}
\usepackage{graphicx}
\usepackage{framed}
\usepackage[normalem]{ulem}
\usepackage{amsmath}
\usepackage{amsthm}
\usepackage{amssymb}
\usepackage{amsfonts}
\usepackage{enumerate}
\usepackage[utf8]{inputenc}
\usepackage[top=1 in,bottom=1in, left=1 in, right=1 in]{geometry}

%A bunch of definitions that make my life easier
\newcommand{\matlab}{{\sc Matlab} }
\newcommand{\abs}[1]{|#1|}
\newcommand{\set}[1]{\{#1\}}
\newcommand{\cvec}[1]{{\mathbf #1}}
\newcommand{\rvec}[1]{\vec{\mathbf #1}}
\newcommand{\ihat}{\hat{\textbf{\i}}}
\newcommand{\jhat}{\hat{\textbf{\j}}}
\newcommand{\khat}{\hat{\textbf{k}}}
\newcommand{\minor}{{\rm minor}}
\newcommand{\trace}{{\rm trace}}
\newcommand{\spn}{{\rm Span}}
\newcommand{\rem}{{\rm rem}}
\newcommand{\ran}{{\rm range}}
\newcommand{\range}{{\rm range}}
\newcommand{\mdiv}{{\rm div}}
\newcommand{\proj}{{\rm proj}}
\newcommand{\R}{\mathbb{R}}
\newcommand{\N}{\mathbb{N}}
\newcommand{\Q}{\mathbb{Q}}
\newcommand{\Z}{\mathbb{Z}}
\newcommand{\<}{\langle}
\renewcommand{\>}{\rangle}
\renewcommand{\emptyset}{\varnothing}
\newcommand{\attn}[1]{\textbf{#1}}
\theoremstyle{definition}
\newtheorem{theorem}{Theorem}
\newtheorem{prob}{Problem}
\newtheorem{corollary}{Corollary}
\newtheorem*{definition}{Definition}
\newtheorem*{example}{Example}
\newtheorem*{note}{Note}
\newtheorem{exercise}{Exercise}
\newcommand{\bproof}{\bigskip {\bf Proof. }}
\newcommand{\eproof}{\hfill\qedsymbol}
\newcommand{\Disp}{\displaystyle}
\newcommand{\qe}{\hfill\(\bigtriangledown\)}
\setlength{\columnseprule}{1 pt}


\title{ Math 5205 -- Topology Assignment 1}
\author{David Draguta}
\date{2021-09-25

\begin{document}

\maketitle

\begin{exercise}
  Let $ \overline \R = \R \cup \set{ - \infty , \infty }$ (this is called the extended set of real numbers).
  \begin{enumerate}
  \item Prove that the function $d: \overline \R \times \overline \R \to [0, \infty)$, given by
    \begin{center}
       $d(x,y) = \abs{\arctan(x) - \arctan(y)}$,
    \end{center}
    where $\arctan(\pm \infty) = \pm \pi / 2$, is a metric on $\overline \R$.
  \item Let $(x_n)_{n=1}^{\infty}$ be a sequence in $\overline \R $ and $x \in \overline \R $. Prove the following:
    \begin{enumerate}
    \item
      If $x \in \R$ then $(x_n)_{n=1}^{\infty}$ converges to $x$ in the metric space $(\overline \R, d)$ iff for every $\epsilon > 0$ there exists $N \in \N$ such that $x_n \in (x - \epsilon, x + \epsilon)$ for all $n \ge N$.
    \item
      If $x = \infty$ then $(x_n)_{n=1}^{\infty}$ converges to $x$ in the metric space $(\overline \R, d)$ iff for every $M \in \R$ there exists $N \in \N$ such that $x_n \in (M, \infty]$ for all $n \ge N$. 
    \end{enumerate}
  \end{enumerate}
\end{exercise}
\begin{proof}
  %remember to check \pm infinity points too
  \begin{enumerate}
  \item
    \begin{itemize}
    \item
      If $d(x,y)=0$ then $\arctan(x) = \arctan(y)$. Since $\arctan$ is injective over the extended reals, we have $x=y$.
    \item
      $d(x,y) = \abs{\arctan(x) - \arctan(y)} = \abs{\arctan(y) - \arctan(x)} = d(y,x)$ holds for any real numbers $x$ and $y$.
    \item
      Applying the triangle inequality for real numbers $\arctan(x),\arctan(y),\arctan(z)$ we get
      \begin{align*}
        d(x,z) = \abs{\arctan(x) - \arctan(z)}
        &= \abs{\arctan(x) - \arctan(y) + \arctan(y) - \arctan(z)} \\
        &\leq  \abs{ \arctan(x) - \arctan(y)} + \abs{ \arctan{y} - \arctan{z}} \\
        &= d(x,y) + d(y,z)
      \end{align*}
    \end{itemize}
  \item
    \begin{enumerate}
    \item
      $\Rightarrow$:
      The sequence $(\arctan(x_n))_{n=1}^{\infty}$ converges to $\arctan(x)$, which is not $\pm \pi/2$. We can fix a small enough $\delta>0$ and find a natural number $M$
      such that for all $n >
      M$, $\set{\pm \pi/2} \not \in (\arctan(x)-\delta, \arctan(x)+\delta)$ and $x_n \in (\arctan(x)-\delta, \arctan(x)+\delta) $.
      Let $y_n = \arctan(x_{n+M})$. Then $(y_n)_{n=0}^{\infty}$ is a sequence of elements in $\R$ converging to $y:=\arctan(x)$. Since $\tan$ is continuous on $(-\pi/2, \pi/2)$, we have $ y_n \to y$ implies $x_n = \tan(y_n) \to x = \tan(y)$. This means that for every $\epsilon > 0$ there exists some natural number $N$ such that for all $n>N, \abs{x_n-x} < \epsilon$, but this is equivalent to $x_n \in (x-\epsilon, x+ \epsilon)$ for all $n>N$. 
      
      $\Leftarrow$:
      %fill in details for full marks...
      The statement is equivalent to $x_n \to x$ as $n \to \infty$ , and by the continuity of $\arctan$ this implies that $\arctan(x_n) \to \arctan(x)$, which can be restated as $d(x_n, x) \to 0$ as $n \to \infty$, or $(x_n)_{n=1}^{\infty}$ converges to $x$ in $(\overline \R, d)$.
    \item
      $\Rightarrow$:

      Suppose there exists some constant $0<M<\infty$, such that for any natural number $K$ we can find some $n>K$ with $x_n \leq M$.
      Then, since $\arctan$ is non-decreasing we have $ \arctan{x_n} \leq \arctan{M} < \pi/2 $.
      Set $\epsilon = \cfrac{1}{2}(\pi/2 - \arctan(M)).$ Then for any natural number $N$ there exists some $n>N$ with
      $\abs{\arctan(x_n) - \pi/2} = \abs{\pi/2 - \arctan(x_n)} = 2 \epsilon > \epsilon$, which is a contradiction.

      $\Leftarrow$:

      We have $x_n \to \infty$ and $\arctan$ is strictly increasing with upper bound $\pi/2$, so that $ \arctan(x_n) \to \pi/2$, which is equivalent to $(x_n)_{n=1}^{\infty}$ converges to $\infty$ in $(\overline \R, d)$. 

      
     
    \end{enumerate}
  \end{enumerate}
\end{proof}

\begin{exercise}
  Let that $A$ and $B$ be closed disjoint subsets of a topological space $X$. Prove that
  $(A \cup B)^\circ = A^\circ \cup B^ \circ$ and $ Fr(A \cup B) = Fr(A) \cup Fr(B)$. 
\end{exercise}
\begin{proof}
  As $A,B$ are disjoint, we have that their interiors are too. If $x \in (A \cup B)^{\circ}$ we have some $U \in \mathcal{U}_x$ such that $U \subseteq A \cup B$.
  Since the intersection of two open sets is open we have that $U \cap A$, $U \cap B$ are open and they can't both be empty. Suppose without loss of generally
  that $U \cap A$ is not empty. Then $U \cap A \in \mathcal{U}_x$ and $U \cap A \subseteq A$ so that $x \in A^{\circ}$.

  Conversely, if $x \in A^{\circ} \cup B^{\circ}$, since the two are disjoint it has to be in one. Say w.l.o.g. that it's in $A^{\circ}$.
  Then, there exists some $U \in \mathcal{U}_x$ with $U \subseteq A \subseteq A \cup B$, so that $x \in (A \cup B)^{\circ}$.

  Next,
  \begin{align*}
    Fr(A \cup B)
    &= \overline{A \cup B} \setminus (A \cup B)^{\circ} \\
    &= \overline{A} \cup \overline{B} \setminus (A^{\circ} \cup B^{\circ}) \\
    &= A \cup B \setminus (A^{\circ} \cup B^{\circ}) \\
    &= ((A \cup B) \setminus A^{\circ}) \cap ((A \cup B) \setminus B^{\circ}) \\
    &= ((A \setminus A^{\circ}) \cup (B \setminus A^{\circ})) \cap ((A \setminus B^{\circ}) \cup (B \setminus B^{\circ})) \\
    &= (Fr(A) \cup B) \cap (A \cup Fr(B)) \\
    &= (Fr(A) \cap A) \cup (Fr(A) \cap Fr(B)) \cup (B \cap A) \cup (B \cap Fr(B)) \\
    &= Fr(A) \cup \emptyset \cup \emptyset \cup Fr(B) \\
    &= Fr(A) \cup Fr(B)
  \end{align*}
  %make sure to explain how that was done.
\end{proof}

\begin{exercise}
  Prove the claims in items 1 and 2 of Problem 2C on p.20 in the text. 
\end{exercise}

\begin{exercise}
  Let $(M, \rho)$ be a pseudometric space and $(M^*, \rho^*)$ denote the pseudometric identification, as in Problem 2C on p.20.

  \begin{enumerate}
  \item
    Let  $h: M \to M^*$ denote the mapping $h(x) = [x]$ (see part 3 of 2C). Prove that fort every $ x \in M$ and every $\epsilon > 0$:
    \begin{enumerate}
    \item
      $h(U_{\rho}(x, \epsilon)) = U_{\rho^*} (h(x), \epsilon)$, and 
    \item
      $h^{-1}(U_{\rho^*}(h(x), \epsilon)) = U_{\rho}(x, \epsilon)$.
    \end{enumerate}
  \item
    Conclude that  when $A \subseteq M$ is open, then so is $h(A)$; moreover $h^{-1}(h(A)) = A$ and $h (M \setminus A) = M^* \setminus h(A)$.
  \item
    Conclude that when $A \subseteq M$ is closed, then so is $h(A)$.
  \item
    Show by example that the claims made in Part 3 of 2C are false. 
  \end{enumerate}
\end{exercise}

\begin{exercise}
  Recall that when $(X, \tau)$ is a topological space and for every $x \in X$, $\mathcal{B}_x$ is a neighbourhood base at x, then:$
  \begin{itemize}
  \item (V1) $\forall V \in \mathcal{B}_x,  x \in V$,
  \item (V2) $\forall V_1, V_2 \in \mathcal{B}_x, \exists V_3 \in \mathcal{B}_x, V_3 \subseteq V_1 \cap V_2$,
  \item (V3) $\forall V \in \mathcal{B}_x, \exists V_0 \in \mathcal{B}_x, \forall y \in V_0, \exists W \in \mathcal{B}_y, W \subseteq V. ]$
  \end{itemize}
  \begin{enumerate}
  \item
    Prove that $\mathcal{B}_x \subseteq \tau$ iff: $\forall V \in \mathcal{B}_x, y \in V, \exists W \in \mathcal{B}_y, W \subseteq V$.
  \item
    Prove that if $X$ is any set and to each $x \in X$ a nonempty family $\mathcal{B}_x$ of subsets of $X$ is assigned so that the family $(\mathcal{B}__x)_{x \in X}$ satisfies conditions $(V1), (V2),$ and $(V3')$, then there exists a unique topology $\tau$ on $X$ for which $\mathcal{B}_x$ is a neighbourhood base at $x$ for each $x \in X$; moreover. $\mathcal{B}_x \subseteq \tau$ for each $x$.

    Hint: Avoid lengthy rewriting of a proof shown in class.

  \item
    Suppose that $\mathcal{B}_x$ is countable and $\mathcal{B}_{x^'}$ is another neighbourhood base at $x$. Prove that there exists a countable subcollection $\mathcal{B}_x^{''} \subseteq \mathcal{B}_x^{'}$ which is a neighourhood base at $x$. 
  \end{enumerate}
\end{exercise}

\begin{exercise}
  From the text: #4A, pp. 35, 36. Which intervals are closed sets in $\mathbb{E}$. 
\end{exercise}

\end{document}
