\documentclass[12pt]{extarticle}
%Some packages I commonly use.
\usepackage[english]{babel}
\usepackage{graphicx}
\usepackage{framed}
\usepackage[normalem]{ulem}
\usepackage{amsmath}
\usepackage{amsthm}
\usepackage{amssymb}
\usepackage{amsfonts}
\usepackage{enumerate}
\usepackage[utf8]{inputenc}
\usepackage[top=1 in,bottom=1in, left=1 in, right=1 in]{geometry}

%A bunch of definitions that make my life easier
\newcommand{\matlab}{{\sc Matlab} }
\newcommand{\abs}[1]{|#1|}
\newcommand{\set}[1]{\{#1\}}
\newcommand\restr[2]{{% we make the whole thing an ordinary symbol
  \left.\kern-\nulldelimiterspace % automatically resize the bar with \right
  #1 % the function
  \vphantom{\big|} % pretend it's a little taller at normal size
  \right|_{#2} % this is the delimiter
  }}
\newcommand{\cvec}[1]{{\mathbf #1}}
\newcommand{\rvec}[1]{\vec{\mathbf #1}}
\newcommand{\ihat}{\hat{\textbf{\i}}}
\newcommand{\im}{\text{im}}
\newcommand{\cok}{\text{cok}}
\newcommand{\jhat}{\hat{\textbf{\j}}}
\newcommand{\khat}{\hat{\textbf{k}}}
\newcommand{\minor}{{\rm minor}}
\newcommand{\trace}{{\rm trace}}
\newcommand{\spn}{{\rm Span}}
\newcommand{\rem}{{\rm rem}}
\newcommand{\ran}{{\rm range}}
\newcommand{\range}{{\rm range}}
\newcommand{\mdiv}{{\rm div}}
\newcommand{\proj}{{\rm proj}}
\newcommand{\R}{\mathbb{R}}
\newcommand{\C}{\mathbb{C}}
\newcommand{\F}{\mathbb{F}}
\newcommand{\N}{\mathbb{N}}
\newcommand{\Q}{\mathbb{Q}}
\newcommand{\Z}{\mathbb{Z}}
\newcommand{\<}{\langle}
\newcommand{\ideal}{\triangleleft}
\renewcommand{\>}{\rangle}
\renewcommand{\emptyset}{\varnothing}
\newcommand{\attn}[1]{\textbf{#1}}
\theoremstyle{definition}
\newtheorem{theorem}{Theorem}
\newtheorem{prob}{Problem}
\newtheorem{corollary}{Corollary}
\newtheorem*{definition}{Definition}
\newtheorem*{example}{Example}
\newtheorem*{note}{Note}
\newtheorem{exercise}{Exercise}
\newcommand{\bproof}{\bigskip {\bf Proof. }}
\newcommand{\eproof}{\hfill\qedsymbol}
\newcommand{\Disp}{\displaystyle}
\newcommand{\qe}{\hfill\(\bigtriangledown\)}
\setlength{\columnseprule}{1 pt}


\title{ Math 5107 -- Algebra Assignment 3}
\author{David Draguta}
\date{2021-11-11}

\begin{document}

\maketitle

\begin{exercise}
  Let $R=\C[x,y]$ and let $I=(y^2+2-x)$. Let $f=y^3-x^2$. Is $f \in I$? Describe $f \mod I$.
\end{exercise}
\begin{proof}
  We can write $f=y^3-x^2$ as
  \begin{align}
    f \equiv -x^2+xy-2y \mod I \not \equiv 0 \mod I ,
  \end{align}
  so that $f \not \in I$.

  Also we can write
  \begin{align*}
    f \equiv -y^4 + y^3 - 4y^2 - 4 \mod I ,
  \end{align*}
  i.e. as a polynomial in y.
\end{proof}
\begin{exercise}
  Let $R=\C[x,y]$ and let $I=(x-1,y)$ and $J=(x-y-1, y^2-x^3+1)$. Is $I \subseteq J$? Is $J \subseteq I$?
\end{exercise}
\begin{proof}
  We have
  \begin{align}
    x-y-1 = x-1 + y,
  \end{align}
  so that $x-y-1 \in I$.

  Also
  \begin{align}
    y^2-x^3+1 = -(x^2+x+1)(x-1) + yy,
  \end{align}
  so that $y^2-x^3+1 \in I$, too.

  Hence, we have $J \subseteq I$.

  $I \not \subseteq J$, since it's impossible for example to write $y$ as a linear combination of the generators of $J$

\end{proof}

\begin{exercise}
  In the ring $\C[x,y]$ consider the ideals $I = (y,y-x^2)$, $J= (x^2 - y^2)$. Find generators for
  \begin{enumerate}
  \item
    $I + J$
  \item
    $IJ$
  \item
    $I \cap J$
  \item
    $\sqrt{I}$
  \end{enumerate}
\end{exercise}

\begin{proof}

  \begin{enumerate}
  \item
    Since elements here are just sums of elements $f + g$ such that $f \in I$ and $g \in J$ we have
    \begin{align}
      I + J = (y,y-x^2,x^2-y^2)
    \end{align}
  \item
    Now, we have finite sums of the form $\sum\limits_i f_i g_i$ for $f_i \in I$ and $g_i \in J$, so our generators are all products of the generators of $I$ and $J$:
    \begin{align}
      IJ = (y(x^2-y^2), (y-x^2)(x^2-y^2))
    \end{align}
    
  \item
    We have that
    \begin{align*}
      x^2 - y^2 = -(y-1)y + -1(y-x^2),
    \end{align*}
    and so $J \subseteq I$, and
    \begin{align*}
      I \cap J = J = (x^2-y^2).
    \end{align*}
  \item
    We have that $\sqrt{I} = I$, and so
    \begin{align*}
      \sqrt{I} = (y, y-x^2)
    \end{align*}
  \end{enumerate}
\end{proof}

\begin{exercise}
  Describe the variety that is the intersection of the sphere $x^2+y^2+z^2 = 2$ and the cone $x^2+y^2=z^2$ as a union of irreducible varieties. Write the corresponding statement for their ideals
\end{exercise}
\begin{proof}
  The sphere is given by
  \begin{align*}
    S = V(x^2+y^2+z^2 - 2),
  \end{align*}
  and the cone by
  \begin{align*}
    C = V(x^2+y^2-z^2).
  \end{align*}
  The ideals generated by the equations defining these zero sets are prime, and so the varieties are irreducible. We have, moreover, 
  \begin{align*}
    S \cup C = V((x^2+y^2+z^2-2) \cap (x^2+y^2 - z^2)).
  \end{align*}
  If $(a,b,c) \in S \cup C$ then
  \begin{align*}
    a^2 + b^2 = c^2,
  \end{align*}
  and
  \begin{align*}
    a^2 + b^2 + c^2 = 2c^2 = 2,
  \end{align*}
  so that $c = \pm 1$; and we get circles of radius 1 on the $x,y$ plane at $z=-1$ and $z=1$.

  Using Macaulay2, this simplifies to
  \begin{align*}
    S \cup C = V(x^4+2x^2y^2 + y^4 - z^4 - 2x^2 - 2y^2 + 2z^2)
  \end{align*}
\end{proof}

\begin{exercise}
  Find the ideal in $\C[x,y]$ corresponding to the variety that is the hyperbola $xy=-1$ union the point $(-1,-1)$ in $\C^2$.
\end{exercise}
\begin{proof}
  The hyperbola is 
  \begin{align*}
    H = V(xy+1)
  \end{align*}
  and the point is
  \begin{align*}
    P = V((x+1,y+1))
  \end{align*}
  Then what we're after is
  \begin{align*}
    H \cup P = V((xy+1) \cap (x+1,y+1)).
  \end{align*}
  Using the Macaulay2 calculator, this simplifies to
  \begin{align*}
    H \cup P = V(xy^2 + xy + y + 1, x^2y + xy + x + 1)
  \end{align*}
\end{proof}
\begin{exercise}
  Let
  \begin{align*}
    \Z_{(3)} &= \set{a/b: a,b \in \Z, b \not \in 3\Z}, \\
    \C[x]_{(x)} &= \set{f/g: f,g \in \C[x], g(0) \neq 0}
  \end{align*}
  Show that these are local rings.
\end{exercise}

\begin{proof}
  It suffices to check that these have unique maximal ideals.

  The ideals to consider are 
  \begin{align*}
    \mathfrak{m}_1 = 3\Z_{(3)}
  \end{align*}
  and 
  \begin{align*}
    \mathfrak{m}_2 = x \C[x]_{(x)}.
  \end{align*}
  We take for granted that these are indeed ideals, and just check that they're maximal and unique. Suppose that
  \begin{align*}
    \mathfrak{m}_1 \subsetneq I \triangleleft 3 \Z_{(3)}.
  \end{align*}
  Let $a/b \in I \setminus \mathfrak{m}_1$.
\end{proof}

\begin{exercise}
  Find generators of the kernel of the map $\varphi^*: \C[x,y] \to \C[t]$ given by
  \begin{align*}
    \varphi^*(x) &= t^2, \\
    \varphi^*(y) &= t^3-t.
  \end{align*}
  Let $\varphi: \mathbb{A}^1 \to \mathbb{A}^2$ be the corresponding map on affine spaces. Find $\varphi^{-1}(V(y))$.
\end{exercise}
\begin{proof}
  We have
  \begin{align*}
    \varphi^*(y^2) 
    &= (t^3-t)^2 \\
    &= t^6 - 2t^4 + t^2 \\
    &= \varphi^*(x^3) -  2\varphi^*(x^2) + \varphi^*(x) \\
    &= \varphi^*(x^3 - 2x^2 + x)
  \end{align*}
  and so
  \begin{align*}
    x^3 - 2x^2 + x - y^2  \in \ker(\varphi^*).
  \end{align*}

\end{proof}
\begin{exercise}
  \begin{enumerate}
  \item
    Let $\nu: \mathbb{A}^2 \to \mathbb{A}^4$ be the polynomial mapping given by
    \begin{align*}
      \nu(x,y) = (x^3, x^2y, xy^2, y^3).
    \end{align*}
    Find the ideal of the image of $\nu$, i.e. $I(\nu(\mathbb{C}^2))$.
  \item
    Let $\sigma = xy + yz + zx.$ We define a map $p: \mathbb{A}^3 \to \mathbb{A}^4$ as
    \begin{align*}
      p(x,y,z)=(x\sigma, y\sigma, z\sigma, -xyz)
    \end{align*}
    Find the ideal of the image.
  \end{enumerate}
\end{exercise}

\end{document}
