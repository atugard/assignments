\documentclass[12pt]{extarticle}
%Some packages I commonly use.
\usepackage[english]{babel}
\usepackage{graphicx}
\usepackage{framed}
\usepackage[normalem]{ulem}
\usepackage{amsmath}
\usepackage{amsthm}
\usepackage{amssymb}
\usepackage{amsfonts}
\usepackage{enumerate}
\usepackage[utf8]{inputenc}
\usepackage[top=1 in,bottom=1in, left=1 in, right=1 in]{geometry}

%A bunch of definitions that make my life easier
\newcommand{\matlab}{{\sc Matlab} }
\newcommand{\abs}[1]{|#1|}
\newcommand{\set}[1]{\{#1\}}
\newcommand\restr[2]{{% we make the whole thing an ordinary symbol
  \left.\kern-\nulldelimiterspace % automatically resize the bar with \right
  #1 % the function
  \vphantom{\big|} % pretend it's a little taller at normal size
  \right|_{#2} % this is the delimiter
  }}
\newcommand{\cvec}[1]{{\mathbf #1}}
\newcommand{\rvec}[1]{\vec{\mathbf #1}}
\newcommand{\ihat}{\hat{\textbf{\i}}}
\newcommand{\im}{\text{im}}
\newcommand{\cok}{\text{cok}}
\newcommand{\jhat}{\hat{\textbf{\j}}}
\newcommand{\khat}{\hat{\textbf{k}}}
\newcommand{\minor}{{\rm minor}}
\newcommand{\trace}{{\rm trace}}
\newcommand{\spn}{{\rm Span}}
\newcommand{\rem}{{\rm rem}}
\newcommand{\ran}{{\rm range}}
\newcommand{\range}{{\rm range}}
\newcommand{\mdiv}{{\rm div}}
\newcommand{\proj}{{\rm proj}}
\newcommand{\R}{\mathbb{R}}
\newcommand{\C}{\mathbb{C}}
\newcommand{\F}{\mathbb{F}}
\newcommand{\N}{\mathbb{N}}
\newcommand{\Q}{\mathbb{Q}}
\newcommand{\Z}{\mathbb{Z}}
\newcommand{\<}{\langle}
\newcommand{\ideal}{\triangleleft}
\renewcommand{\>}{\rangle}
\renewcommand{\emptyset}{\varnothing}
\newcommand{\attn}[1]{\textbf{#1}}
\theoremstyle{definition}
\newtheorem{theorem}{Theorem}
\newtheorem{prob}{Problem}
\newtheorem{corollary}{Corollary}
\newtheorem*{definition}{Definition}
\newtheorem*{example}{Example}
\newtheorem*{note}{Note}
\newtheorem{exercise}{Exercise}
\newcommand{\bproof}{\bigskip {\bf Proof. }}
\newcommand{\eproof}{\hfill\qedsymbol}
\newcommand{\Disp}{\displaystyle}
\newcommand{\qe}{\hfill\(\bigtriangledown\)}
\setlength{\columnseprule}{1 pt}


\title{ Math 5107 -- Algebra Assignment 3}
\author{David Draguta}
\date{2021-11-11}

\begin{document}

\maketitle

\begin{exercise}
  Let $R=\C[x,y]$ and let $I=(y^2+2-x)$. Let $f=y^3-x^2$. Is $f \in I$? Describe $f \mod I$.
\end{exercise}
\begin{proof}
  We can write $f=y^3-x^2$ as
  \begin{align}
    f \equiv -x^2+xy-2y \mod I \not \equiv 0 \mod I ,
  \end{align}
  so that $f \not \in I$.

  Also we can write
  \begin{align*}
    f \equiv -y^4 + y^3 - 4y^2 - 4 \mod I ,
  \end{align*}
  i.e. as a polynomial in y.
\end{proof}
\begin{exercise}
  Let $R=\C[x,y]$ and let $I=(x-1,y)$ and $J=(x-y-1, y^2-x^3+1)$. Is $I \subseteq J$? Is $J \subseteq I$?
\end{exercise}
\begin{proof}
  We have
  \begin{align}
    x-y-1 = x-1 + y,
  \end{align}
  so that $x-y-1 \in I$.

  Also
  \begin{align}
    y^2-x^3+1 = -(x^2+x+1)(x-1) + yy,
  \end{align}
  so that $y^2-x^3+1 \in I$, too.

  Hence, we have $J \subseteq I$.
\end{proof}

\begin{exercise}
  In the ring $\C[x,y]$ consider the ideals $I = (y,y-x^2)$, $J= (x^2 - y^2)$. Find generators for
  \begin{enumerate}
  \item
    $I + J$
  \item
    $IJ$
  \item
    $I \cap J$
  \item
    $\sqrt{I}$
  \end{enumerate}
\end{exercise}
\begin{proof}
  \begin{enumerate}
  \item
    Since elements here are just sums of elements $f + g$ such that $f \in I$ and $g \in J$ we have
    \begin{align}
      I + J = (y,y-x^2,x^2-y^2)
    \end{align}
  \item
    Now, we have finite sums of the form $\sum\limits_i f_i g_i$ for $f_i \in I$ and $g_i \in J$, so our generators are all products of the generators of $I$ and $J$:
    \begin{align}
      IJ = (y(x^2-y^2), (y-x^2)(x^2-y^2))
    \end{align}
  \item
    
  \end{enumerate}
\end{proof}
\end{document}
