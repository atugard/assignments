\documentclass[12pt]{extarticle}
%Some packages I commonly use.
\usepackage[english]{babel}
\usepackage{graphicx}
\usepackage{framed}
\usepackage[normalem]{ulem}
\usepackage{amsmath}
\usepackage{amsthm}
\usepackage{amssymb}
\usepackage{amsfonts}
\usepackage{enumerate}
\usepackage[utf8]{inputenc}
\usepackage[top=1 in,bottom=1in, left=1 in, right=1 in]{geometry}

%A bunch of definitions that make my life easier
\newcommand{\matlab}{{\sc Matlab} }
\newcommand{\abs}[1]{|#1|}
\newcommand{\set}[1]{\{#1\}}
\newcommand{\cvec}[1]{{\mathbf #1}}
\newcommand{\rvec}[1]{\vec{\mathbf #1}}
\newcommand{\ihat}{\hat{\textbf{\i}}}
\newcommand{\jhat}{\hat{\textbf{\j}}}
\newcommand{\khat}{\hat{\textbf{k}}}
\newcommand{\minor}{{\rm minor}}
\newcommand{\trace}{{\rm trace}}
\newcommand{\spn}{{\rm Span}}
\newcommand{\rem}{{\rm rem}}
\newcommand{\ran}{{\rm range}}
\newcommand{\range}{{\rm range}}
\newcommand{\mdiv}{{\rm div}}
\newcommand{\proj}{{\rm proj}}
\newcommand{\R}{\mathbb{R}}
\newcommand{\N}{\mathbb{N}}
\newcommand{\Q}{\mathbb{Q}}
\newcommand{\Z}{\mathbb{Z}}
\newcommand{\<}{\langle}
\renewcommand{\>}{\rangle}
\renewcommand{\emptyset}{\varnothing}
\newcommand{\attn}[1]{\textbf{#1}}
\theoremstyle{definition}
\newtheorem{theorem}{Theorem}
\newtheorem{prob}{Problem}
\newtheorem{corollary}{Corollary}
\newtheorem*{definition}{Definition}
\newtheorem*{example}{Example}
\newtheorem*{note}{Note}
\newtheorem{exercise}{Exercise}
\newcommand{\bproof}{\bigskip {\bf Proof. }}
\newcommand{\eproof}{\hfill\qedsymbol}
\newcommand{\Disp}{\displaystyle}
\newcommand{\qe}{\hfill\(\bigtriangledown\)}
\setlength{\columnseprule}{1 pt}


\title{ Math 5205 -- Topology Assignment 3}
\author{David Draguta}
\date{2021-10-18}

\begin{document}

\maketitle

\begin{exercise}
  Let $(X_{\alpha})_{\alpha \in A}$ be a non-empty family of topological spaces and $\varphi: A \to A$ a bijection of $A$ onto $A$. Prove that the cartesian product
  $\prod\limits_{\alpha \in A}X_{\alpha}$ and $\prod\limits_{\alpha \in A}X_{\varphi{\alpha}}$ are homeomorphic. 
\end{exercise}
\begin{proof}
  We define $f: \prod\limits_{\alpha \in A}X_{\alpha} \to \prod\limits_{\alpha \in A}X_{\varphi(\alpha)}$ by $ f(x) = x \circ \varphi$ and $g: f: \prod\limits_{\alpha \in A}X_{\varphi(\alpha)} \to \prod\limits_{\alpha \in A}X_{\alpha}$ by $g(x') = x' \circ \varphi^{-1}$, and check that these are continuous inverses of each other. We first check that the two are inverses of each other (which shows they're bijections)
  \begin{align*}
    g(f(x))  &= g(x \circ \varphi)     = (x \circ \varphi) \circ \varphi^{-1} = x \circ (\varphi \circ \varphi^{-1}) \\
    &= x, 
  \end{align*}
  and
  \begin{align*}
    f(g(x')) &= f(x'\circ \varphi^{-1}) = (x' \circ \varphi^{-1}) \circ \varphi = x' \circ (\varphi^{-1} \circ \varphi) \\
    &= x'.
  \end{align*}
  Next, we use Thm 101 part 3 to show that $f,g$ are continuous. For each $\alpha \in A$ and $(x_{\alpha})_{\alpha \in A} \in \prod\limits_{\alpha \in A}X_{\alpha}$ we have 
  \begin{align*}
    (\pi_\alpha \circ f)((x_{\alpha})_{\alpha \in A}) &= \pi_\alpha(f((x_{\alpha})_{\alpha \in A})) = \pi_{\alpha}((x_{\alpha})_{\alpha \in A} \circ \varphi) = \pi_{\alpha}((x_{\varphi(\alpha)})_{\alpha \in A}) = x_{\varphi(\alpha)} \\
    &= \pi_{\varphi(\alpha)}(x_{\alpha}_{\alpha \in A}),
  \end{align*}
  and so
  \begin{align*}
    \pi_{\alpha} \circ f  = \pi_{\varphi(\alpha)}
  \end{align*}
  is continous for all $\alpha \in A$. We conclude that $f$ is continuous (with respect to the weak topology).

  Similarly for $\alpha \in A$ and $(x_{\varphi(\alpha)})_{\alpha \in A} \in \prod\limits_{\alpha \in A} X_{\varphi(\alpha)}$, we have
  \begin{align*}
    (\pi_{\alpha}\circ g)((x_{\varphi(\alpha)})_{\alpha \in A}) &= \pi_{\alpha}(g((x_{\varphi(\alpha)})_{\alpha \in A})) = \pi_{\alpha}((x_{\varphi(\alpha)})_{\alpha \in A} \circ \varphi^{-1}) = \pi_{\alpha}((x_{\alpha})_{\alpha \in A}) = x_{\alpha} \\
    &= \pi_{\varphi^{-1}(\alpha)}((x_{\varphi(\alpha)})_{\alpha \in A}),
  \end{align*}
  so
  \begin{align*}
    \pi_{\alpha}\circ g = \pi_{\varphi^{-1}(\alpha)}
  \end{align*}
  is continuous for all $\alpha \in A$.
  
  Thus both $f$ and $g$ are continuous, and the two spaces are homeomorphic. 
\end{proof}
\begin{exercise}
  Let $A = B \cup C$ where $A,B,C$ are nonempty sets with $B \cap C = \emptyset$ and let $(X_{\alpha})_{\alpha \in A}$ be a family of topological spaces.
  Prove that $\prod\limits_{\alpha \in A} X_{\alpha}$ is homeomorphic to $\prod\limits_{\lambda \in \Lambda} \prod\limits_{\alpha \in A_{\lambda}} X_{\alpha}$.
\end{exercise}
\begin{proof}
  We prove 8C part 1 in the textbook.
  
  Let 
  \begin{align*}
    X = \prod\limits_{\alpha \in A} X_{\alpha}
  \end{align*}
  and
  \begin{align*}
    Y= \prod_{\lambda \in \Lambda}Y_{\lambda},
  \end{align*}
  where $Y_{\lambda} = \prod_{\alpha \in A_{\lambda}} X_{\alpha}$.

  Next, let
  \begin{align*}
    &f: X \to Y, \quad (x_{\alpha})_{\alpha \in A} \mapsto ((x_{\alpha})_{\alpha \in A_{\lambda}})_{\lambda \in \Lambda}
  \end{align*}
  and
  \begin{align*}
    &g: Y \to X, \quad ((x_{\alpha})_{\alpha \in A_{\lambda}})_{\lambda \in \Lambda} \mapsto (x_{\alpha})_{\alpha \in A}.
  \end{align*}
  These are obviously inverses. It remains to check that they're continuous. We apply part 3 of theorem 101 in the course notes.
  \begin{align*}
    \pi_{\lambda}^Y \circ f ((x_{\alpha})_{\alpha \in A}) &= \pi_{\lambda}^Y (((x_{\alpha})_{\alpha \in A_{\lambda}})_{\lambda \in \Lambda}) = (x_{\alpha})_{\alpha \in A_{\lambda}} \\
    &= (\prod\limits_{\alpha \in A_{\lambda}}\pi_{\alpha}^X )((x_{\alpha})_{\alpha \in A}),
  \end{align*}
  and so
  \begin{align*}
    \pi_{\lambda}^Y \circ f = \prod\limits_{\alpha \in A_{\lambda}}\pi_{\alpha}^X.
  \end{align*}
  To show that
  \begin{align*}
    \prod\limits_{\alpha \in A_{\lambda}}\pi_{\alpha}^X: X \to Y_{\lambda}
  \end{align*}
  is continuous, again we apply part 3 of theorem 101:
  \begin{align*}
    \pi_{\alpha}^{Y_{\lambda}} \circ \prod\limits_{\alpha \in A_{\lambda}}\pi_{\alpha}^X((x_{\alpha})_{\alpha \in A}) =\pi_{\alpha}^{Y_{\lambda} ((x_{\alpha})_{\alpha \in A_{\lambda}}) = \pi_{\alpha}^X((x_{\alpha})_{\alpha \in A}),
  \end{align*}
  and so
  \begin{align*}
    \pi_{\alpha}^{Y_{\lambda}} \circ \prod\limits_{\alpha \in A_{\lambda}}\pi_{\alpha}^X = \pi_{\alpha}^X.
  \end{align*}
  Now, for all $\alpha \in A_{\lambda}, \pi_{\alpha}^X$ is continuous, and so $\prod\limits_{\alpha \in A_{\lambda}}\pi_{\alpha}^X$ is continuous, which is true for all $\lambda \in \Lambda$. Hence, $f$ is continuous.

  For the other direction, we have
  \begin{align*}
    \pi_{\alpha}^X \circ g (((x_{\alpha})_{\alpha \in A_{\lambda}})_{\lambda \in \Lambda}) &= \pi_{\alpha}^X((x_{\alpha})_{\alpha \in A}) = x_{\alpha} \\
    &= \pi_{\alpha}^{Y_{\lambda}} \circ \pi_{\lambda}^Y (((x_{\alpha})_{\alpha \in A_{\lambda}})_{\lambda \in \Lambda}),
  \end{align*}
  and so
  \begin{align*}
    \pi_{\alpha}^X \circ g = \pi_{\alpha}^{Y_{\lambda}} \circ \pi_{\lambda}^Y,
  \end{align*}
  where the right hand side is a composition of continuous functions, and is therefore continuous. By theorem 101 part 3 we conclude that $g$ is continuous. Hence $X$ is homeomorphic to $Y$.

  Now, partitioning $A$ into $A_{1} = B$ and $A_{2} = C$ and applying this result yields the solution to the problem. 
\end{proof}
\end{document}



