\documentclass[12pt]{extarticle}
%Some packages I commonly use.
\usepackage[english]{babel}
\usepackage{graphicx}
\usepackage{framed}
\usepackage[normalem]{ulem}
\usepackage{amsmath}
\usepackage{amsthm}
\usepackage{amssymb}
\usepackage{amsfonts}
\usepackage{enumerate}
\usepackage{ytableau}
\usepackage[utf8]{inputenc}
\usepackage[top=1 in,bottom=1in, left=1 in, right=1 in]{geometry}

%A bunch of definitions that make my life easier
\newcommand{\matlab}{{\sc Matlab} }
\newcommand{\abs}[1]{|#1|}
\newcommand{\set}[1]{\{#1\}}
\newcommand{\cvec}[1]{{\mathbf #1}}
\newcommand{\rvec}[1]{\vec{\mathbf #1}}
\newcommand{\ihat}{\hat{\textbf{\i}}}
\newcommand{\jhat}{\hat{\textbf{\j}}}
\newcommand{\khat}{\hat{\textbf{k}}}
\newcommand{\minor}{{\rm minor}}
\newcommand{\trace}{{\rm trace}}
\newcommand{\spn}{{\rm Span}}
\newcommand{\rem}{{\rm rem}}
\newcommand{\ran}{{\rm range}}
\newcommand{\range}{{\rm range}}
\newcommand{\mdiv}{{\rm div}}
\newcommand{\proj}{{\rm proj}}
\newcommand{\sgn}{\text{sgn}}
\newcommand{\R}{\mathbb{R}}
\newcommand{\N}{\mathbb{N}}
\newcommand{\Q}{\mathbb{Q}}
\newcommand{\Z}{\mathbb{Z}}
\newcommand{\<}{\langle}
\renewcommand{\>}{\rangle}
\renewcommand{\emptyset}{\varnothing}
\newcommand{\attn}[1]{\textbf{#1}}
\theoremstyle{definition}
\newtheorem{theorem}{Theorem}
\newtheorem{prob}{Problem}
\newtheorem{corollary}{Corollary}
\newtheorem*{definition}{Definition}
\newtheorem*{example}{Example}
\newtheorem*{note}{Note}
\newtheorem{exercise}{Exercise}
\newcommand{\bproof}{\bigskip {\bf Proof. }}
\newcommand{\eproof}{\hfill\qedsymbol}
\newcommand{\Disp}{\displaystyle}
\newcommand{\qe}{\hfill\(\bigtriangledown\)}
\newcommand\restr[2]{{% we make the whole thing an ordinary symbol
  \left.\kern-\nulldelimiterspace % automatically resize the bar with \right
  #1 % the function
  \vphantom{\big|} % pretend it's a little taller at normal size
  \right|_{#2} % this is the delimiter
  }}
\setlength{\columnseprule}{1 pt}


\title{ Math 4995/5327
  \\
  Assignment 5}
\author{David Draguta}

\begin{document}

\maketitle

\begin{exercise}
  Give a combinatorial formula for $\< h_{\lambda}, h_{\mu} \>$, $\lambda, \mu \in \text{Par}$. More precisely, show that $\<h_{\lambda}, h_{\mu}\>$ is equal to the number of Young tableaux satisfying certain conditions. 
\end{exercise}
\begin{proof}
  We have by Prop 2.5.5.
  \begin{align*}
    h_{\lambda} = \sum\limits_{\nu \vdash \abs{\lambda}} M_{\lambda, \nu}(h,m)m_{\nu},
  \end{align*}
   for $M_{\lambda, \nu}(h,m)$ the number of row non-decreasing tableaux of shape $\lambda$ and weight $\nu$, and so
  \begin{align*}
    \< h_{\lambda}, h_{\mu} \>
    &= \< \sum\limits_{\nu \vdash \abs{\lambda}} M_{\lambda, \nu}(h,m)m_{\nu}, h_{\mu} \> \\
    &= \sum\limits_{\nu \vdash \abs{\lambda}} M_{\lambda, \nu}(h,m) \< m_{\nu}, h_{\mu} \> \\
    &= \sum\limits_{\nu \vdash \abs{\lambda}} M_{\lambda, \nu}(h,m) \< h_{\mu}, m_{\nu}\> \\
    &= \sum\limits_{\nu \vdash \abs{\lambda}} M_{\lambda, \nu}(h,m) \delta_{\mu \nu} \\
    &= M_{\lambda, \mu}(h,m),
  \end{align*}
  where we used the bilinearity and symmetry of the inner product, and it's definition.
\end{proof}
\begin{exercise}
  Write the skew Schur function $s_{(3,3,1)/(2,1)}$ as a linear combination of monomial symmetric functions. 
\end{exercise}
\begin{proof}
  By Proposition 3.4.15, we have
  \begin{align*}
    s_{(3,3,1)/(2,1)} = \sum\limits_{\nu \in Par} K_{(3,3,1)/(2,1), \nu} m_{\nu}
  \end{align*}
  Tableaux of weight $\nu=(3,1)$:
  \begin{align*}
    \begin{ytableau}
      \none & \none & 1 \\
      \none & 1 & 2 \\
      1
    \end{ytableau}\quad,\quad
    \begin{ytableau}
      \none & \none & 1 \\
      \none & 1 & 1 \\
      2
    \end{ytableau}
  \end{align*}
  
\end{proof}

\begin{exercise}
  Suppose $\lambda, \mu, \nu \in Par$ with $\lambda \supseteq \mu$. Show that
  \begin{align*}
    \< s_{\lambda / \mu}, h_{\nu}, \> = K_{\lambda/\mu, \nu}
  \end{align*}
  and
  \begin{align*}
    \< s_{\lambda / \mu}, e_{\nu} \> = K_{\lambda' / \mu', \nu}.
  \end{align*}
\end{exercise}
\begin{proof}
  \begin{align*}
    \< s_{\lambda / \mu}, h_{\nu} \>
    &= \< \sum\limits_{\rho \in Par} K_{\lambda/\mu, \rho} m_{\rho} , h_{\nu} \> \\
    &= \sum\limits_{\rho \in Par} K_{\lambda/\mu, \rho} \< m_{\rho} , h_{\nu} \> \\
    &= \sum\limits_{\rho \in Par} K_{\lambda/\mu, \rho} \< h_{\nu},  m_{\rho} \> \\
    &= \sum\limits_{\rho \in Par} K_{\lambda/\mu, \rho} \delta_{\nu \rho} \\
    &= K_{\lambda/\mu, \nu}, \\    
  \end{align*}
  and
  \begin{align*}
    \< s_{\lambda / \mu}, e_{\nu} \>
    &= \< \omega(s_{\lambda / \mu}), \omega(e_{\nu}) \> \\
    &= \< s_{\lambda' / \mu'}, h_{\nu} \> \\
    &= K_{\lambda' / \mu', \nu}.
  \end{align*}
\end{proof}
\end{document}
