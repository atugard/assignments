\documentclass[12pt]{extarticle}
%Some packages I commonly use.
\usepackage[english]{babel}
\usepackage{graphicx}
\usepackage{framed}
\usepackage[normalem]{ulem}
\usepackage{amsmath}
\usepackage{amsthm}
\usepackage{amssymb}
\usepackage{amsfonts}
\usepackage{enumerate}
\usepackage[utf8]{inputenc}
\usepackage[top=1 in,bottom=1in, left=1 in, right=1 in]{geometry}

%A bunch of definitions that make my life easier
\newcommand{\matlab}{{\sc Matlab} }
\newcommand{\abs}[1]{|#1|}
\newcommand{\set}[1]{\{#1\}}
\newcommand{\cvec}[1]{{\mathbf #1}}
\newcommand{\rvec}[1]{\vec{\mathbf #1}}
\newcommand{\ihat}{\hat{\textbf{\i}}}
\newcommand{\jhat}{\hat{\textbf{\j}}}
\newcommand{\khat}{\hat{\textbf{k}}}
\newcommand{\minor}{{\rm minor}}
\newcommand{\trace}{{\rm trace}}
\newcommand{\spn}{{\rm Span}}
\newcommand{\rem}{{\rm rem}}
\newcommand{\ran}{{\rm range}}
\newcommand{\range}{{\rm range}}
\newcommand{\mdiv}{{\rm div}}
\newcommand{\proj}{{\rm proj}}
\newcommand{\R}{\mathbb{R}}
\newcommand{\N}{\mathbb{N}}
\newcommand{\Q}{\mathbb{Q}}
\newcommand{\Z}{\mathbb{Z}}
\newcommand{\<}{\langle}
\renewcommand{\>}{\rangle}
\renewcommand{\emptyset}{\varnothing}
\newcommand{\attn}[1]{\textbf{#1}}
\theoremstyle{definition}
\newtheorem{theorem}{Theorem}
\newtheorem{prob}{Problem}
\newtheorem{corollary}{Corollary}
\newtheorem*{definition}{Definition}
\newtheorem*{example}{Example}
ssssssssssssssssssssssssssssssssssssssssssssssssssssssssssssssssssssssssssssssssssssssssssssssssssssssssssssssssssssssssssssssssssssssssssssssssssssssssssssssssssssssssssssssssssssssss\newtheorem*{note}{Note}
\newtheorem{exercise}{Exercise}
\newcommand{\bproof}{\bigskip {\bf Proof. }}
\newcommand{\eproof}{\hfill\qedsymbol}
\newcommand{\Disp}{\displaystyle}
\newcommand{\qe}{\hfill\(\bigtriangledown\)}
\setlength{\columnseprule}{1 pt}


\title{ Math 4995/5327
  \\
  Assignment 1}
\author{David Draguta}

\begin{document}

\maketitle

\begin{exercise}
  Prove that for $n \leq 5$ the dominance partial ordering on $\text{Par}(n)$ is a \textit{total} ordering. 
\end{exercise}
\begin{proof}
  First, for $\lambda = (\lambda_1, \lambda_2, \dots)$ such that $\lambda \vdash n$, we have for all $k \geq 1$
  \begin{align*}
    \sum\limits_{i=1}^k \lambda_i \leq \sum\limits_{i=1}^k \lambda_i,
  \end{align*}
  so that the relation is reflexive. This is also true for the empty partition \emptyset = (0,0,\dots). 

  For $n=0,1$, we have
  \begin{align*}
    \text{Par}(n) = \set{\emptyset}
  \end{align*}
  and
  \begin{align*}
    \text{Par}(n) = \set{(1)},
  \end{align*}
  respectively, and so the result holds since the relation is reflexive.
  
  For $n=2$:
  \begin{align*}
    \text{Par}(n) = \set{(2), (1,1)},
  \end{align*}
  and $(1,1) \leq (2)$ since $ 1 \leq 2$ and $ 1 + 1 \leq 2 + 0$, so the result holds.
  
  For $n=3$:
  \begin{align*}
    \text{Par}(n) = \set{(3), (2,1), (1,1,1)},
  \end{align*}
  and all partitions of $n$ (omiting relating a partition to itself) are relatable: 

  $(2,1) \leq (3)$, since $2 \leq 3$ and $2 + 1 \leq 3 + 0$;

  $(1,1,1) \leq (3)$ as $ 1 \leq 3$, $1 + 1 =2 \leq 3 $, and $1+1+1 = 3 \leq 3$; and

  $(1,1,1) \leq (2,1)$, since $1 \leq 2$, $1 + 1 = 2 \leq 3 = 2+1$, and $1+1+1=3 \leq 2+1+0 = 3$.

  For n=4:
  \begin{align*}
    \text{Par}(n) = \set{(4), (3,1), (2,2), (2,1,1), (1,1,1,1)}
  \end{align*}
  and again all partitions of $n$ (omiting relating a partition to itself) are relatable:

  $(3,1) \leq (4)$, since $3 \leq 4$ and $3 + 1 = 4 \leq 4$;

  $(2,2) \leq (4)$, since $2 \leq 4$ and $2+2 = 4 \leq 4$;

  $(2,1,1) \leq (4)$, since $ 2 \leq 4 $, $2+1=3 \leq 4$, and $2+1+1 = 4 \leq 4$;

  $(1,1,1,1) \leq (4)$, since $1 \leq 4$, $1+1=2 \leq 4$, $1+1+1=3 \leq 4$, and $1+1+1+1 = 4\leq 4$;

  $(2,2) \leq (3,1)$, since $ 2 \leq 3$ and $2+2=4 \leq 3+1=4 $;

  $(2,1,1) \leq (3,1)$, since $2 \leq 3$, $2+1=3 \leq 3+1 = 4$, and $2+1+1 = 4 \leq 3+1+0 = 4$;

  $(1,1,1,1) \leq (3,1)$, since $1 \leq 3$, $1+1=2 \leq 3+1=4$, $1+1+1 \leq 3+1+0$, and $1+1+1+1 = 4 \leq 3+1+0+0 = 4$;

  $(2,1,1) \leq (2,2)$, since $ 2 \leq 2$, $2+1 = 3 \leq 2+2 = 4$, and $2+1+1 = 3 \leq 2+2+0=4$;

  $(1,1,1,1) \leq (2,2)$, since $1 \leq 2$, $1+1 = 2 \leq 2+2=4$, $1+1+1 = 3 \\leq 2+2+0 = 4$, and $1+1+1+1 = 4 \leq 2+2+0+0=4$;

  $(1,1,1,1) \leq (2,1,1)$, since $1 \leq 2$, $1+1=2 \leq 2+1=3$, $1+1+1=3 \leq 2+1+1=4$, and $1+1+1+1=4 \leq 2+1+1+0 = 4$,

  so that the relation is total.

  For n = 5:
  \begin{align*}
    \text{Par}(n) = \set{(5), (4,1), (3,2), (3,1,1), (2,2,1), (2,1,1,1), (1,1,1,1,1)}
  \end{align*}
  and the relation is total since (omitting relating partitions to themselves):

  $(4,1) \leq (5)$, since $ 4 \leq 5$ and $4 + 1 = 5 \leq 5 $;

  $(3,2) \leq (5)$, since $4 \leq 5$ and $3+2=5 \leq 5$;

  $(3,1,1) \leq (5)$, since $3 \leq 5$, $3+1=4 \leq 5$, and $3+1+1=5 \leq 5$;

  $(2,2,1) \leq (5)$, since $2 \leq 5$, $2+2=4 \leq 5$, and $2+2+1 = 5 \leq 5$;

  $(2,1,1,1) \leq (5)$, since $2 \leq 5$, $2+1=3 \leq 5$, $2+1+1=4 \leq 5$, and $2+1+1+1 = 5 \leq 5$;

  $(1,1,1,1,1) \leq (5)$, since $1 \leq 5$, $1+1=2 \leq 5$, $1+1+1 = 3 \leq 5$, $1+1+1+1 = 4 \leq 5$, and $1+1+1+1+1 = 5 \leq 5$;

  $(3,2) \leq (4,1)$, since $ 3 \leq 4$, and $3 + 2 = 5 \leq 4+1 = 5$;

  $(3,1,1) \leq (4,1)$, since $ 3 \leq 4$, $3 + 1 = 4 \leq 4 + 1 = 5$, and $3+1+1 = 5 \leq 4+1+0=5$;

  $(2,2,1) \leq (4,1)$, since $ 2 \leq 4$, $2+2=4 \leq 4+1=5$, and $2+2+1 = 5 \leq 4+1+0 = 5$;

  $(2,1,1,1) \leq (4,1)$, since $ 2 \leq 4$, $2+1=3 \leq 4+1=5$, $2+1+1=4 \leq 4+1 = 5$, and $2+1+1+1 = 5 \leq 4+1 = 5$;

  $(1,1,1,1,1) \leq (4,1)$, since $1 \leq 4$, $1+1=2 \leq 4+1=5$, $1+1+1 = 3 \leq 4+1=5$, $1+1+1+1=4 \leq 4+1=5$, and $1+1+1+1+1 = 5 \leq 5$;

  $(3,1,1) \leq (3,2)$, since $ 3 \leq 3$, $3+1=4 \leq 3+2=5$, and $3+1+1 = 5 \leq 3+2 = 5$;

  $(2,2,1) \leq (3,2)$, since $2 \leq 3$, $2+2=4 \leq 3+2=5$, and $2+2+1=5 \leq 3+2=5$;

  $(2,1,1,1) \leq (3,2)$, since $ 2 \leq 3$, $2 + 1 = 3 \leq 3 + 2 = 5$, $2+1+1 = 4 \leq 3+2 = 5$, and $2+1+1+1 = 5 \leq 3+2=5$;

  $(1,1,1,1,1) \leq (3,2)$, since $1 \leq 3$, $1+1=2 \leq 3+2=5$, $1+1+1=3 \leq 3+2=5$, $1+1+1+1=4 \leq 3+2=5$, and $1+1+1+1+1=5 \leq 3+2=5$;

  $(2,2,1) \leq (3,1,1)$, since $ 2 \leq 3$, $2+2=4 \leq 3+1=4$, and $2+2+1 \leq 3+1+1=5$;

  $(2,1,1,1) \leq (3,1,1)$, since $2 \leq 3$, $2+1=3 \leq 3+1=4$, $2+1+1=4 \leq 3+1+1=5$, and $2+1+1+1=5 \leq 3+1+1=5$;

  $(1,1,1,1,1) \leq (3,1,1)$, since $1 \leq 3$, $1+1=2 \leq 3+1=4$, $1+1+1=3 \leq 3+1+1=5$, $1+1+1+1=4 \leq 3+1+1=5$, and $1+1+1+1+1=5 \leq 3+1+1=5$;

  $(1,1,1,1,1) \leq (2,2,1)$, since $1 \leq 2$, $1+1=2 \leq 2+2=4$, $1+1+1 = 3 \leq 2+2+1 = 5$, $1+1+1+1=4 \leq 2+2+1=5$, $1+1+1+1+1 = 5 \leq 2+2+1=5$;

  $(1,1,1,1,1) \leq (2,1,1,1)$, since $1 \leq 2$, $1+1=2 \leq 2+1=3$, $1+1+1=3 \leq 2+1+1=4$, $1+1+1+1=4 \leq 2+1+1+1=5$, and $1+1+1+1+1 = 5 \leq 2+1+1+1=5$.

  Hence, for $n \leq 5$ the dominance relation is a total order. 

\end{proof}
\begin{exercise}
  Suppose $R$ is a ring and $X$ is a set of ring automorphisms. Show that
  \begin{align*}
    R^X := \set{r \in R: \sigma(r)=r \text{ for all } \sigma \in X}
  \end{align*}
  is a subring of $R$. 
\end{exercise}
\begin{proof}
  First thing,
  \begin{align*}
    1 = \sigma(-1 \cdot -1) = \sigma(-1) \cdot \sigma(-1) = \sigma(-1)^2,
  \end{align*}
  so that
  \begin{align*}
    \sigma(-1) = \pm 1.
  \end{align*}
  Since $\sigma$ is an automorphism, it's injective. Since $\sigma(1) = 1$, it can't be that $\sigma(-1)=1$ too; so it must be that $\sigma(-1)=-1$. I'm only stating this here because I'll use it below. Also, $R^X \subseteq R$ is immediate from the definition, and so we don't check that below.
  
  \begin{itemize}
  \item
    $(R^X, +, 0)$ is a subgroup of $(R,+,0)$:

    By the definition of a ring morphism we know that $\sigma(0)=0$, so $0 \in R^X$ and $R^X$ is non empty. 

    If $a,b \in R^X$ then $\sigma(a)=a$ and $\sigma(b)=b$ so that $\sigma(a+b) = \sigma(a) + \sigma(b) = a + b$, and $a+b \in R^X$.

    If $a \in R^X$ then $\sigma(a)=a$, so that $\sigma(-a) = \sigma(-1)\sigma(a) = -1 \cdot a = -a$, and $-a \in R^X$.

    This is sufficient for $(R^X,+,0)$ to be a subgroup of $(R,+,0)$. \footnote{https://en.wikipedia.org/wiki/Subgroup}
  \item
    $(R^X, \cdot, 1)$ is a submonoid of $(R,\cdot,1)$:

    For $a,b \in R^X$, we have $\sigma(ab) = \sigma(a)\sigma(b) = ab$, so that $ab \in R^X$; also $\sigma(1) = 1$ so that $1 \in R^X$.

    This is sufficient for $(R^X,\cdot,1)$ to be a submonoid of $(R,\cdot,1)$. \footnote{https://en.wikipedia.org/wiki/Monoid#Submonoids}
  \end{itemize}
  Hence, $R^X$ is a subring of $R$. \footnote{https://en.wikipedia.org/wiki/Subring#Definition}

\end{proof}
\begin{exercise}
  Suppose $R$ is a commutative ring and $a \in R$. Prove that
  \begin{align*}
    (1-at)^{-1} = \sum\limits_{n=0}^{\infty} a^n t^n.
  \end{align*}
\end{exercise}
\begin{proof}
  I'm going to use induction on the index of the coefficients of the inverse and the recurrence relations in the proof of Proposition 1.4.1 in the course notes \footnote{https://alistairsavage.ca/symfunc/notes/Savage-SymmetricFunctions.pdf}.

  We first define
  \begin{align*}
   \sum\limits_{n=0}^{\infty} a_nt^n :=  1-at
  \end{align*}
  where $a_0 = 1$, $a_1 = -a$, and $a_n = 0$ for $n > 1$; and we wish to solve for $b_n$ in 
  \begin{align*}
    (\sum\limits_{i=0}^{\infty} a_nt^n)^{-1}  = \sum\limits_{n=0}^{\infty} b_n t^n.
  \end{align*}
  using the recurrence relations

  \begin{align*}
    b_0 = a_0^{-1}, \quad b_n = -a_0^{-1} \sum\limits_{i=1}^{n
    }a_i b_{n-i}, \quad n \geq 1.
  \end{align*}

  We have for $n=0$

  \begin{align*}
    b_0 = a_0^{-1}= 1^{-1} = 1 = a^0.
  \end{align*}

  For $n=1$:
  \begin{align*}
    b_1 = -a_0^{-1}(a_1 b_0) = -1 (-a*1) = a;
  \end{align*}

  and for $n \geq 1$
  \begin{align*}
    b_n
    &= -a_0^{-1}\sum\limits_{i=1}^n a_i b_{n-i} \\
    &= -1 (a_1\cdot b_{n-1} + a_2 \cdot b_{n-2} + a_3 \cdot b_{n-3} + \dots + a_n \cdot b_0) \\
    &= -1 (-a \cdot a^{n-1} + 0 \cdot a^{n-2} + 0 \cdot a^{n-3} + \dots + 0 \cdot1) \\
    &= a^n,
  \end{align*}

  so that
    \begin{align*}
    (1-at)^{-1} = \sum\limits_{n=0}^{\infty} a^n t^n.
  \end{align*}

\end{proof}
\end{document}
