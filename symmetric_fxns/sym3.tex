\documentclass[12pt]{extarticle}
%Some packages I commonly use.
\usepackage[english]{babel}
\usepackage{graphicx}
\usepackage{framed}
\usepackage[normalem]{ulem}
\usepackage{amsmath}
\usepackage{amsthm}
\usepackage{amssymb}
\usepackage{amsfonts}
\usepackage{enumerate}
\usepackage{ytableau}
\usepackage[utf8]{inputenc}
\usepackage[top=1 in,bottom=1in, left=1 in, right=1 in]{geometry}

%A bunch of definitions that make my life easier
\newcommand{\matlab}{{\sc Matlab} }
\newcommand{\abs}[1]{|#1|}
\newcommand{\set}[1]{\{#1\}}
\newcommand{\cvec}[1]{{\mathbf #1}}
\newcommand{\rvec}[1]{\vec{\mathbf #1}}
\newcommand{\ihat}{\hat{\textbf{\i}}}
\newcommand{\jhat}{\hat{\textbf{\j}}}
\newcommand{\khat}{\hat{\textbf{k}}}
\newcommand{\minor}{{\rm minor}}
\newcommand{\trace}{{\rm trace}}
\newcommand{\spn}{{\rm Span}}
\newcommand{\rem}{{\rm rem}}
\newcommand{\ran}{{\rm range}}
\newcommand{\range}{{\rm range}}
\newcommand{\mdiv}{{\rm div}}
\newcommand{\proj}{{\rm proj}}
\newcommand{\R}{\mathbb{R}}
\newcommand{\N}{\mathbb{N}}
\newcommand{\Q}{\mathbb{Q}}
\newcommand{\Z}{\mathbb{Z}}
\newcommand{\<}{\langle}
\renewcommand{\>}{\rangle}
\renewcommand{\emptyset}{\varnothing}
\newcommand{\attn}[1]{\textbf{#1}}
\theoremstyle{definition}
\newtheorem{theorem}{Theorem}
\newtheorem{prob}{Problem}
\newtheorem{corollary}{Corollary}
\newtheorem*{definition}{Definition}
\newtheorem*{example}{Example}
\newtheorem*{note}{Note}
\newtheorem{exercise}{Exercise}
\newcommand{\bproof}{\bigskip {\bf Proof. }}
\newcommand{\eproof}{\hfill\qedsymbol}
\newcommand{\Disp}{\displaystyle}
\newcommand{\qe}{\hfill\(\bigtriangledown\)}
\newcommand\restr[2]{{% we make the whole thing an ordinary symbol
  \left.\kern-\nulldelimiterspace % automatically resize the bar with \right
  #1 % the function
  \vphantom{\big|} % pretend it's a little taller at normal size
  \right|_{#2} % this is the delimiter
  }}
\setlength{\columnseprule}{1 pt}


\title{ Math 4995/5327
  \\
  Assignment 3}
\author{David Draguta}

\begin{document}

\maketitle





%% \ytableausetup
%% {mathmode, boxframe=normal, boxsize=2em}
%% \begin{ytableau}
%% 1 & 2 & 3 & \none[\dots]
%% & \scriptstyle 2n - 1 & 2n \\
%% 2 & 3 & 4 & \none[\dots]
%% & 2n \\
%% \none[\vdots] & \none[\vdots]
%% & \none[\vdots] \\
%% \scriptstyle 2n - 1 & 2n \\
%% 2n
%% \end{ytableau}
%% \ytableausetup{centertableaux}



\begin{exercise}
  Find and prove a formula for $M_{\lambda, (n-1,1)}(e,m)$, where $\lambda \vdash n$.
\end{exercise}
\begin{proof}
  We require that $\mu = (n-1,1)$ is a partition, so $n-1 \geq 1$, or $n \geq 2$. The only choice of shape for the tableaux is $\lambda = (2, 1^{n-2})$ or $\lambda = (1^n)$ .

  In the case of $\lambda = (2, 1^{n-2})$, the only possible tableau is 
  \begin{align*}
    \begin{ytableau}
      1 & 2 \\
      1 \\
      \none[ \vdots ] \\
      1
    \end{ytableau}
  \end{align*}
  since we must put a 1 in the top-left box, and since it's row strict, we must put a 2 in the second box of the first row; but now there are no further choices for the remaining boxes -- they must be 1s.

  For $\lambda = (1^n)$, we have the following $n$ possible tableaux
    \begin{align*}
    \begin{ytableau}
      2 \\
      1 \\
      \none[ \vdots ] \\
      1
    \end{ytableau}\quad , \quad
    \begin{ytableau}
      1 \\
      2 \\
      \none[ \vdots ] \\
      1
    \end{ytableau}\quad , \quad \dots \quad , \quad
        \begin{ytableau}
      1 \\
      1 \\
      \none[ \vdots ] \\
      2
    \end{ytableau} \quad,
  \end{align*}
    since exactly one $2$ must be placed somewhere, after which placement no further choice remains.
    
    We conclude that
    \begin{align*}
      M_{\lambda, \mu}(e,m) =
      \begin{cases}
        1 & \text{ if } \lambda = (2, 1^{n-2}) \\
        n & \text{ if } \lambda = (1^n)
      \end{cases}
    \end{align*}
\end{proof}
\begin{exercise}
  Find and prove a formula for $M_{\lambda, (n-1,1)}(h,m)$, where $\lambda \vdash n$.
\end{exercise}
\begin{proof}
  As the content is $\mu = (n-1,1)$, whatever the shape of the tableau we can only have one 2 in its filling, and so the 2 must be placed in the rightmost box of any row.
  \footnote{
    Because there can only be exactly one $2$ and the rows are row weak, so once you commit to placing a $2$ you can only place numbers $\geq$ 2 in boxes to the right of this box. 
  }
  Once this choice of placement of 2 is made, the remaining $n-1$ boxes must be filled with ones. The only choice then is of where to place this $2$. There are $\ell(\lambda)$ such choices, for a tableau of shape $\lambda$, corresponding to the rightmost boxes of the $\ell(\lambda)$ rows:

  \begin{align*}
    \begin{ytableau}
      1 & 1 & \none[\dots] & 2 \\
      1 & 1 & \none[\dots] & 1 \\
      \none[\vdots]
      & \none[\vdots] \\
      1 & 1 \\
      1
    \end{ytableau}\quad,\quad
    \begin{ytableau}
      1 & 1 & \none[\dots] & 1 \\
      1 & 1 & \none[\dots] & 2 \\
      \none[\vdots]
      & \none[\vdots] \\
      1 & 1 \\
      1
    \end{ytableau}\quad,\quad \dots \quad , \quad
    \begin{ytableau}
      1 & 1 & \none[\dots] & 1 \\
      1 & 1 & \none[\dots] & 1 \\
      \none[\vdots]
      & \none[\vdots] \\
      1 & 2 \\
      1
    \end{ytableau}\quad, \quad
    \begin{ytableau}
      1 & 1 & \none[\dots] & 1 \\
      1 & 1 & \none[\dots] & 1 \\
      \none[\vdots]
      & \none[\vdots] \\
      1 & 1 \\
      2
    \end{ytableau}\quad.
  \end{align*}
  Thus, $M_{\lambda, (n-1,1)}(h,m) = \ell(\lambda)$.
\end{proof}
\begin{exercise}
  Find and prove a formula for $M_{\lambda, (1^n)}(p,m)$, where $\lambda \vdash n$.
\end{exercise}

\begin{proof}
  We require that the rows have constant entries in the filled tableaux, but our weight requires that we have exactly one of each entry, so $\lambda = (1^n)$ is the only possible shape for the tableau.
  We then have $n$ choices for the first row, $n-1$ choices for the second, and so on, until finally $1$ choice remains for the $n$th row. Thus, we have $n!$ such fillings, and we conclude:
  \begin{align*}
    M_{\lambda, (1^n)}(p,m) =
    \begin{cases}
      n! & \text{ if } \lambda = (1^n) \\
      0 & \text{ otherwise}
    \end{cases}
  \end{align*}
\end{proof}
\end{document}
