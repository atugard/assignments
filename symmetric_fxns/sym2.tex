\documentclass[12pt]{extarticle}
%Some packages I commonly use.
\usepackage[english]{babel}
\usepackage{graphicx}
\usepackage{framed}
\usepackage[normalem]{ulem}
\usepackage{amsmath}
\usepackage{amsthm}
\usepackage{amssymb}
\usepackage{amsfonts}
\usepackage{enumerate}
\usepackage[utf8]{inputenc}
\usepackage[top=1 in,bottom=1in, left=1 in, right=1 in]{geometry}

%A bunch of definitions that make my life easier
\newcommand{\matlab}{{\sc Matlab} }
\newcommand{\abs}[1]{|#1|}
\newcommand{\set}[1]{\{#1\}}
\newcommand{\cvec}[1]{{\mathbf #1}}
\newcommand{\rvec}[1]{\vec{\mathbf #1}}
\newcommand{\ihat}{\hat{\textbf{\i}}}
\newcommand{\jhat}{\hat{\textbf{\j}}}
\newcommand{\khat}{\hat{\textbf{k}}}
\newcommand{\minor}{{\rm minor}}
\newcommand{\trace}{{\rm trace}}
\newcommand{\spn}{{\rm Span}}
\newcommand{\rem}{{\rm rem}}
\newcommand{\ran}{{\rm range}}
\newcommand{\range}{{\rm range}}
\newcommand{\mdiv}{{\rm div}}
\newcommand{\proj}{{\rm proj}}
\newcommand{\R}{\mathbb{R}}
\newcommand{\N}{\mathbb{N}}
\newcommand{\Q}{\mathbb{Q}}
\newcommand{\Z}{\mathbb{Z}}
\newcommand{\<}{\langle}
\renewcommand{\>}{\rangle}
\renewcommand{\emptyset}{\varnothing}
\newcommand{\attn}[1]{\textbf{#1}}
\theoremstyle{definition}
\newtheorem{theorem}{Theorem}
\newtheorem{prob}{Problem}
\newtheorem{corollary}{Corollary}
\newtheorem*{definition}{Definition}
\newtheorem*{example}{Example}
\newtheorem*{note}{Note}
\newtheorem{exercise}{Exercise}
\newcommand{\bproof}{\bigskip {\bf Proof. }}
\newcommand{\eproof}{\hfill\qedsymbol}
\newcommand{\Disp}{\displaystyle}
\newcommand{\qe}{\hfill\(\bigtriangledown\)}
\newcommand\restr[2]{{% we make the whole thing an ordinary symbol
  \left.\kern-\nulldelimiterspace % automatically resize the bar with \right
  #1 % the function
  \vphantom{\big|} % pretend it's a little taller at normal size
  \right|_{#2} % this is the delimiter
  }}
\setlength{\columnseprule}{1 pt}


\title{ Math 4995/5327
  \\
  Assignment 1}
\author{David Draguta}

\begin{document}

\maketitle

\begin{exercise}
  For a subset $X \subseteq \mathfrak{S}_n$, define
  \begin{align*}
    Pol_n^{X} = \set{f \in Pol_n: f^{\pi} = f \text{ for all } \pi \in X}.
  \end{align*}
  For a subset $P \subseteq Pol_n$, define
  \begin{align*}
    G(P) = \set{\pi \in \mathfrak{S}_n: f^{\pi} = f \text{ for all } f \in P}.
  \end{align*}
  Is it true that $G(Pol_n^H) = H$ for all subgroups $H \subseteq \mathfrak{S}_n$, $n \in \N$? Prove or give a counter example. 
\end{exercise}

\begin{proof}
  It's true.

  Let $\pi \in H$, then for any $f \in Pol_n^H$, $f^{\pi} = f$, so that $\pi \in G(Pol_n^H)$, $H \subseteq G(Pol_n^H)$.

  We note that $f_{\lambda} := \sum\limits_{\alpha \in \lambda H} x^{\alpha} \in Pol_n^H$, for all partitions $\lambda \in Par_n$.
  \footnote{Let $x^\alpha$ be a term in $f_{\lambda}$, where $\alpha = \lambda^{\sigma}$ for some $\sigma \in H$; then for $\pi \in H$ we have
  \begin{align*}
    (x^\alpha)^{\pi}
    &= x_{\pi(1)}^{\alpha_1} \cdots x_{\pi(n)}^{\alpha_n} \\
    &= x_{1}^{\alpha_{\pi^{-1}(1)}} \cdots x_{n}^{\alpha_{\pi^{-1}(n)}} \\
    &= x^{\alpha^{\pi^{-1}}} \\ 
    &= x^{{\lambda^{\sigma}}^{\pi^{-1}}}.
  \end{align*}

  As $\pi^{-1} \in H$, $H\pi^{-1} = H$, so 
  
  \begin{align*}
    \set{\lambda^{\sigma \pi^{-1}}: \sigma \in H} = \lambda(H \pi^{-1}) = \lambda H,
  \end{align*}
  and we end up with the same index set $\lambda H$.

  }

  Let $\pi \in G(Pol_n^H)$. Fix $\lambda = (n, n-1, \dots, 1)$, and let $\lambda^{\sigma} \in \lambda H$ be such that ${(x^{\lambda})}^{\pi} = x^{\lambda^{\sigma}}$. \footnote{Such a $\sigma$ exists since $\pi$ fixes $f_{\lambda}$, and ${(x^\lambda)}^{\pi}$ is a term in $f_{\lambda}$.}
  
  We have:
  \begin{align*}
    x_{\pi(1)}^nx_{\pi(2)}^{n-1} \cdots x_{\pi(n)}
    &= x_1^{\lambda_{\sigma(1)}} x_2^{\lambda_{\sigma(2)}} \cdots x_n^{\lambda_{\sigma(n)}} \\
    &= x_{\sigma^{-1}(1)}^{\lambda_1} x_{\sigma^{-1}(2)}^{\lambda_2} \cdots x_{\sigma^{-1}(n)}^{\lambda_n} \\
    &= x_{\sigma^{-1}(1)}^{n} x_{\sigma^{-1}(2)}^{n-1} \cdots x_{\sigma^{-1}(n)}^{1}.
  \end{align*}
  As the parts of the partition are all distinct, we must have for each $1 \leq i \leq n$,
  \begin{align*}
    x_{\pi(i)}^{n-i+1} = x_{\sigma^{-1}(i)}^{n-i+1},
  \end{align*}
  which implies that for all $1 \leq i \leq n$,
  \begin{align*}
    \pi(i) = \sigma^{-1}(i),
  \end{align*}
  i.e. $\pi = \sigma^{-1} \in H$, $G(Pol_n^H) \subseteq H$, $G(Pol_n^H) = H$.
\end{proof}

\begin{exercise}
  Prove Lemma 2.2.3 in the lecture notes. Furthermore, prove that $\rho_{m,n}^k$ is \emph{not} bijective if $m>n$ and $k>n$.
\end{exercise}
\begin{proof}
  Let $g \in Sym_n^k$, then $g \in Sym_n$. As $\rho_{m,n}$ is surjective, there exists an $f \in Sym_m$ such that $\rho_{m,n}(f) = g$. We decompose $f$ into it's homogeneous components as $f = \sum\limits_{i} f_i$, and write
  \footnote{$f = \sum\limits_{0 \leq i \leq d} f_i$ is of bounded degree, but just to keep notation clean I left that out.}
  \begin{align*}
    \sum\limits_i \rho_{m,n}(f_i) =  \rho_{m,n}(f) = g
  \end{align*}
  By definition of graded ring morphism, $\rho_{m,n}(f_i) \in Sym_n^i$, and as $g$ is homogeneous of degree $k$ and the sum is direct, it must be that $\rho_{m,n}(f_i) = 0$ for $i \neq k$ and $\rho_{m,n}(f_i) = g$ for $i=k$, i.e.
  \begin{align*}
    \rho_{m,n}^k(f_k) = \restr{\rho_{m,n}}{Sym_m^k}(f_k) = \rho_{m,n}(f_k) = g,
  \end{align*}
  and $\rho_{m,n}^k$ is surjective.

  Suppose $\rho_{m,n}^k(f) = 0$. Write $f = \sum\limits_{i=1}^{p(k)} a_{\lambda_i} m_{\lambda_i}(x_1, \dots, x_m)$
  , for $\lambda_i \in Par(k)$, $a_{\lambda_i} \in \Z$.
  \footnote{
  By Corollary 2.1.3 (b) since $m \geq k$ the mononial symmetric polynomials $m_{\lambda}(x_1, \dots, x_m)$ form a basis of $Sym_m^k$ for $\lambda \in Par(k)$.
  }
  Then
  \begin{align*}
   \sum\limits_{i=1}^{p(k)} a_{\lambda_i} m_{\lambda_i}(x_1, \dots, x_n) =  \rho_{m,n}^k(f) = 0,
  \end{align*}
  and as the monomial symmetric polynomials $m_{\lambda_i}(x_1, \dots, x_n) \in Sym_n^k$ are linearly independent
  \footnote{
  Again, by Corollary 2.1.3 as $n \geq k$, for $\lambda \in Par(k)$, $m_{\lambda}(x_1, \dots, x_n)$ form a basis of $Sym_n^k$, and particularly are linearly independent over $\Z$.
  }
  , we have $a_{\lambda_i}=0$ for each $i$, i.e. $f=0$ and $\rho_{m,n}^k$ is injective.
\end{proof}
\end{document}
