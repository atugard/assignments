\documentclass[12pt]{extarticle}
%Some packages I commonly use.
\usepackage[english]{babel}
\usepackage{graphicx}
\usepackage{framed}
\usepackage[normalem]{ulem}
\usepackage{amsmath}
\usepackage{amsthm}
\usepackage{amssymb}
\usepackage{amsfonts}
\usepackage{enumerate}
\usepackage[utf8]{inputenc}
\usepackage[top=1 in,bottom=1in, left=1 in, right=1 in]{geometry}

%A bunch of definitions that make my life easier
\newcommand{\matlab}{{\sc Matlab} }
\newcommand{\abs}[1]{|#1|}
\newcommand{\set}[1]{\{#1\}}
\newcommand{\cvec}[1]{{\mathbf #1}}
\newcommand{\rvec}[1]{\vec{\mathbf #1}}
\newcommand{\ihat}{\hat{\textbf{\i}}}
\newcommand{\jhat}{\hat{\textbf{\j}}}
\newcommand{\khat}{\hat{\textbf{k}}}
\newcommand{\minor}{{\rm minor}}
\newcommand{\trace}{{\rm trace}}
\newcommand{\spn}{{\rm Span}}
\newcommand{\rem}{{\rm rem}}
\newcommand{\ran}{{\rm range}}
\newcommand{\range}{{\rm range}}
\newcommand{\mdiv}{{\rm div}}
\newcommand{\proj}{{\rm proj}}
\newcommand{\R}{\mathbb{R}}
\newcommand{\N}{\mathbb{N}}
\newcommand{\Q}{\mathbb{Q}}
\newcommand{\Z}{\mathbb{Z}}
\newcommand{\<}{\langle}
\renewcommand{\>}{\rangle}
\renewcommand{\emptyset}{\varnothing}
\newcommand{\attn}[1]{\textbf{#1}}
\theoremstyle{definition}
\newtheorem{theorem}{Theorem}
\newtheorem{prob}{Problem}
\newtheorem{corollary}{Corollary}
\newtheorem*{definition}{Definition}
\newtheorem*{example}{Example}
\newtheorem*{note}{Note}
\newtheorem{exercise}{Exercise}
\newcommand{\bproof}{\bigskip {\bf Proof. }}
\newcommand{\eproof}{\hfill\qedsymbol}
\newcommand{\Disp}{\displaystyle}
\newcommand{\qe}{\hfill\(\bigtriangledown\)}
\setlength{\columnseprule}{1 pt}


\title{ Math 4995/5327
  \\
  Assignment 1}
\author{David Draguta}

\begin{document}

\maketitle

\begin{exercise}
  For a subset $X \subseteq \mathfrak{S}_n$, define
  \begin{align*}
    Pol_n^{X} = \set{f \in Pol_n: f^{\pi} = f \text{ for all } \pi \in X}.
  \end{align*}
  For a subset $P \subseteq Pol_n$, define
  \begin{align*}
    G(P) = \set{\pi \in \mathfrak{S}_n: f^{\pi} = f \text{ for all } f \in P}.
  \end{align*}
  Is it true that $G(Pol_n^H)$ for all subgroups $H \subseteq \mathfrak{S}_n$, $n \in \N$? Prove or give a counter example. 
\end{exercise}

\begin{proof}
  If $\pi \in H$, then for any $f \in Pol_n^H$, $f^{\pi} = f$, so that $\pi \in G(Pol_n^H)$, $H \subseteq G(Pol_n^H)$.

  We note that $\sum\limits_{\alpha \in \lambda H} x^{\alpha} \in Pol_n^H$, for all partitions $\lambda \in Par_n$, and in particular for $\lambda = (n, n-1, \dots, 1)$.
  
  Let $\pi \in G(Pol_n^H)$, then there exists some $\sigma \in H$ such that
  \begin{align*}
    {(x^{\lambda})}^{\pi} = x_{\pi(1)}^nx_{\pi(2)}^{n-1} \cdots x_{\pi(n)} = x_{\sigma(1)}^nx_{\sigma(2)}^{n-1} \cdots x_{\sigma(n)} = {(x^{\lambda})}^{\sigma},
  \end{align*}
  and so $\pi = \sigma \in H$, $G(Pol_n^H) \subseteq H$.

  We've shown $H = G(Pol_n^H)$.

\end{proof}

\begin{exercise}
  Prove Lemma 2.2.3 in the lecture notes. Furthermore, prove that $\rho_{m,n}^k$ is \emph{not} bijective if $m>n$ and $k>n$.
\end{exercise}
\end{document}
