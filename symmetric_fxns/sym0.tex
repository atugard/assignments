\documentclass[12pt]{extarticle}
%Some packages I commonly use.
\usepackage[english]{babel}
\usepackage{graphicx}
\usepackage{framed}
\usepackage[normalem]{ulem}
\usepackage{amsmath}
\usepackage{amsthm}
\usepackage{amssymb}
\usepackage{amsfonts}
\usepackage{enumerate}
\usepackage[utf8]{inputenc}
\usepackage[top=1 in,bottom=1in, left=1 in, right=1 in]{geometry}

%A bunch of definitions that make my life easier
\newcommand{\matlab}{{\sc Matlab} }
\newcommand{\abs}[1]{|#1|}
\newcommand{\set}[1]{\{#1\}}
\newcommand{\cvec}[1]{{\mathbf #1}}
\newcommand{\rvec}[1]{\vec{\mathbf #1}}
\newcommand{\ihat}{\hat{\textbf{\i}}}
\newcommand{\jhat}{\hat{\textbf{\j}}}
\newcommand{\khat}{\hat{\textbf{k}}}
\newcommand{\minor}{{\rm minor}}
\newcommand{\trace}{{\rm trace}}
\newcommand{\spn}{{\rm Span}}
\newcommand{\rem}{{\rm rem}}
\newcommand{\ran}{{\rm range}}
\newcommand{\range}{{\rm range}}
\newcommand{\mdiv}{{\rm div}}
\newcommand{\proj}{{\rm proj}}
\newcommand{\R}{\mathbb{R}}
\newcommand{\N}{\mathbb{N}}
\newcommand{\Q}{\mathbb{Q}}
\newcommand{\Z}{\mathbb{Z}}
\newcommand{\<}{\langle}
\renewcommand{\>}{\rangle}
\renewcommand{\emptyset}{\varnothing}
\newcommand{\attn}[1]{\textbf{#1}}
\theoremstyle{definition}
\newtheorem{theorem}{Theorem}
\newtheorem{prob}{Problem}
\newtheorem{corollary}{Corollary}
\newtheorem*{definition}{Definition}
\newtheorem*{example}{Example}
\newtheorem*{note}{Note}
\newtheorem{exercise}{Exercise}
\newcommand{\bproof}{\bigskip {\bf Proof. }}
\newcommand{\eproof}{\hfill\qedsymbol}
\newcommand{\Disp}{\displaystyle}
\newcommand{\qe}{\hfill\(\bigtriangledown\)}
\setlength{\columnseprule}{1 pt}


\title{ Math 4995/5327 Symmetric Functions - Assignment 1}
\author{David Draguta}

\begin{document}

\maketitle

\begin{exercise}
  What are your mathematical interests? Is there some specific material you hope to learn in this course?
\end{exercise}
\begin{proof}
  I want to understand my experience. That's really broad, but I like A.N. Whitehead's Process and Reality, where he claims that experience is the most information dense, complete, whatever (not in a technical sense) thing that we then abstract away from into pieces... Well, my goal is to try to understand experience, and it seems there is a subject-superject thing there... the subject generates symbols which are interpreted as subjectivity, and the superject ... well, I'm not sure the nature of the superject, but there's a holographic intermediary between subject-superject through which classical information flows? Anyway, I want to know the math that I'd need to understand both sides of that coin... hopefully that made sense, lol.
\end{proof}

\begin{exercise}
  How comfortable are you preparing documents with LaTeX?
\end{exercise}
\begin{proof}
  Not comfortable enough to know how to typeset LaTeX without looking it up on the interwebs, but comfortable enough to write assignments in it, define new commands, etc... I know enough to find out how to do things if they need to be done, and if I'm motivated enough to do it...
\end{proof}

\begin{exercise}
  How familiar are you with tensor product of vector spaces?
\end{exercise}

\begin{proof}
  Not very... I've seen the Tor functor, and maybe that's the most involved thing I've worked with that uses tensor products... 
\end{proof}
\end{document}
