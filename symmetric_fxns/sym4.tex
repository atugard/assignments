\documentclass[12pt]{extarticle}
%Some packages I commonly use.
\usepackage[english]{babel}
\usepackage{graphicx}
\usepackage{framed}
\usepackage[normalem]{ulem}
\usepackage{amsmath}
\usepackage{amsthm}
\usepackage{amssymb}
\usepackage{amsfonts}
\usepackage{enumerate}
\usepackage{ytableau}
\usepackage[utf8]{inputenc}
\usepackage[top=1 in,bottom=1in, left=1 in, right=1 in]{geometry}

%A bunch of definitions that make my life easier
\newcommand{\matlab}{{\sc Matlab} }
\newcommand{\abs}[1]{|#1|}
\newcommand{\set}[1]{\{#1\}}
\newcommand{\cvec}[1]{{\mathbf #1}}
\newcommand{\rvec}[1]{\vec{\mathbf #1}}
\newcommand{\ihat}{\hat{\textbf{\i}}}
\newcommand{\jhat}{\hat{\textbf{\j}}}
\newcommand{\khat}{\hat{\textbf{k}}}
\newcommand{\minor}{{\rm minor}}
\newcommand{\trace}{{\rm trace}}
\newcommand{\spn}{{\rm Span}}
\newcommand{\rem}{{\rm rem}}
\newcommand{\ran}{{\rm range}}
\newcommand{\range}{{\rm range}}
\newcommand{\mdiv}{{\rm div}}
\newcommand{\proj}{{\rm proj}}
\newcommand{\sgn}{\text{sgn}}
\newcommand{\R}{\mathbb{R}}
\newcommand{\N}{\mathbb{N}}
\newcommand{\Q}{\mathbb{Q}}
\newcommand{\Z}{\mathbb{Z}}
\newcommand{\<}{\langle}
\renewcommand{\>}{\rangle}
\renewcommand{\emptyset}{\varnothing}
\newcommand{\attn}[1]{\textbf{#1}}
\theoremstyle{definition}
\newtheorem{theorem}{Theorem}
\newtheorem{prob}{Problem}
\newtheorem{corollary}{Corollary}
\newtheorem*{definition}{Definition}
\newtheorem*{example}{Example}
\newtheorem*{note}{Note}
\newtheorem{exercise}{Exercise}
\newcommand{\bproof}{\bigskip {\bf Proof. }}
\newcommand{\eproof}{\hfill\qedsymbol}
\newcommand{\Disp}{\displaystyle}
\newcommand{\qe}{\hfill\(\bigtriangledown\)}
\newcommand\restr[2]{{% we make the whole thing an ordinary symbol
  \left.\kern-\nulldelimiterspace % automatically resize the bar with \right
  #1 % the function
  \vphantom{\big|} % pretend it's a little taller at normal size
  \right|_{#2} % this is the delimiter
  }}
\setlength{\columnseprule}{1 pt}


\title{ Math 4995/5327
  \\
  Assignment 4}
\author{David Draguta}

\begin{document}

\maketitle

\begin{exercise}
  Show that for all $\alpha \in WComp_n$, the polynomial
  \begin{align*}
    a_{\alpha} = a_{\alpha}(x_1, \dots, x_n) := \sum\limits_{\pi \in \mathfrak{S}_n} \sgn(\pi) x^{\alpha^{\pi}}
  \end{align*}
  is alternating.
\end{exercise}
\begin{proof}
  \begin{align*}
    a_{\alpha}^{\sigma}
    &= (\sum\limits_{\pi \in \mathfrak{S_n}} \sgn(\pi) x^{\alpha^{\pi}})^{\sigma} \\ 
    &= \sum\limits_{\pi \in \mathfrak{S_n}} \sgn(\pi) (x^{\alpha^{\pi}})^{\sigma} \\
    &= \sum\limits_{\pi \in \mathfrak{S_n}} \sgn(\pi) (x_1^{\alpha_{\pi(1)}} \cdots x_n^{\alpha_{\pi(n)}})^{\sigma} \\
    &= \sum\limits_{\pi \in \mathfrak{S_n}} \sgn(\pi) x_{\sigma(1)}^{\alpha_{\pi(1)}} \cdots x_{\sigma(n)}^{\alpha_{\pi(n)}} \\
    &= \sum\limits_{\pi \in \mathfrak{S_n}} \sgn(\pi) x_{1}^{\alpha_{\pi(\sigma^{-1}(1))}} \cdots x_{n}^{\alpha_{\pi(\sigma^{-1}(n))}} \\
    &= \sum\limits_{\pi \in \mathfrak{S_n}} \sgn(\pi) x^{\alpha^{\pi\sigma^{-1}}} \\
    &= \sgn(\sigma)\sum\limits_{\pi \in \mathfrak{S_n}} \sgn(\pi\sigma^{-1}) x^{\alpha^{\pi\sigma^{-1}}} \\
    &= \sgn(\sigma)\sum\limits_{\rho \in \mathfrak{S_n}\sigma^{-1}} \sgn(\rho) x^{\alpha^{\rho}} \\
    &= \sgn(\sigma)\sum\limits_{\rho \in \mathfrak{S_n}} \sgn(\rho) x^{\alpha^{\rho}} \\
    &= \sgn(\sigma)a_{\alpha},
  \end{align*}
  where we used the group morphism properties of $\sgn: \mathfrak{S_n} \to \set{-1,1}$, specifically that
  $\sgn(\pi) = \sgn(\pi \sigma^{-1} \sigma) = \sgn(\pi \sigma^{-1}) \sgn(\sigma)$.
\end{proof}
\begin{exercise}
  Using only results from Section 3.2 and earlier in the lecture notes, write the Schur function $s_{(3,1)}$ as a linear combination of the complete homogeneous symmetric functions, the elementary symmetric functions, and the monomial symmetric functions. \emph{Hint}: Use your expression in terms of the complete homogeneous symmetric functions or the elementary symmetric functions to deduce an expression in terms of the monomial symmetric functions. 
\end{exercise}
\begin{proof}
  We use equation 3.10 in the lecture notes to get the schur function as a linear combination of the complete homogeneous symmetric functions:
  \begin{align*}
    s_{(3,1)} &
    = 
    \begin{vmatrix}
      h_3 & h_4 \\
      1 & h_1 
    \end{vmatrix} \\
    &=
    h_3h_1 - h_4 \\
    &= h_{(3,1)} - h_{(4)}.
  \end{align*}
  Next, using equation 3.11, we get 
  \begin{align*}
    s_{(3,1)} 
    &= 
    \begin{vmatrix}
      e_2 & e_3 & e_4 \\
      1 & e_1 & e_2 \\
      0 & 1 & e_1
    \end{vmatrix} \\
    &= e_2
    \begin{vmatrix}
      e_1 & e_2 \\
      1 & e_1
    \end{vmatrix}
    -
    \begin{vmatrix}
      e_3 & e_4 \\
      1 & e_1
    \end{vmatrix}
    \\
    &= e_2(e_1^2 - e_2) - (e_3e_1 - e_4) \\
    &= e_2e_1^2 - e_2^2 - e_3e_1 + e_4 \\
    &= e_{(2,1,1)} - e_{(2,2)} - e_{(3,1)} + e_4.
  \end{align*}
  Lastly, from equation 2.32 in the lecture notes, we get
  \begin{align*}
    h_{(3,1)} = m_{(4)} + 2m_{(3,1)} + 2m_{(2,2)} + 3m_{(2,1,1)} + 4m_{(1^4)}
  \end{align*}
  and
  \begin{align*}
    h_4 =  m_{(4)} + m_{(3,1)} + m_{(2,2)} + m_{(2,1,1)} + m_{(1^4)},
  \end{align*}
  so that 
  \begin{align*}
    s_{(3,1)} 
    &= h_{(3,1)} - h_4 \\
    &= (m_{(4)} + 2m_{(3,1)} + 2m_{(2,2)} + 3m_{(2,1,1)} + 4m_{(1^4)})
    -  (m_{(4)} + m_{(3,1)} + m_{(2,2)} + m_{(2,1,1)} + m_{(1^4)}) \\
    &= m_{(3,1)} + m_{(2,2)} + 2m_{(2,1,1)} + 3m_{(1^4)}. 
  \end{align*}
\end{proof}
\end{document}
