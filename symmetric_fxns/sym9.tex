\documentclass[12pt]{extarticle}
%Some packages I commonly use.
\usepackage[english]{babel}
\usepackage{graphicx}
\usepackage{framed}
\usepackage[normalem]{ulem}
\usepackage{amsmath}
\usepackage{amsthm}
\usepackage{amssymb}
\usepackage{amsfonts}
\usepackage{enumerate}
\usepackage{ytableau}
\usepackage[utf8]{inputenc}
\usepackage[top=1 in,bottom=1in, left=1 in, right=1 in]{geometry}

%A bunch of definitions that make my life easier
\newcommand{\matlab}{{\sc Matlab} }
\newcommand{\abs}[1]{|#1|}
\newcommand{\set}[1]{\{#1\}}
\newcommand{\cvec}[1]{{\mathbf #1}}
\newcommand{\rvec}[1]{\vec{\mathbf #1}}
\newcommand{\ihat}{\hat{\textbf{\i}}}
\newcommand{\jhat}{\hat{\textbf{\j}}}
\newcommand{\khat}{\hat{\textbf{k}}}
\newcommand{\minor}{{\rm minor}}
\newcommand{\trace}{{\rm trace}}
\newcommand{\spn}{{\rm Span}}
\newcommand{\rem}{{\rm rem}}
\newcommand{\ran}{{\rm range}}
\newcommand{\range}{{\rm range}}
\newcommand{\mdiv}{{\rm div}}
\newcommand{\proj}{{\rm proj}}
\newcommand{\sgn}{\text{sgn}}
\newcommand{\R}{\mathbb{R}}
\newcommand{\N}{\mathbb{N}}
\newcommand{\Q}{\mathbb{Q}}
\newcommand{\Z}{\mathbb{Z}}
\newcommand{\<}{\langle}
\renewcommand{\>}{\rangle}
\renewcommand{\emptyset}{\varnothing}
\newcommand{\attn}[1]{\textbf{#1}}
\theoremstyle{definition}
\newtheorem{theorem}{Theorem}
\newtheorem{prob}{Problem}
\newtheorem{corollary}{Corollary}
\newtheorem*{definition}{Definition}
\newtheorem*{example}{Example}
\newtheorem*{note}{Note}
\newtheorem{exercise}{Exercise}
\newcommand{\bproof}{\bigskip {\bf Proof. }}
\newcommand{\eproof}{\hfill\qedsymbol}
\newcommand{\Disp}{\displaystyle}
\newcommand{\qe}{\hfill\(\bigtriangledown\)}
\newcommand\restr[2]{{% we make the whole thing an ordinary symbol
  \left.\kern-\nulldelimiterspace % automatically resize the bar with \right
  #1 % the function
  \vphantom{\big|} % pretend it's a little taller at normal size
  \right|_{#2} % this is the delimiter
  }}
\setlength{\columnseprule}{1 pt}


\title{ Math 4995/5327
  \\
  Assignment 9}
\author{David Draguta}

\begin{document}

\maketitle

\begin{exercise}
  Prove that the involution $\omega$ of Heis defined by
  \begin{align*}
    \omega: Heis \to Heis, \quad \omega(a)= \omega \circ a \circ \omega, \quad a \in Heis,
  \end{align*}
  where on the right-hand side, $\omega$ is the involution of $Sym$ given by $\omega(e_r) = h_r$, satisfies $\omega(f^{\pm}) = (\omega(f))^{\pm}$.
\end{exercise}
\begin{proof}
  Let $f \in Sym$, then, $f^+ \in Heis$, and for $u \in Sym$, 
  \begin{align*}
    \omega(f^+)(u)
    &=
    (\omega \circ f^+ \circ \omega)(u) \\
    &=
    (\omega \circ f^+) (\omega(u)) \\
    &=
    \omega ( f\omega(u)) \\
    &=
    \omega(f)\omega^2(u) \\
    &=
    \omega(f)u \\
    &=\omega(f)^+(u).
  \end{align*}
  Since this holds for arbitrary $u \in Sym$ it holds for all $u \in Sym$, and $\omega(f^+) = \omega(f)^+$.

  We also have $f^- \in Heis$, and for $u \in Sym$,
  \begin{align*}
    \omega(f^-)(u)
    &=
    (\omega \circ f^- \circ \omega)(u) \\
    &=
    (\omega \circ f^-) \circ (\omega(u)) \\
    &=
    \omega \circ (f^-(\omega(u))) \\
    &=
    \omega \circ (\sum\limits_{\lambda \in Par} (\omega(u), fs_\lambda)s_\lambda) \\
    &=
    \sum\limits_{\lambda \in Par} (\omega(u), fs_\lambda)s_{\lambda^'} \\
    &=
    \sum\limits_{\lambda \in Par} (u, \omega(f)s_{\lambda^{'}}) s_{\lambda^'} \\
    &=
    \sum\limits_{\lambda \in Par} ({\omega(f)}^-u, s_{\lambda^{'}}) s_{\lambda^{'}} \\
    &=
    {\omega(f)}^-(u).
  \end{align*}
  Since this holds for any symmetric function $u$, we get $\omega(f^-) = \omega(f)^-$.

  We've shown that for all symmetric functions $f \in Sym$, we have $\omega(f^{\pm}) = \omega(f)^{\pm}$. 
\end{proof}

\begin{exercise}
  Suppose that
  \begin{align*}
    f = \sum\limits_{\lambda \in Par} c_\lambda p_\lambda,
  \end{align*}
  for some $c_\lambda \in \Q$, $\lambda \in Par$, only finitely many of which are non zero. Choose $\lambda$ of maximal size amongst partitions such that $c_\lambda \neq 0$.

  Show that
  \begin{align*}
    p_\lambda^-f &\overset{4.2.7}{=} (\prod\limits_{i=1}^{\ell(\lambda)} \lambda_i \cfrac{\partial}{\partial p_{\lambda_i}}) f
  \end{align*}
  is a nonzero constant polynomial.
\end{exercise}
\begin{proof}
  Let $\lambda \in Par$ be maximal (under partition inclusion $\subseteq$) such that $c_\lambda \neq 0$, and let $\mu \in Par$. If $\lambda$ and $\mu$ are not comparable under partition inclusion, then there's some $\lambda_i$ that does not appear as a part in $\mu$, and so 
  \begin{align*}
    \cfrac{\partial}{\partial p_{\lambda_i}}p_\mu = 0.
  \end{align*}
  Otherwise, $\lambda, \mu$ are comparable. If $\lambda \subsetneq \mu$, then by our choice of $\lambda$, $c_\mu = 0$. If $\mu \subsetneq \lambda$, we have either $\lambda$ has a distinct part $\lambda_i$, and $\cfrac{\partial}{\partial p_{\lambda_i}}p_\mu = 0$, or $\lambda$ and $\mu$ have the same parts, but $\lambda$ has a greater multiplicity of at least one part $\lambda_j$, and $(\cfrac{\partial}{\partial \lambda_j})^{m_{\lambda_j}(\lambda)}(p_\mu) = 0$. In either case, the result is that
  \begin{align*}
    \prod\limits_{i=1}^{\ell(\lambda)}\cfrac{\partial}{\partial p_{\lambda_i}}p_\mu = 0.
  \end{align*}
  We use all of these facts in the following:
  \end{align*}
  \begin{align*}
    p_\lambda^-f
    &=
    (\prod\limits_{i=1}^{\ell(\lambda)} \lambda_i \cfrac{\partial}{\partial p_{\lambda_i}}) f \\
    &=
    \sum\limits_{\mu \in Par}c_\mu\prod\limits_{i=1}^{\ell(\lambda)} \lambda_i \cfrac{\partial}{\partial p_{\lambda_i}}p_\mu \\
    &=
    \sum\limits_{\lambda \subseteq \mu}c_\mu\prod\limits_{i=1}^{\ell(\lambda)} \lambda_i \cfrac{\partial}{\partial p_{\lambda_i}}p_\mu \\
    &=
    c_\lambda\prod\limits_{i=1}^{\ell(\lambda)} \lambda_i \cfrac{\partial}{\partial p_{\lambda_i}}p_\lambda \\
    &=
    c_\lambda\prod\limits_{i=1}^{\ell(\lambda)} \lambda_i m_{\lambda_i}(\lambda) \in \Q .\\
  \end{align*}
  
\end{proof}
\end{document}
