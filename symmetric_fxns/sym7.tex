\documentclass[12pt]{extarticle}
%Some packages I commonly use.
\usepackage[english]{babel}
\usepackage{graphicx}
\usepackage{framed}
\usepackage[normalem]{ulem}
\usepackage{amsmath}
\usepackage{amsthm}
\usepackage{amssymb}
\usepackage{amsfonts}
\usepackage{enumerate}
\usepackage{ytableau}
\usepackage[utf8]{inputenc}
\usepackage[top=1 in,bottom=1in, left=1 in, right=1 in]{geometry}

%A bunch of definitions that make my life easier
\newcommand{\matlab}{{\sc Matlab} }
\newcommand{\abs}[1]{|#1|}
\newcommand{\set}[1]{\{#1\}}
\newcommand{\cvec}[1]{{\mathbf #1}}
\newcommand{\rvec}[1]{\vec{\mathbf #1}}
\newcommand{\ihat}{\hat{\textbf{\i}}}
\newcommand{\jhat}{\hat{\textbf{\j}}}
\newcommand{\khat}{\hat{\textbf{k}}}
\newcommand{\minor}{{\rm minor}}
\newcommand{\trace}{{\rm trace}}
\newcommand{\spn}{{\rm Span}}
\newcommand{\rem}{{\rm rem}}
\newcommand{\ran}{{\rm range}}
\newcommand{\range}{{\rm range}}
\newcommand{\mdiv}{{\rm div}}
\newcommand{\proj}{{\rm proj}}
\newcommand{\sgn}{\text{sgn}}
\newcommand{\R}{\mathbb{R}}
\newcommand{\N}{\mathbb{N}}
\newcommand{\Q}{\mathbb{Q}}
\newcommand{\Z}{\mathbb{Z}}
\newcommand{\<}{\langle}
\renewcommand{\>}{\rangle}
\renewcommand{\emptyset}{\varnothing}
\newcommand{\attn}[1]{\textbf{#1}}
\theoremstyle{definition}
\newtheorem{theorem}{Theorem}
\newtheorem{prob}{Problem}
\newtheorem{corollary}{Corollary}
\newtheorem*{definition}{Definition}
\newtheorem*{example}{Example}
\newtheorem*{note}{Note}
\newtheorem{exercise}{Exercise}
\newcommand{\bproof}{\bigskip {\bf Proof. }}
\newcommand{\eproof}{\hfill\qedsymbol}
\newcommand{\Disp}{\displaystyle}
\newcommand{\qe}{\hfill\(\bigtriangledown\)}
\newcommand\restr[2]{{% we make the whole thing an ordinary symbol
  \left.\kern-\nulldelimiterspace % automatically resize the bar with \right
  #1 % the function
  \vphantom{\big|} % pretend it's a little taller at normal size
  \right|_{#2} % this is the delimiter
  }}
\setlength{\columnseprule}{1 pt}


\title{ Math 4995/5327
  \\
  Assignment 7}
\author{David Draguta}

\begin{document}

\maketitle

\begin{exercise}
  Let $n \in \Z$, $n \geq 1$. Prove that $\Q \otimes_{\Z} \Z/n\Z$ is the zero $\Z$-module.
\end{exercise}

\begin{proof}
  Let $a \otimes b \in Q \otimes_{\Z}\Z/n\Z$. We have $a/n \in \Q$, so:
  \begin{align*}
    a \otimes b = n\cfrac{a}{n} \otimes b = \cfrac{a}{n} \otimes nb = \cfrac{a}{n} \otimes 0 = 0
  \end{align*}
\end{proof}

\begin{exercise}
  Show that, for $f,g \in Sym$, we have $(f \otimes g)^{\perp} = f^{\perp} \otimes g^{\perp}$.
\end{exercise}

\begin{proof}
  Let $u,v \in Sym$. Then for all $p,q \in Sym$, 
  \begin{align*}
    ((f \otimes g)^{\perp}(u \otimes v), (p \otimes q)) 
    &= (u \otimes v, (f \otimes g)(p \otimes q))\\
    &= (u \otimes v, fp \otimes gq) \\
    &= (u,fp)(v,gq) \\
    &= (f^{\perp} u, p)(g^{\perp}v, q) \\
    &= (f^{\perp}u \otimes g^{\perp}v, p \otimes q) \\
    &= (f^{\perp} \otimes g^{\perp} (u \otimes v), p \otimes q),
  \end{align*}
  and so 
  \begin{align*}
    (f \otimes g)^{\perp}(u \otimes v) = (f^{\perp} \otimes g^{\perp} (u \otimes v).
  \end{align*}

  Since this holds for all $u, v \in Sym$, we conclude
  \begin{align*}
    (f \otimes g)^{\perp} = f^{\perp} \otimes g^{\perp}.
  \end{align*}
\end{proof}
\end{document}
