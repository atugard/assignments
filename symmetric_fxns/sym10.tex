\documentclass[12pt]{extarticle}
%Some packages I commonly use.
\usepackage[english]{babel}
\usepackage{graphicx}
\usepackage{framed}
\usepackage[normalem]{ulem}
\usepackage{amsmath}
\usepackage{amsthm}
\usepackage{amssymb}
\usepackage{amsfonts}
\usepackage{enumerate}
\usepackage{ytableau}
\usepackage[utf8]{inputenc}
\usepackage[top=1 in,bottom=1in, left=1 in, right=1 in]{geometry}

%A bunch of definitions that make my life easier
\newcommand{\matlab}{{\sc Matlab} }
\newcommand{\abs}[1]{|#1|}
\newcommand{\set}[1]{\{#1\}}
\newcommand{\cvec}[1]{{\mathbf #1}}
\newcommand{\rvec}[1]{\vec{\mathbf #1}}
\newcommand{\ihat}{\hat{\textbf{\i}}}
\newcommand{\jhat}{\hat{\textbf{\j}}}
\newcommand{\khat}{\hat{\textbf{k}}}
\newcommand{\minor}{{\rm minor}}
\newcommand{\trace}{{\rm trace}}
\newcommand{\spn}{{\rm Span}}
\newcommand{\rem}{{\rm rem}}
\newcommand{\ran}{{\rm range}}
\newcommand{\range}{{\rm range}}
\newcommand{\mdiv}{{\rm div}}
\newcommand{\proj}{{\rm proj}}
\newcommand{\sgn}{\text{sgn}}
\newcommand{\R}{\mathbb{R}}
\newcommand{\N}{\mathbb{N}}
\newcommand{\Q}{\mathbb{Q}}
\newcommand{\Z}{\mathbb{Z}}
\newcommand{\<}{\langle}
\renewcommand{\>}{\rangle}
\renewcommand{\emptyset}{\varnothing}
\newcommand{\attn}[1]{\textbf{#1}}
\theoremstyle{definition}
\newtheorem{theorem}{Theorem}
\newtheorem{prob}{Problem}
\newtheorem{corollary}{Corollary}
\newtheorem*{definition}{Definition}
\newtheorem*{example}{Example}
\newtheorem*{note}{Note}
\newtheorem{exercise}{Exercise}
\newcommand{\bproof}{\bigskip {\bf Proof. }}
\newcommand{\eproof}{\hfill\qedsymbol}
\newcommand{\Disp}{\displaystyle}
\newcommand{\qe}{\hfill\(\bigtriangledown\)}
\newcommand\restr[2]{{% we make the whole thing an ordinary symbol
  \left.\kern-\nulldelimiterspace % automatically resize the bar with \right
  #1 % the function
  \vphantom{\big|} % pretend it's a little taller at normal size
  \right|_{#2} % this is the delimiter
  }}
\setlength{\columnseprule}{1 pt}


\title{ Math 4995/5327
  \\
  Assignment 9}
\author{David Draguta}

\begin{document}

\maketitle

\begin{exercise}
  It follows from Theorem 5.4.1 and (5.6) that $P_{r}P_{-r} = P_{-r}P_r - r, r \geq 1.$ So what is wrong with the following computation?
  \begin{align*}
    P_r P_{-r}
    &= \sum\limits_{j,k \in \Z} \psi_{j+r}^+ \psi_{j}^-\psi_{k-r}^+\psi_k^- \\
    &\overset{5.10}{=} - \sum\limits_{j,k \in \Z} \psi_{j+r}^+\psi_{k-r}^+\psi_j^-\psi_k^- + \sum\limits_{j,k \in \Z} \delta_{j,k-r}\psi_{j+r}^+\psi_k^- \\
    &\overset{5.10}{=} - \sum\limits_{j,k \in \Z} \psi_{k-r}^+\psi_{j+r}^+\psi_k^-\psi_j^- + \sum\limits_{j \in \Z} \psi_{j+r}^+\psi_{j+r}^- \\
    &\overset{5.10}{=} \sum\limits_{j,k \in \Z} \psi_{k-r}^+\psi_{k}^-\psi_{j+r}^+\psi_j^- - \sum\limits_{j,k \in \Z} \delta_{k,j+r}\psi_{k-r}^+\psi_k^- + \sum\limits_{j \in \Z} \psi_{j+r}^+ \psi_{j+r}^- \\
    &= P_{-r}P_{r} - \sum\limits_{j \in \Z} \psi_{j +r}^+\psi_{j+r}^- + \sum\limits_{j \in \Z} \psi_{j+r}^+\psi_{j+r}^-  \\
    &= P_{-r}P_{r}
  \end{align*}
\end{exercise}
\begin{proof}
  Let $j \in \mathbf{i}$ mean that there exists some non negative $s \in \Z$ such that $i_s = j$. We have, then, that  
  \begin{align*}
    \sum\limits_{j \in \Z} \psi_{j +r}^+\psi_{j+r}^- v_{\mathbf{i}}
    &=
    \sum\limits_{j+r \in \mathbf{i}} v_{\mathbf{i}},
  \end{align*}
  which is an infinite sum of non zero, antisymmetric, infinite rank tensor terms, which in general is not well defined; and so we can't just cancel the expression in the second last line.
\end{proof}

\begin{exercise}
  
\end{exercise}
\end{document}
