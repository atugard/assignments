\documentclass[12pt]{extarticle}
%Some packages I commonly use.
\usepackage[english]{babel}
\usepackage{graphicx}
\usepackage{framed}
\usepackage[normalem]{ulem}
\usepackage{amsmath}
\usepackage{amsthm}
\usepackage{amssymb}
\usepackage{amsfonts}
\usepackage{enumerate}
\usepackage{ytableau}
\usepackage[utf8]{inputenc}
\usepackage[top=1 in,bottom=1in, left=1 in, right=1 in]{geometry}

%A bunch of definitions that make my life easier
\newcommand{\matlab}{{\sc Matlab} }
\newcommand{\abs}[1]{|#1|}
\newcommand{\set}[1]{\{#1\}}
\newcommand{\cvec}[1]{{\mathbf #1}}
\newcommand{\rvec}[1]{\vec{\mathbf #1}}
\newcommand{\ihat}{\hat{\textbf{\i}}}
\newcommand{\jhat}{\hat{\textbf{\j}}}
\newcommand{\khat}{\hat{\textbf{k}}}
\newcommand{\minor}{{\rm minor}}
\newcommand{\trace}{{\rm trace}}
\newcommand{\spn}{{\rm Span}}
\newcommand{\rem}{{\rm rem}}
\newcommand{\ran}{{\rm range}}
\newcommand{\range}{{\rm range}}
\newcommand{\mdiv}{{\rm div}}
\newcommand{\proj}{{\rm proj}}
\newcommand{\sgn}{\text{sgn}}
\newcommand{\R}{\mathbb{R}}
\newcommand{\N}{\mathbb{N}}
\newcommand{\Q}{\mathbb{Q}}
\newcommand{\Z}{\mathbb{Z}}
\newcommand{\<}{\langle}
\renewcommand{\>}{\rangle}
\renewcommand{\emptyset}{\varnothing}
\newcommand{\attn}[1]{\textbf{#1}}
\theoremstyle{definition}
\newtheorem{theorem}{Theorem}
\newtheorem{prob}{Problem}
\newtheorem{corollary}{Corollary}
\newtheorem*{definition}{Definition}
\newtheorem*{example}{Example}
\newtheorem*{note}{Note}
\newtheorem{exercise}{Exercise}
\newcommand{\bproof}{\bigskip {\bf Proof. }}
\newcommand{\eproof}{\hfill\qedsymbol}
\newcommand{\Disp}{\displaystyle}
\newcommand{\qe}{\hfill\(\bigtriangledown\)}
\newcommand\restr[2]{{% we make the whole thing an ordinary symbol
  \left.\kern-\nulldelimiterspace % automatically resize the bar with \right
  #1 % the function
  \vphantom{\big|} % pretend it's a little taller at normal size
  \right|_{#2} % this is the delimiter
  }}
\setlength{\columnseprule}{1 pt}

\title{ Math 4995/5327
  \\
  Assignment 10}
\author{David Draguta}

\begin{document}

\maketitle

\begin{exercise}
  It follows from Theorem 5.4.1 and (5.6) that $P_{r}P_{-r} = P_{-r}P_r - r, r \geq 1.$ So what is wrong with the following computation?
  \begin{align*}
    P_r P_{-r}
    &= \sum\limits_{j,k \in \Z} \psi_{j+r}^+ \psi_{j}^-\psi_{k-r}^+\psi_k^- \\
    &\overset{5.10}{=} - \sum\limits_{j,k \in \Z} \psi_{j+r}^+\psi_{k-r}^+\psi_j^-\psi_k^- + \sum\limits_{j,k \in \Z} \delta_{j,k-r}\psi_{j+r}^+\psi_k^- \\
    &\overset{5.10}{=} - \sum\limits_{j,k \in \Z} \psi_{k-r}^+\psi_{j+r}^+\psi_k^-\psi_j^- + \sum\limits_{j \in \Z} \psi_{j+r}^+\psi_{j+r}^- \\
    &\overset{5.10}{=} \sum\limits_{j,k \in \Z} \psi_{k-r}^+\psi_{k}^-\psi_{j+r}^+\psi_j^- - \sum\limits_{j,k \in \Z} \delta_{k,j+r}\psi_{k-r}^+\psi_k^- + \sum\limits_{j \in \Z} \psi_{j+r}^+ \psi_{j+r}^- \\
    &= P_{-r}P_{r} - \sum\limits_{j \in \Z} \psi_{j +r}^+\psi_{j+r}^- + \sum\limits_{j \in \Z} \psi_{j+r}^+\psi_{j+r}^-  \\
    &= P_{-r}P_{r}
  \end{align*}
\end{exercise}
\begin{proof}
  Let $v_{\mathbf{i}}$ be a semi infinite wedge. Then
  \begin{align*}
    \sum\limits_{j \in \Z} \psi_{j +r}^+\psi_{j+r}^- v_{\mathbf{i}}
    &=
    \sum\limits_{\set{j \in \Z: j+r \text{ appears in } \mathbf{i}}} v_{\mathbf{i}} 
  \end{align*}
  is an infinite sum, which in general isn't well defined. The problem in the computation is that we're algebraically manipulating this sum as though it were well defined, and in particular in the second last line we cancel the sums, arriving at our conclusion, but that operation of cancellation of two not well defined sums is not well defined.
\end{proof}
\newpage
\begin{exercise}
  With $\varphi(tu)$ as in 5.42, show that
  \begin{align*}
    f(t)\varphi(tu) = f(u^{-1})\varphi(tu), \quad \forall f(t) \in \Q[t]
  \end{align*}
\end{exercise}
\begin{proof}
  Let $f(t) = \sum\limits_{n \geq 0} a_n t^n$, for $a_n \in \Q$; then
  \begin{align*}
    f(t)\varphi(tu)
    &=
    \sum\limits_{n \geq 0} a_n t^n (\sum\limits_{m \in \Z} (tu)^m) \\
    &=
    \sum\limits_{n \geq 0} a_n \sum\limits_{m \in \Z} (t^{n+m} u^{m}) \\
    &=
    \sum\limits_{n \geq 0} a_nu^{-n} \sum\limits_{m \in \Z} (tu)^{n+m} \\
    &=
    \sum\limits_{n \geq 0} a_nu^{-n} \sum\limits_{k \in \Z} (tu)^{k} \\
    &=
    f(u^{-1}) \varphi(tu)
  \end{align*}
\end{proof}
\begin{exercise}
  Prove that $B^-(t)B^-(u) +tu^{-1}B^-(u)B^-(t) = 0$.
\end{exercise}
\begin{proof}
  \begin{align*}
    B^-(t)B^-(u)
    &=
    E^+(-t^{-1})H^-(t) E^+(-u^{-1})H^-(u) \\
    &\overset{5.34}{=}
    (1-tu^{-1})E^+(-t^{-1})E^+(-u^{-1})H^{-}(t)H^-(u) \\
    &=
    (1-tu^{-1})E^+(-u^{-1})E^+(-t^{-1})H^-(u)H^{-}(t) \\
    &=
    -tu^{-1}(1-ut^{-1})E^+(-u^{-1})E^+(-t^{-1})H^-(u)H^{-}(t) \\
    &\overset{5.34}{=}
    -tu^{-1}E^+(-u^{-1})H^-(u)E^+(-t^{-1})H^{-}(t)  \\
    &=
    -tu^{-1}B^-(u)B^-(t)  \\
  \end{align*}
\end{proof}
\end{document}
