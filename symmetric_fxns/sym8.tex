\documentclass[12pt]{extarticle}
%Some packages I commonly use.
\usepackage[english]{babel}
\usepackage{graphicx}
\usepackage{framed}
\usepackage[normalem]{ulem}
\usepackage{amsmath}
\usepackage{amsthm}
\usepackage{amssymb}
\usepackage{amsfonts}
\usepackage{enumerate}
\usepackage{ytableau}
\usepackage[utf8]{inputenc}
\usepackage[top=1 in,bottom=1in, left=1 in, right=1 in]{geometry}

%A bunch of definitions that make my life easier
\newcommand{\matlab}{{\sc Matlab} }
\newcommand{\abs}[1]{|#1|}
\newcommand{\set}[1]{\{#1\}}
\newcommand{\cvec}[1]{{\mathbf #1}}
\newcommand{\rvec}[1]{\vec{\mathbf #1}}
\newcommand{\ihat}{\hat{\textbf{\i}}}
\newcommand{\jhat}{\hat{\textbf{\j}}}
\newcommand{\khat}{\hat{\textbf{k}}}
\newcommand{\minor}{{\rm minor}}
\newcommand{\trace}{{\rm trace}}
\newcommand{\spn}{{\rm Span}}
\newcommand{\rem}{{\rm rem}}
\newcommand{\ran}{{\rm range}}
\newcommand{\range}{{\rm range}}
\newcommand{\mdiv}{{\rm div}}
\newcommand{\proj}{{\rm proj}}
\newcommand{\sgn}{\text{sgn}}
\newcommand{\R}{\mathbb{R}}
\newcommand{\N}{\mathbb{N}}
\newcommand{\Q}{\mathbb{Q}}
\newcommand{\Z}{\mathbb{Z}}
\newcommand{\<}{\langle}
\renewcommand{\>}{\rangle}
\renewcommand{\emptyset}{\varnothing}
\newcommand{\attn}[1]{\textbf{#1}}
\theoremstyle{definition}
\newtheorem{theorem}{Theorem}
\newtheorem{prob}{Problem}
\newtheorem{corollary}{Corollary}
\newtheorem*{definition}{Definition}
\newtheorem*{example}{Example}
\newtheorem*{note}{Note}
\newtheorem{exercise}{Exercise}
\newcommand{\bproof}{\bigskip {\bf Proof. }}
\newcommand{\eproof}{\hfill\qedsymbol}
\newcommand{\Disp}{\displaystyle}
\newcommand{\qe}{\hfill\(\bigtriangledown\)}
\newcommand\restr[2]{{% we make the whole thing an ordinary symbol
  \left.\kern-\nulldelimiterspace % automatically resize the bar with \right
  #1 % the function
  \vphantom{\big|} % pretend it's a little taller at normal size
  \right|_{#2} % this is the delimiter
  }}
\setlength{\columnseprule}{1 pt}


\title{ Math 4995/5327
  \\
  Assignment 8}
\author{David Draguta}

\begin{document}

\maketitle

\begin{exercise}
  Find the group of grouplike elements of Sym.
\end{exercise}
\begin{proof}
  Let $f \in Sym$ be grouplike. Then, for all $\lambda, \mu \in Par$ we have
  \begin{align*}
    (f, s_{\lambda} s_{\mu}) = (\triangle f, s_{\lambda} \otimes s_{\mu}) = (f \otimes f, s_{\lambda} \otimes s_{\mu}) = (f, s_{\lambda})(f, s_\mu).
  \end{align*}
  Hence,
  \begin{align*}
    s_{\lambda}^{\perp} f = \sum\limits_{\mu} (f, s_\lambda s_\mu) s_{\mu} =  \sum\limits_{\mu} (f, s_{\lambda})(f, s_\mu)s_{\mu} =  (f, s_{\lambda})\sum\limits_{\mu}(f, s_\mu)s_{\mu} = (f,s_\lambda)f.
  \end{align*}
  We define
  \begin{align*}
    G = \set{f \in Sym: s_\lambda^{\perp} f = (f,s_\lambda)f, \text{ for all } \lambda \in Par}.
  \end{align*}
  If $f$ is grouplike, then we just showed $f \in G$.

  For any $f \in Sym$ we have, by an equality demonstrated in the proof of Proposition 4.4.5 in the course notes, that
  \begin{align*}
    \triangle f = \sum\limits_{\lambda \in Par} s_\lambda^{\perp}f \otimes s_{\lambda}.
  \end{align*}

  Using this, we get for $f \in G$
  \begin{align*}
    \triangle f
    &= 
    \sum\limits_{\lambda \in Par} s_\lambda^{\perp}f \otimes s_{\lambda} \\
    &=  \sum\limits_{\lambda \in Par} (f,s_\lambda)f \otimes s_{\lambda} \\
    &=  \sum\limits_{\lambda \in Par} f \otimes (f,s_\lambda)s_{\lambda} \\
    &= f \otimes \sum\limits_{\lambda \in Par}(f,s_\lambda)s_{\lambda} \\
    &=
    f \otimes f,
  \end{align*}
  i.e. that $f$ is grouplike. We've shown $G$ contains exactly the grouplike elements of $Sym$. That $(G, \nabla)$ is indeed a group is given by Exercise 4.3.2 in the course notes.
\end{proof}
\begin{exercise}
  Use the Murnaghan-Nakayama rule to write the power sum $p_r$, $r \geq 1$, as a linear combination of Schur functions.
\end{exercise}
\begin{proof}
  Setting $\lambda= \emptyset$ in Proposition 3.7.1 of the course notes yields
  \begin{align*}
    p_r  = \sum\limits_{\mu}(-1)^{ht(\mu)} s_\mu,
  \end{align*}
  where the sum is over all border strip partitions $\mu \in Par$ with $r$ boxes, i.e. the sum's over  partitions $(r), (r-1,1), (r-2,1,1), (r-3,1,1,1), \dots, (1^r)$ with corresponding heights
  $0,1,2,3, \dots, r-1$; and so this reduces to
  \begin{align*}
    p_r =  s_{(r)} - s_{(r-1,1)} + s_{(r-2,1,1)} + \dots + (-1)^{r-1} s_{(1^r)}.
  \end{align*}
\end{proof}
\end{document}
