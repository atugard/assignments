\documentclass[12pt]{extarticle}
%Some packages I commonly use.
\usepackage[english]{babel}
\usepackage{graphicx}
\usepackage{framed}
\usepackage[normalem]{ulem}
\usepackage{amsmath}
\usepackage{amsthm}
\usepackage{amssymb}
\usepackage{amsfonts}
\usepackage{enumerate}
\usepackage{ytableau}
\usepackage[utf8]{inputenc}
\usepackage[top=1 in,bottom=1in, left=1 in, right=1 in]{geometry}

%A bunch of definitions that make my life easier
\newcommand{\matlab}{{\sc Matlab} }
\newcommand{\abs}[1]{|#1|}
\newcommand{\set}[1]{\{#1\}}
\newcommand{\cvec}[1]{{\mathbf #1}}
\newcommand{\rvec}[1]{\vec{\mathbf #1}}
\newcommand{\ihat}{\hat{\textbf{\i}}}
\newcommand{\jhat}{\hat{\textbf{\j}}}
\newcommand{\khat}{\hat{\textbf{k}}}
\newcommand{\minor}{{\rm minor}}
\newcommand{\trace}{{\rm trace}}
\newcommand{\spn}{{\rm Span}}
\newcommand{\rem}{{\rm rem}}
\newcommand{\ran}{{\rm range}}
\newcommand{\range}{{\rm range}}
\newcommand{\mdiv}{{\rm div}}
\newcommand{\proj}{{\rm proj}}
\newcommand{\sgn}{\text{sgn}}
\newcommand{\R}{\mathbb{R}}
\newcommand{\N}{\mathbb{N}}
\newcommand{\Q}{\mathbb{Q}}
\newcommand{\Z}{\mathbb{Z}}
\newcommand{\<}{\langle}
\renewcommand{\>}{\rangle}
\renewcommand{\emptyset}{\varnothing}
\newcommand{\attn}[1]{\textbf{#1}}
\theoremstyle{definition}
\newtheorem{theorem}{Theorem}
\newtheorem{prob}{Problem}
\newtheorem{corollary}{Corollary}
\newtheorem*{definition}{Definition}
\newtheorem*{example}{Example}
\newtheorem*{note}{Note}
\newtheorem{exercise}{Exercise}
\newcommand{\bproof}{\bigskip {\bf Proof. }}
\newcommand{\eproof}{\hfill\qedsymbol}
\newcommand{\Disp}{\displaystyle}
\newcommand{\qe}{\hfill\(\bigtriangledown\)}
\newcommand\restr[2]{{% we make the whole thing an ordinary symbol
  \left.\kern-\nulldelimiterspace % automatically resize the bar with \right
  #1 % the function
  \vphantom{\big|} % pretend it's a little taller at normal size
  \right|_{#2} % this is the delimiter
  }}
\setlength{\columnseprule}{1 pt}


\title{ Math 4995/5327
  \\
  Assignment 6}
\author{David Draguta}

\begin{document}

\maketitle

\begin{exercise}
  In this exercise we consider partitions of $n=4$.
  \begin{enumerate}
  \item
    Write the partitions of $n$ in reverse lexicographic order.
  \item
    Write down the matrices $J,K,L,\epsilon,$ and $z$.
  \item
    Compute $M(h,f), M(s,e), M(p,f), M(p,h)$.
  \item
    Write $p_{(1^n)}$ as a linear combination of the forgotten symmetric functions.
  \item
    Write $p_n$ as a linear combination of the complete homogeneous symmetric functions. 
  \end{enumerate}
\end{exercise}
\begin{proof}
  \begin{enumerate}
  \item
    $(4) >_{lex} (3,1) >_{lex} (2,2) >_{lex} (2,1,1) >_{lex} (1^4)$.
  \item
    \begin{align*}
      (4)' &= (1^4), \\
      (3,1)' &= (2,1,1), \\
      (2,2)'&=(2,2), \\
      (2,1,1)'&=(3,1), \\
      (1^4)'&=(4) ,
    \end{align*}
    so
    \begin{align*}
      \begin{pmatrix}
        0 & 0 & 0 & 0 & 1 \\
        0 & 0 & 0 & 1 & 0 \\
        0 & 0 & 1 & 0 & 0 \\
        0 & 1 & 0 & 0 & 0 \\
        1 & 0 & 0 & 0 & 0
      \end{pmatrix}.
    \end{align*}
    $K$ is dominant upper unitriangular, so we need only compute $K_{\lambda,\mu}$, for $\lambda>_{lex} \mu$, i.e. the number of semistandard tableaux of this shape and weight. We now compute the matrix $K$ row by row.
    \begin{itemize}
    \item
      $\lambda = (4)$:

      $K_{\lambda, (3,1)} = 1$:
      \begin{align*}
        \begin{ytableau}
          1 & 1 & 1 & 2 
        \end{ytableau}\quad.
      \end{align*}

      $K_{\lambda, (2,2)} = 1$:
      \begin{align*}
        \begin{ytableau}
          1 & 1 & 2 & 2 
        \end{ytableau}\quad.
      \end{align*}

      $K_{\lambda, (2,1,1)} = 1$:
      \begin{align*}
        \begin{ytableau}
          1 & 1 & 2 & 3 
        \end{ytableau}\quad.
      \end{align*}

      $K_{\lambda, (1^4)} = 1$:
      \begin{align*}
        \begin{ytableau}
          1 & 2 & 3 & 4 
        \end{ytableau}\quad.
      \end{align*}
      The first row is $(1,1,1,1,1)$.

    \item
      $\lambda=(3,1)$:
      
      $K_{\lambda, (2,2)} = 1$:
      \begin{align*}
        \begin{ytableau}
          1 & 1 & 2 \\
          2
        \end{ytableau}\quad.
      \end{align*}

      $K_{\lambda, (2,1,1)} = 2$:
      \begin{align*}
        \begin{ytableau}
          1 & 1 & 3 \\
          2
        \end{ytableau}\quad,\quad
        \begin{ytableau}
          1 & 1 & 2 \\
          3
        \end{ytableau}\quad.
      \end{align*}

      $K_{\lambda, (1^4)} = 3 $:
      \begin{align*}
        \begin{ytableau}
          1 & 3 & 4 \\
          2
        \end{ytableau}\quad,\quad
        \begin{ytableau}
          1 & 2 & 4 \\
          3
        \end{ytableau}\quad,\quad
        \begin{ytableau}
          1 & 2 & 3 \\
          4
        \end{ytableau}\quad.        
      \end{align*}
      The second row is $(0,1,1,2,3)$.
    \item
      $\lambda=(2,2)$:

      $K_{\lambda, (2,1,1)} = 1$:
      \begin{align*}
        \begin{ytableau}
          1 & 1\\
          2 & 3
        \end{ytableau}\quad.
      \end{align*}

      $K_{\lambda, (1^4)} = 2$:
      \begin{align*}
        \begin{ytableau}
          1 & 3\\
          2 & 4
        \end{ytableau}\quad,\quad
        \begin{ytableau}
          1 & 2\\
          3 & 4
        \end{ytableau}\quad.
      \end{align*}
      The third row is $(0,0,1,1,2)$.
    \item
      $\lambda=(2,1,1)$:
      
      $K_{\lambda, (1^4)} = 3$:
      \begin{align*}
        \begin{ytableau}
          1 & 2 \\
          3  \\
          4 
        \end{ytableau}\quad,\quad
        \begin{ytableau}
          1 & 3 \\
          2  \\
          4 
        \end{ytableau}\quad,\quad
        \begin{ytableau}
          1 & 4 \\
          2  \\
          3
        \end{ytableau}\quad.
      \end{align*}
      The fourth row is $(0,0,0,1,3)$.
    \item
      $\lambda = (1^4)$:

      There's nothing to do here. The fifth row is $(0,0,0,0,1)$.
    \end{itemize}
    Putting our rows into a matrix we get:
    \begin{align*}
      K=
      \begin{pmatrix}
        1 & 1 & 1 & 1 & 1 \\
        0 & 1 & 1 & 2 & 3 \\
        0 & 0 & 1 & 1 & 2 \\
        0 & 0 & 0 & 1 & 3 \\
        0 & 0 & 0 & 0 & 1 
      \end{pmatrix}
    \end{align*}

    $L$ is dominant lower triangular, so we need only check $\lambda \leq_{lex} \mu$. $L_{\lambda, \mu}$ is the number of Young Tableaux of shape $\lambda$ and weight $\mu$ with constant rows.
    We proceed as before, computing the rows of the matrix, one by one.

    \begin{itemize}
    \item
      $\lambda=(4)$:
      
      $L_{\lambda, (4)} = 1$:
      \begin{align*}
        \begin{ytableau}
          1 & 1 & 1 & 1
        \end{ytableau}\quad.      
      \end{align*}
      The first row is $(1,0,0,0,0)$.
    \item
      $\lambda = (3,1)$:

      $L_{\lambda, (4)} = 1$:
      \begin{align*}
        \begin{ytableau}
          1 & 1 & 1 \\
          1
        \end{ytableau}\quad.      
      \end{align*}      

      $L_{\lambda, (3,1)} = 1$:
      \begin{align*}
        \begin{ytableau}
          1 & 1 & 1 \\
          2
        \end{ytableau}\quad.      
      \end{align*}

      The second row is $(1,1,0,0,0)$.
      
    \item
      $\lambda=(2,2)$:

      $L_{\lambda, (4)} = 1$:
      \begin{align*}
        \begin{ytableau}
          1 & 1 \\
          1 & 1
        \end{ytableau}
      \end{align*}

      $L_{\lambda, (3,1)} = 0$: If we place $2$ ones in a row, we're left to place a $1$ and a $2$, and so can't make up a second constant row. 

      $L_{\lambda, (2,2)} = 2$:
      \begin{align*}
        \begin{ytableau}
          1 & 1 \\
          2 & 2
        \end{ytableau}\quad,\quad
        \begin{ytableau}
          2 & 2 \\
          1 & 1
        \end{ytableau}\quad.              
      \end{align*}
      
      The third row is $(1,0,2,0,0)$.

    \item 
      $\lambda=(2,1,1)$:

      $L_{\lambda,(4)} = 1$:
      \begin{align*}
        \begin{ytableau}
          1 & 1 \\
          1 \\
          1
        \end{ytableau}
      \end{align*}

      $L_{\lambda, (3,1)} = 2 $:
      \begin{align*}
        \begin{ytableau}
          1 & 1 \\
          1 \\
          2
        \end{ytableau}\quad,\quad
        \begin{ytableau}
          1 & 1 \\
          2 \\
          1
        \end{ytableau}.
      \end{align*}

      $L_{\lambda, (2,2)} = 2 $:
      \begin{align*}
        \begin{ytableau}
          1 & 1 \\
          2 \\
          2
        \end{ytableau}\quad,\quad
        \begin{ytableau}
          2 & 2 \\
          1 \\
          1
        \end{ytableau}.
      \end{align*}

      $L_{\lambda, (2,1,1)} = 3 $:
      \begin{align*}
        \begin{ytableau}
          1 & 1 \\
          2 \\
          3
        \end{ytableau}\quad,\quad
        \begin{ytableau}
          1 & 1 \\
          3 \\
          2
        \end{ytableau} \quad .
      \end{align*}
      The third row is $(1,2,2,3,0)$.
    \item
      $\lambda(1^4)$:

      $L_{\lambda, (4)} = 1 $:
      \begin{align*}
        \begin{ytableau}
          1 \\
          1 \\
          1 \\
          1
        \end{ytableau}\quad.
      \end{align*}

      $L_{\lambda, (3,1)} = 4 $:
      \begin{align*}
        \begin{ytableau}
          1 \\
          1 \\
          1 \\
          2
        \end{ytableau}\quad,\quad
        \begin{ytableau}
          1 \\
          1 \\
          2 \\
          1
        \end{ytableau}\quad,\quad
        \begin{ytableau}
          1 \\
          2 \\
          1 \\
          1
        \end{ytableau}\quad,\quad
        \begin{ytableau}
          2 \\
          1 \\
          1 \\
          1
        \end{ytableau}\quad.\quad            
      \end{align*}

      $L_{\lambda, (2,2)} = 6$:

      \begin{align*}
        \begin{ytableau}
          1 \\
          1 \\
          2 \\
          2
        \end{ytableau}\quad,\quad
        \begin{ytableau}
          1 \\
          2 \\
          1 \\
          2
        \end{ytableau}\quad,\quad
        \begin{ytableau}
          1 \\
          2 \\
          2 \\
          1
        \end{ytableau}\quad,\quad 
        \begin{ytableau}
          2 \\
          1 \\
          1 \\
          2
        \end{ytableau}\quad,\quad
        \begin{ytableau}
          2 \\
          1 \\
          2 \\
          1
        \end{ytableau}\quad,\quad
        \begin{ytableau}
          2 \\
          2 \\
          1 \\
          1
        \end{ytableau}\quad.\quad          
      \end{align*}

      $L_{\lambda, (2,1,1)} = 12$:
      \begin{align*}
        &\begin{ytableau}
           1 \\
           1 \\
           2 \\
           3
         \end{ytableau}\quad,\quad
        \begin{ytableau}
          1 \\
          1 \\
          3 \\
          2
        \end{ytableau}\quad,\quad
        \begin{ytableau}
          1 \\
          2 \\
          1 \\
          3
        \end{ytableau}\quad,\quad 
        \begin{ytableau}
          1 \\
          2 \\
          3 \\
          1
        \end{ytableau}\quad,\quad \\
        &\begin{ytableau}
           1 \\
           3 \\
           1 \\
           2
         \end{ytableau}\quad,\quad
        \begin{ytableau}
          1 \\
          3 \\
          2 \\
          1
        \end{ytableau}\quad,\quad 
        \begin{ytableau}
          2 \\
          1 \\
          1 \\
          3
        \end{ytableau}\quad,\quad 
        \begin{ytableau}
          2 \\
          1 \\
          3 \\
          1
        \end{ytableau}\quad,\quad  \\ 
        &\begin{ytableau}
           2 \\
           3 \\
           1 \\
           1
         \end{ytableau}\quad,\quad 
        \begin{ytableau}
          3 \\
          1 \\
          1 \\
          2
        \end{ytableau}\quad,\quad 
        \begin{ytableau}
          3 \\
          1 \\
          2 \\
          1
        \end{ytableau}\quad,\quad      
        \begin{ytableau}
          3 \\
          2 \\
          1 \\
          1
        \end{ytableau}\quad.\quad
      \end{align*}

      $L_{\lambda, (1^4)} = 24$:
      \begin{align*}
        &\begin{ytableau}
           1 \\
           2 \\
           3 \\
           4
         \end{ytableau}\quad,\quad
        \begin{ytableau}
          1 \\
          2 \\
          4 \\
          3
        \end{ytableau}\quad,\quad
        \begin{ytableau}
          1 \\
          3 \\
          2 \\
          4
        \end{ytableau}\quad,\quad
        \begin{ytableau}
          1 \\
          3 \\
          4 \\
          2
        \end{ytableau}\quad,\quad \\
        &\begin{ytableau}
           1 \\
           4 \\
           2 \\
           3
         \end{ytableau}\quad,\quad
        \begin{ytableau}
          1 \\
          4 \\
          3 \\
          2
        \end{ytableau}\quad,\quad
        \begin{ytableau}
          2 \\
          3 \\
          1 \\
          4
        \end{ytableau}\quad,\quad
        \begin{ytableau}
          2 \\
          3 \\
          4 \\
          1
        \end{ytableau}\quad,\quad\\
        &\begin{ytableau}
           2 \\
           4 \\
           1 \\
           3
         \end{ytableau}\quad,\quad
        \begin{ytableau}
          2 \\
          4 \\
          3 \\
          1
        \end{ytableau}\quad,\quad
        \begin{ytableau}
          3 \\
          4 \\
          1 \\
          2
        \end{ytableau}\quad,\quad 
        \begin{ytableau}
          3 \\
          4 \\
          2 \\
          1
        \end{ytableau}\quad,\quad\\
        &\begin{ytableau}
           2 \\
           1 \\
           3 \\
           4
         \end{ytableau}\quad,\quad
        \begin{ytableau}
          2 \\
          1 \\
          4 \\
          3
        \end{ytableau}\quad,\quad
        \begin{ytableau}
          3 \\
          1 \\
          2 \\
          4
        \end{ytableau}\quad,\quad
        \begin{ytableau}
          3 \\
          1 \\
          4 \\
          2
        \end{ytableau}\quad,\quad\\
        &\begin{ytableau}
           3 \\
           2 \\
           1 \\
           4
         \end{ytableau}\quad,\quad
        \begin{ytableau}
          3 \\
          2 \\
          4 \\
          1
        \end{ytableau}\quad,\quad 
        \begin{ytableau}
          4 \\
          1 \\
          2 \\
          3
        \end{ytableau}\quad,\quad
        \begin{ytableau}
          4 \\
          1 \\
          3 \\
          2
        \end{ytableau}\quad,\quad\\
        &\begin{ytableau}
           4 \\
           2 \\
           1 \\
           3
         \end{ytableau}\quad,\quad
        \begin{ytableau}
          4 \\
          2 \\
          3 \\
          1
        \end{ytableau}\quad,\quad
        \begin{ytableau}
          4 \\
          3 \\
          1 \\
          2
        \end{ytableau}\quad,\quad
        \begin{ytableau}
          4 \\
          3 \\
          2 \\
          1
        \end{ytableau}\quad.\quad
      \end{align*}
      The fifth row is $(1,4,6,12,24)$.
    \end{itemize}
    Putting all the rows into a matrix yields
    \begin{align*}
      L = 
      \begin{pmatrix}
        1 & 0 & 0 & 0 & 0 \\
        1 & 1 & 0 & 0 & 0 \\
        1 & 0 & 2 & 0 & 0 \\
        1 & 2 & 2 & 3 & 0 \\
        1 & 4 & 6 & 12 & 24
      \end{pmatrix}
    \end{align*}
    Lastly:
    \begin{align*}
      \epsilon
      &= 
      \begin{pmatrix}
        (-1)^{\abs{\lambda}- \ell(4)} & 0 & 0 & 0 & 0 \\
        0 & (-1)^{\abs{\lambda}-\ell(3,1)} & 0 & 0 & 0 \\
        0 & 0 & (-1)^{\abs{\lambda}-\ell(2,2)} & 0 & 0 \\
        0 & 0 & 0 & (-1)^{\abs{\lambda}-\ell(2,1,1)} & 0 \\
        0 & 0 & 0 & 0 & (-1)^{\abs{\lambda}-\ell(1^4)} 
      \end{pmatrix} \\
      &=
      \begin{pmatrix}
        (-1)^{4- 1} & 0 & 0 & 0 & 0 \\
        0 & (-1)^{4-2} & 0 & 0 & 0 \\
        0 & 0 & (-1)^{4-2} & 0 & 0 \\
        0 & 0 & 0 & (-1)^{4-3} & 0 \\
        0 & 0 & 0 & 0 & (-1)^{4-4} 
      \end{pmatrix} \\
      &=
      \begin{pmatrix}
        -1 & 0 & 0 & 0 & 0 \\
        0 & 1 & 0 & 0 & 0 \\
        0 & 0 & 1 & 0 & 0 \\
        0 & 0 & 0 & -1 & 0 \\
        0 & 0 & 0 & 0 & 1
      \end{pmatrix} \\
      \text{and}
      \\
      z
      &=
      \begin{pmatrix}
        z_{(4)} & 0 & 0 & 0 & 0 \\
        0 & z_{(3,1)} & 0 & 0 & 0 \\
        0 & 0 & z_{(2,2)} & 0 & 0 \\
        0 & 0 & 0 & z_{(2,1,1)} & 0 \\
        0 & 0 & 0 & 0 & z_{(1^4)}
      \end{pmatrix} \\
      &=
      \begin{pmatrix}
        4^11! & 0 & 0 & 0 & 0 \\
        0 & 3^11! & 0 & 0 & 0 \\
        0 & 0 & 2^22! & 0 & 0 \\
        0 & 0 & 0 & 1^22!2^11! & 0 \\
        0 & 0 & 0 & 0 & 1^44!
      \end{pmatrix} \\
      &=
      \begin{pmatrix}
        4 & 0 & 0 & 0 & 0 \\
        0 & 3 & 0 & 0 & 0 \\
        0 & 0 & 8 & 0 & 0 \\
        0 & 0 & 0 & 4 & 0 \\
        0 & 0 & 0 & 0 & 24
      \end{pmatrix}
    \end{align*}
  \end{enumerate}
\end{proof}
\end{document}

