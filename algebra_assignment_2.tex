\documentclass[12pt]{extarticle}
%Some packages I commonly use.
\usepackage[english]{babel}
\usepackage{graphicx}
\usepackage{framed}
\usepackage[normalem]{ulem}
\usepackage{amsmath}
\usepackage{amsthm}
\usepackage{amssymb}
\usepackage{amsfonts}
\usepackage{enumerate}
\usepackage[utf8]{inputenc}
\usepackage[top=1 in,bottom=1in, left=1 in, right=1 in]{geometry}

%A bunch of definitions that make my life easier
\newcommand{\matlab}{{\sc Matlab} }
\newcommand{\abs}[1]{|#1|}
\newcommand{\set}[1]{\{#1\}}
\newcommand{\cvec}[1]{{\mathbf #1}}
\newcommand{\rvec}[1]{\vec{\mathbf #1}}
\newcommand{\ihat}{\hat{\textbf{\i}}}
\newcommand{\jhat}{\hat{\textbf{\j}}}
\newcommand{\khat}{\hat{\textbf{k}}}
\newcommand{\minor}{{\rm minor}}
\newcommand{\trace}{{\rm trace}}
\newcommand{\spn}{{\rm Span}}
\newcommand{\rem}{{\rm rem}}
\newcommand{\ran}{{\rm range}}
\newcommand{\range}{{\rm range}}
\newcommand{\mdiv}{{\rm div}}
\newcommand{\proj}{{\rm proj}}
\newcommand{\R}{\mathbb{R}}
\newcommand{\C}{\mathbb{C}}
\newcommand{\N}{\mathbb{N}}
\newcommand{\Q}{\mathbb{Q}}
\newcommand{\Z}{\mathbb{Z}}
\newcommand{\<}{\langle}
\newcommand{\ideal}{\triangleleft}
\renewcommand{\>}{\rangle}
\renewcommand{\emptyset}{\varnothing}
\newcommand{\attn}[1]{\textbf{#1}}
\theoremstyle{definition}
\newtheorem{theorem}{Theorem}
\newtheorem{prob}{Problem}
\newtheorem{corollary}{Corollary}
\newtheorem*{definition}{Definition}
\newtheorem*{example}{Example}
\newtheorem*{note}{Note}
\newtheorem{exercise}{Exercise}
\newcommand{\bproof}{\bigskip {\bf Proof. }}
\newcommand{\eproof}{\hfill\qedsymbol}
\newcommand{\Disp}{\displaystyle}
\newcommand{\qe}{\hfill\(\bigtriangledown\)}
\setlength{\columnseprule}{1 pt}


\title{ Math 5107 -- Algebra Assignment 2}
\author{David Draguta}
\date{2021-10-16}

\begin{document}

\maketitle

\begin{exercise}
  Are the following rings or modules noetherian, artinian, both, neither?
  \begin{enumerate}
  \item
    $\Z$
  \item
    $\Z/n\Z$
  \item
    $\C[x]$
  \item
    $\C[x,y]$
  \item
    $\Q[x]/(x^3-7x)$
  \item
    $\C[x,y]/(x^3+y^3-1)$
  \item
    $\C[x,x^{-1}]$
  \item
    $C([0,1])$: the ring of continuous real valued functions on $[0,1]$
  \item
    The ring $\Q[x,xy,xy^2,xy^3,\dots] \subseteq \Q[x,y]$
  \item
    $\Q[x^iy^j|j<i\sqrt{2}, i,j \geq 0] \subseteq \Q[x,y]$
  \item
    The $\Z$ module $\Z[1/2]/\Z$.
  \item
    The $\Z$ module $\Q/\Z$.
  \item
    The $k[x]$ module $k(x)/k[x]$.
  \item
    The $k[x]$ module $k[x,x^{-1}]/k[x]$ 
  \end{enumerate}
\end{exercise}

\begin{proof}
  \begin{enumerate}
  \item $\Z$:
    
    For any prime $p$ we can form the infinitely descending chain of ideals
    \begin{align*}
      \Z \supseteq p\Z \supseteq \dots \supseteq p^n\Z \supseteq \dots,
    \end{align*}
    so $\Z$ is not artinian.

    $\Z$ is noetherian, as $\Z$ is a PID.
  \item
    $\Z/n\Z$:

    We have $\abs{\set{a\Z: a \leq n}} \leq n$, i.e. the set of ideals of $\Z/n\Z$ is finite, and so any ascending or descending chain will have to stabilize, making this ring both noetherian and artinian. 
  \item
    $\C[x]$:

    $\C$ is noetherian, so by Hilbert's Basis Theorem, $\C[x]$ is too. $\C[x]$ is not artinian, because for example $(x) \supseteq (x^2) \supseteq (x^3) \supseteq \dots $ doesn't stabilize.
  \item
    $\C[x,y]$:

    $\C[x,y] = \C[x][y]$, so by Hilbert's Basis Theorem, $C[x,y]$ is noetherian; but it's not artinian, because $(x) \supseteq (x^2) \supseteq (x^3) \supseteq \dots $ doesn't stabilize.
  \item
    $\Q[x]/(x^3-7x)$:

    \begin{align*}
      \Q[x]/(x^3-7x) &\cong \Q[x]/(x) \times \Q[x]/(x^2-7) \\
      &\cong \Q \times \Q(\sqrt{7}).
    \end{align*}
    Since ideals in the product of two rings $R \times S$ are of the form $I \times J$ for $I \ideal R$ and $J \ideal S$, and since fields have only trivial ideals, we conclude that the ring in question is both artinian and noetherian.

  \item
    $\C[x,y]/(x^3+y^3-1)$:

    
  \item
    $\C[x,x^{-1}]$:

 s  \item
    $C([0,1])$: the ring of continuous real valued functions on $[0,1]$:

    Consider $I_a = \set{f \in C([0,1]): supp(f) \subseteq [0,a]}$. Then we have
    \begin{align*}
      I_{0} \subseteq I_{1/2} \subseteq \dots \subseteq I_{1-1/n} \subseteq \dots 
    \end{align*}
    and
    \begin{align*}
      I_{1} \supseteq I_{1/2} \supseteq \dots \supseteq I_{1/n} \supseteq \dots 
    \end{align*}
    are ascending and descending chains that don't stabilize in countably many steps, so this ring is neither noetherian nor artinian.

  \item
    The ring $\Q[x,xy,xy^2,xy^3,\dots] \subseteq \Q[x,y]$:

    Since $\Q$ is noetherian, $\Q[x]$ is too, and so is $\Q[x,y]$. Any ascending chain of ideals in the chain on the left will be an ascending chain on the right, and so it must stabilize. Hence, the ring in question is noetherian.

    $(x) \supseteq (xy) \supseteq \dots \supseteq (xy^n) \supseteq \dots $ does not stabilize on the left hand side, so it's not artinian. 
  \item
  \end{enumerate}
\end{proof}

\begin{exercise}
  Describe $\hom_{k^{n \times n}}(k^{n \times r}, k^{n \times s})$
\end{exercise}
\begin{proof}
  Let 
  \begin{align*}
    M = \hom_{k^{n \times n}}(k^{n \times r}, k^{n \times s}).
  \end{align*}
  We can fully describe $M$ by matrices $k^{r \times s}$, so let's write
  \begin{align*}
    M = \set{\alpha_A : A \in k^{r \times s}}.
  \end{align*}
  If $\alpha_A \in M$ and $B \in k^{n \times r}$, then we define how $\alpha_A$ acts on $B$ by
  \begin{align*}
    \alpha_A(B) = BA.
  \end{align*}
  We define the addition operation $+_M: M \times M \to M$ by 
  \begin{align*}
    (\alpha_{A_1} +_M \alpha_{A_2})(B) = \alpha_{A_1 + A_2}(B) = B(A_1 + A_2).
  \end{align*}
  The additive identity is given by $\alpha_0$ for $0 \in k^{r \times s}$, and so $M$ is an additive abelian group.
  
  Scalar multiplication $\cdot_M: k^{n \times n} \times M \to M$ is given by
  \begin{align*}
    (C \cdot_M \alpha_A) (B) = C(\alpha_A(B))
  \end{align*}
  Let's package all of this information as a tuple
  \begin{align*}
    \mathcal{M} = (M, +_M, \cdot_M).
  \end{align*}
  $\mathcal{M}$ fully describes the module in question. 
\end{proof}

\end{document}
