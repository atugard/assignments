\documentclass[12pt]{extarticle}
%Some packages I commonly use.
\usepackage[english]{babel}
\usepackage{graphicx}
\usepackage{framed}
\usepackage[normalem]{ulem}
\usepackage{amsmath}
\usepackage{amsthm}
\usepackage{amssymb}
\usepackage{amsfonts}
\usepackage{enumerate}
\usepackage[utf8]{inputenc}
\usepackage[top=1 in,bottom=1in, left=1 in, right=1 in]{geometry}

%A bunch of definitions that make my life easier
\newcommand{\matlab}{{\sc Matlab} }
\newcommand{\abs}[1]{|#1|}
\newcommand{\set}[1]{\{#1\}}
\newcommand{\cvec}[1]{{\mathbf #1}}
\newcommand{\rvec}[1]{\vec{\mathbf #1}}
\newcommand{\ihat}{\hat{\textbf{\i}}}
\newcommand{\jhat}{\hat{\textbf{\j}}}
\newcommand{\khat}{\hat{\textbf{k}}}
\newcommand{\minor}{{\rm minor}}
\newcommand{\trace}{{\rm trace}}
\newcommand{\spn}{{\rm Span}}
\newcommand{\rem}{{\rm rem}}
\newcommand{\ran}{{\rm range}}
\newcommand{\range}{{\rm range}}
\newcommand{\mdiv}{{\rm div}}
\newcommand{\proj}{{\rm proj}}
\newcommand{\R}{\mathbb{R}}
\newcommand{\C}{\mathbb{C}}
\newcommand{\N}{\mathbb{N}}
\newcommand{\Q}{\mathbb{Q}}
\newcommand{\Z}{\mathbb{Z}}
\newcommand{\<}{\langle}
\newcommand{\ideal}{\triangleleft}
\renewcommand{\>}{\rangle}
\renewcommand{\emptyset}{\varnothing}
\newcommand{\attn}[1]{\textbf{#1}}
\theoremstyle{definition}
\newtheorem{theorem}{Theorem}
\newtheorem{prob}{Problem}
\newtheorem{corollary}{Corollary}
\newtheorem*{definition}{Definition}
\newtheorem*{example}{Example}
\newtheorem*{note}{Note}
\newtheorem{exercise}{Exercise}
\newcommand{\bproof}{\bigskip {\bf Proof. }}
\newcommand{\eproof}{\hfill\qedsymbol}
\newcommand{\Disp}{\displaystyle}
\newcommand{\qe}{\hfill\(\bigtriangledown\)}
\setlength{\columnseprule}{1 pt}


\title{ Math 5107 -- Algebra Assignment 2}
\author{David Draguta}
\date{2021-10-16}

\begin{document}

\maketitle

\begin{exercise}
  Are the following rings or modules noetherian, artinian, both, neither?
  \begin{enumerate}
  \item
    $\Z$
  \item
    $\Z/n\Z$
  \item
    $\C[x]$
  \item
    $\C[x,y]$
  \item
    $\Q[x]/(x^3-7x)$
  \item
    $\C[x,y]/(x^3+y^3-1)$
  \item
    $\C[x,x^{-1}]$
  \item
    $C([0,1])$: the ring of continuous real valued functions on $[0,1]$
  \item
    The ring $\Q[x,xy,xy^2,xy^3,\dots] \subseteq \Q[x,y]$
  \item
    $\Q[x^iy^j|j<i\sqrt{2}, i,j \geq 0] \subseteq \Q[x,y]$
  \item
    The $\Z$ module $\Z[1/2]/\Z$.
  \item
    The $\Z$ module $\Q/\Z$.
  \item
    The $k[x]$ module $k(x)/k[x]$.
  \item
    The $k[x]$ module $k[x,x^{-1}]/k[x]$ 
  \end{enumerate}
\end{exercise}

\begin{proof}
  \begin{enumerate}
  \item $\Z$:
    
    For any prime $p$ we can form the infinitely descending chain of ideals
    \begin{align*}
      \Z \supseteq p\Z \supseteq \dots \supseteq p^n\Z \supseteq \dots,
    \end{align*}
    so $\Z$ is not artinian.

    $\Z$ is noetherian, as $\Z$ is a PID.
  \item
    $\Z/n\Z$:

    We have $\abs{\set{a\Z: a \leq n}} \leq n$, i.e. the set of ideals of $\Z/n\Z$ is finite, and so any ascending or descending chain will have to stabilize, making this ring both noetherian and artinian. 
  \item
    $\C[x]$:

    $\C$ is noetherian, so by Hilbert's Basis Theorem, $\C[x]$ is too. $\C[x]$ is not artinian, because for example $(x) \supseteq (x^2) \supseteq (x^3) \supseteq \dots $ doesn't stabilize.
  \item
    $\C[x,y]$:

    $\C[x,y] = \C[x][y]$, so by Hilbert's Basis Theorem, $C[x,y]$ is noetherian; but it's not artinian, because $(x) \supseteq (x^2) \supseteq (x^3) \supseteq \dots $ doesn't stabilize.
  \item
    $\Q[x]/(x^3-7x)$:

    \begin{align*}
      \Q[x]/(x^3-7x) &\cong \Q[x]/(x) \times \Q[x]/(x^2-7) \\
      &\cong \Q \times \Q(\sqrt{7}).
    \end{align*}
    Since ideals in the product of two rings $R \times S$ are of the form $I \times J$ for $I \ideal R$ and $J \ideal S$, and since fields have only trivial ideals, we conclude that the ring in question is both artinian and noetherian.

  \item
    $\C[x,y]/(x^3+y^3-1)$:

    $(x) \supseteq (x^2) \supseteq (x^3) \supseteq \dots $ doesn't stabilize, so the ring is not artinian. 
    
    As $\C[x,y]$ is noetherian, all ideals are finitely generated, and this remains true when we mod out by $(x^3 + y^3 - 1)$, because the ideals in the ring in question are in one to one correspondence with the ideals of $\C[x,y]$ that contain $(x^3+y^3-1)$, all of which then must be finitely generated, as we can just project the generators onto the quotient ring. That means that $\C[x,y]/(x^3+y^3-1)$ is noetherian.
    
  \item
    $\C[x,x^{-1}]$:

    $\C[x]$ is noetherian, and $\C[x,x^{-1}] = \C[x]_x$ is the localization of the ring at $x$. The localization of a noetherian ring is noetherian, and so the ring in question is noetherian.

    Let $f \in \C[x,x^{-1}] \setminus \C[x,x^{-1}]^x$, i.e. $f$ is not a unit. Then we have
    \begin{align*}
      (f) \supseteq (f^2) \supseteq (f^3) \supseteq \cdots 
    \end{align*}
    does not stabilize, so that the ring in question is not artinian.
    
  \item
    $C([0,1])$: the ring of continuous real valued functions on $[0,1]$:

    Consider $I_a = \set{f \in C([0,1]): supp(f) \subseteq [0,a]}$. Then we have
    \begin{align*}
      I_{0} \subseteq I_{1/2} \subseteq \dots \subseteq I_{1-1/n} \subseteq \dots 
    \end{align*}
    and
    \begin{align*}
      I_{1} \supseteq I_{1/2} \supseteq \dots \supseteq I_{1/n} \supseteq \dots 
    \end{align*}
    are ascending and descending chains that don't stabilize in countably many steps, so this ring is neither noetherian nor artinian.

  \item
    The ring $\Q[x,xy,xy^2,xy^3,\dots] \subseteq \Q[x,y]$:

    Since $\Q$ is noetherian, $\Q[x]$ is too, and so is $\Q[x,y]$. Any ascending chain of ideals in the chain on the left will be an ascending chain on the right, and so it must stabilize. Hence, the ring in question is noetherian.

    $(x) \supseteq (xy) \supseteq \dots \supseteq (xy^n) \supseteq \dots $ does not stabilize on the left hand side, so it's not artinian.
  \item
    $\Q[x^iy^j|j<i\sqrt{2}, i,j \geq 0] \subseteq \Q[x,y]$:

    This ring is noetherian, for the same reason that the previous ring was noetherian.

    However, the following descending chain 
    \begin{align*}
      (xy^{\max\limits_{j \in \N}(j<\sqrt{2})}) \supseteq (x^2y^{\max\limits_{j \in \N}(j<2\sqrt{2})}) \supseteq \dots \supseteq (x^iy^{\max\limits_{j \in \N}(j<i\sqrt{2})}) \supseteq \dots
    \end{align*}
    doesn't stabilize, so this ring is not artinian. 
  \item
    The $\Z$ module $\Z[1/2]/\Z$:

    \begin{align*}
      \Z[1/2]/\Z = \set{a\cfrac{1}{2} + \Z: a \in \Z}.
    \end{align*}

    Any non trivial element of this module will generate the whole $\Z$-module, and so the module is simple, so has a composition series, so is both artinian and noetherian.
    
  \item
    The $\Z$ module $\Q/\Z$:

    Consider 
    \begin{align*}
      \Z\cfrac{1}{2} \subseteq \Z\cfrac{1}{4} \subseteq \Z\cfrac{1}{8} \subseteq \dots 
    \end{align*}
    and
    \begin{align*}
      \Z\cfrac{2}{3} \supseteq \Z\cfrac{4}{3} \supseteq \Z\cfrac{8}{3} \supseteq \dots       
    \end{align*}
    Neither chains stabilize, and so we conclude that the module is neither noetherian nor artinian.

  \item
    The $k[x]$ module $k(x)/k[x]$:

    We use very similar counter examples:
    \begin{align*}
      k[x] \cfrac{1}{x} \subseteq k[x] \cfrac{1}{x^2} \subseteq k[x] \cfrac{1}{x^3} \subseteq \dots
    \end{align*}
    and
    \begin{align*}
      k[x] \cfrac{x-1}{x} \supseteq k[x] \cfrac{(x-1)^2}{x} \supseteq k[x] \cfrac{(x-1)^3}{x} \supseteq \dots 
    \end{align*}
    Neither stabilize, so the module is neither noetherian nor artinian.

  \item
    The $k[x]$ module $k[x,x^{-1}]/k[x]$:

    Both of the counter examples from the previous question work in this case too, which shows that the module in question is neither noetherian nor artinian. 

  \end{enumerate}
\end{proof}

\begin{exercise}
  Describe $\hom_{k^{n \times n}}(k^{n \times r}, k^{n \times s})$
\end{exercise}
\begin{proof}
  Let 
  \begin{align*}
    M = \hom_{k^{n \times n}}(k^{n \times r}, k^{n \times s}).
  \end{align*}
  We can fully describe $M$ by matrices $k^{r \times s}$, so let's write
  \begin{align*}
    M = \set{\alpha_A : A \in k^{r \times s}}.
  \end{align*}
  If $\alpha_A \in M$ and $B \in k^{n \times r}$, then we define how $\alpha_A$ acts on $B$ by
  \begin{align*}
    \alpha_A(B) = BA.
  \end{align*}
  We define the addition operation $+_M: M \times M \to M$ by 
  \begin{align*}
    (\alpha_{A_1} +_M \alpha_{A_2})(B) = \alpha_{A_1 + A_2}(B) = B(A_1 + A_2).
  \end{align*}
  The additive identity is given by $\alpha_0$ for $0 \in k^{r \times s}$, and so $M$ is an additive abelian group.
  
  Scalar multiplication $\cdot_M: k^{n \times n} \times M \to M$ is given by
  \begin{align*}
    (C \cdot_M \alpha_A) (B) = C(\alpha_A(B))
  \end{align*}
  Let's package all of this information as a tuple
  \begin{align*}
    \mathcal{M} = (M, +_M, \cdot_M).
  \end{align*}
  $\mathcal{M}$ fully describes the module in question. 
\end{proof}

\begin{exercise}
  Let
  \begin{align*}
    A = \set{
      \begin{pmatrix}
        a & b \\
        0 & c 
      \end{pmatrix}
      |
      a,b,c \in k}.
  \end{align*}
  We have submodules:
  \begin{align*}
    A_0 &:= 0, \\
    A_1 &:= \set{
    \begin{pmatrix}
      a & 0 \\
      0 & 0
    \end{pmatrix}
    |
    a \in k
    }, \\
    A_2 &:= \set{
      \begin{pmatrix}
        a & b \\
        0 & 0
      \end{pmatrix}
      |
      a,b \in k
    }, \\
    A_3 &:= A.
  \end{align*}
  Then $A_0 \subseteq A_1 \subseteq A_2 \subseteq A_3$ is a composition series for $A$.

  We have maximal ideals
  \begin{align*}
    \set{
      \begin{pmatrix}
        a & b \\
        0 & 0
      \end{pmatrix}
      |
      a,b \in k
      }
  \end{align*}
  and
  \begin{align*}
    \set{
    \begin{pmatrix}
        0 & b \\
        0 & c
    \end{pmatrix}
    |
    b,c \in k
    }
  \end{align*}
  and so as the Jacobson radical is the intersection of all maximal ideals we have
  \begin{align*}
    J(A) =
    \set{
    \begin{pmatrix}
      0 & b \\
      0 & 0
    \end{pmatrix}
    |
    b \in k
    }.
  \end{align*}
\end{exercise}

\begin{exercise}
  Find the Wedderburn decomposition of $\R G$ where $G$ is a group of
  \begin{enumerate}
  \item
    $\Z / 4 \Z$
  \item
    $S_3$
  \item
    $Q_8 = \set{\pm 1, \pm i, \pm j, \pm k}$
  \item
    $D_4 = \set{ \sigma, \tau | \tau^2 = \sigma^4 = e, \tau \sigma \tau = \sigma^{-1}}$
  \end{enumerate}
\end{exercise}
\begin{proof}
  \begin{enumerate}
  \item
    $G=\Z / 4 \Z$:

    Since $G = \set{\sigma: \sigma^4=e}$, we have 
    \begin{align*}
      \R G &= \R[x]/(x^4-1) \cong \R[x]/(x^2-1) \oplus \R[x]/(x^2+1) \\
      &\cong \R[x]/(x-1) \oplus \R[x]/(x+1) \oplus \C \\
      &\cong \R \oplus \R \oplus \C
    \end{align*}
  \item
    $G = S_3$:
    
    \begin{align*}
      \R G &= \R \oplus \R \oplus \R^{2 \times 2}
    \end{align*}

  \item
    $G = Q_8$:

    \begin{align*}
      \R G = \R \oplus \R \oplus \R \oplus \R \oplus \R^{2\times2}
    \end{align*}

  \item
    $G = D_4$:
    
    \begin{align*}
      \R G = \R \oplus \R \oplus \R \oplus \R \oplus \R^{2\times2}
    \end{align*}
    
  \end{enumerate}
\end{proof}
\end{document}
