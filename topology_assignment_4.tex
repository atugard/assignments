\documentclass[12pt]{extarticle}
%Some packages I commonly use.
\usepackage[english]{babel}
\usepackage{graphicx}
\usepackage{framed}
\usepackage[normalem]{ulem}
\usepackage{amsmath}
\usepackage{amsthm}
\usepackage{amssymb}
\usepackage{amsfonts}
\usepackage{mathtools}
\usepackage{enumerate}
\usepackage[utf8]{inputenc}
\usepackage[top=1 in,bottom=1in, left=1 in, right=1 in]{geometry}

%A bunch of definitions that make my life easier
\newcommand{\matlab}{{\sc Matlab} }
\newcommand{\abs}[1]{|#1|}
\newcommand{\set}[1]{\{#1\}}
\newcommand{\cvec}[1]{{\mathbf #1}}
\newcommand{\rvec}[1]{\vec{\mathbf #1}}
\newcommand{\ihat}{\hat{\textbf{\i}}}
\newcommand{\jhat}{\hat{\textbf{\j}}}
\newcommand{\khat}{\hat{\textbf{k}}}
\newcommand{\minor}{{\rm minor}}
\newcommand{\trace}{{\rm trace}}
\newcommand{\spn}{{\rm Span}}
\newcommand{\rem}{{\rm rem}}
\newcommand{\ran}{{\rm range}}
\newcommand{\range}{{\rm range}}
\newcommand{\mdiv}{{\rm div}}
\newcommand{\proj}{{\rm proj}}
\newcommand{\R}{\mathbb{R}}
\newcommand{\N}{\mathbb{N}}
\newcommand{\Q}{\mathbb{Q}}
\newcommand{\Z}{\mathbb{Z}}
\newcommand{\<}{\langle}
\renewcommand{\>}{\rangle}
\renewcommand{\emptyset}{\varnothing}
\newcommand{\attn}[1]{\textbf{#1}}
\theoremstyle{definition}
\newtheorem{theorem}{Theorem}
\newtheorem{prob}{Problem}
\newtheorem{corollary}{Corollary}
\newtheorem*{definition}{Definition}
\newtheorem*{example}{Example}
\newtheorem*{note}{Note}
\newtheorem{exercise}{Exercise}
\newcommand{\bproof}{\bigskip {\bf Proof. }}
\newcommand{\eproof}{\hfill\qedsymbol}
\newcommand{\Disp}{\displaystyle}
\newcommand{\qe}{\hfill\(\bigtriangledown\)}
\setlength{\columnseprule}{1 pt}


\title{ Math 5205 -- Topology Assignment 4}
\author{David Draguta}
\date{2021-11-06}

\begin{document}

\maketitle

\begin{exercise}
  \begin{enumerate}
  \item
    Exhibit a sequence $(x_n)$ on a set $X$ and a subnet of $(x_n)$ which is not a sequence.
  \item
    Exhibit a net which has no sequence as a subnet
  \item
    Let $P: \Lambda \to X$ be a net and $Q = P \circ \varphi : M \to X$ a subnet of $P$ (i.e. $\Lambda$ and $M$ are directed sets, $\varphi: M \to \Lambda$, and $\forall \lambda_0 \in \Lambda$ 
  \end{enumerate}
\end{exercise}
\begin{proof}
  \begin{enumerate}
  \item
    We let $P: \N \to \N, n \mapsto n$ be our sequence and write it as $(n)_{n \in \N}$; and $\varphi : \R \to \N, r \mapsto \abs{\lceil r \rceil} $ our map,
    such that $Q = P \circ \varphi$ is a subnet, which we write as $(x_r)_{r \in \R}$, for $x_r = \abs{\lceil r \rceil}$. Then, $(x_r)_{r \in \R}$ is a subnet which is not a sequence. 
  \item
    Let $\Lambda = \abs{2^{\R}}$. Then we can just let $P: \Lambda \to X$ be the identity, where $X = \Lambda$. Then the existence of some $\varphi: \N \to \Lambda$ such that $Q = P \circ \varphi$
    is a sequence implies that the cofinality...
  \item

  \end{enumerate}
\end{proof}
\begin{exercise}
  \begin{enumerate}
  \item
    Let $X = Y \times Z$ where $Y,Z$ are topological spaces, and let $B \subseteq Y, C \subseteq Z$. Prove that $\overline{B} \times \overline{C}  = \overline{B \times C}$ by using nets.
    Hint: If $y \in \overline{B}$ and $z \in \overline{C}$ then there is a net $(b_\xi)_{\xi \in \Xi}$ in $B$ with $b_{\xi} \to y$ and there is a net $(c_{\omega})_{\omega \in \Omega}$ in $C$
    with $c_{\omega} \to z$, but what you need is a net $(x_{\lambda})_{\lambda \in \Lambda} = ((y_{\lambda}, z_{\lambda}))_{\lambda \in \Lambda}$ in $B \times C$.
  \item
    Let $(M, \rho)$ be a metric space. A net $(x_{\lambda})_{\lambda \in \Lambda}$ in $M$ is called a Cauchy net if
    \begin{align*}
      \forall \epsilon > 0, \exists \lambda_0 \in \Lambda, \forall \lambda, \lambda' \geq \lambda_0 \quad  \rho(x_{\lambda}, x_{\lambda'}) < \epsilon .
    \end{align*}
    Recall that $(M, \rho)$ is called complete if every cauchy sequence converges. Prove that if $(M, \rho)$ is complete then every cauchy net in $M$ converges.

    Hint: Construct a sequence $\lambda_1 \leq \lambda_2 \leq \dots$ of elements of $\Lambda$, such that for each $n$ and all $\lambda, \lambda' \geq \lambda_n, \rho(x_{\lambda}, x_{\lambda'}) \leq 2^{-n}$.
    Show that the sequence $y_n = x_{\lambda_n}$ is a cauchy sequence in $M$ (caution: $y_n$ need not be a subnet). 
  \end{enumerate}
\end{exercise}
\begin{proof}
  \begin{enumerate}
  \item
    Let $(x,y) \in \overline{B} \times \overline{C}$, then by theorem 144 we have sequences $(x_{\xi})_{\xi \in \Xi}$ in $B$ and $(y_{\omega})_{\omega \in \Omega}$ in $C$
    such that $x_{\xi} \to x$ and $y_{\omega} \to y$. We let $\Lambda = \Xi \times \Omega$, and say for $\lambda_1 = (\xi_1, \omega_1), \lambda_2 = (\xi_2, \omega_2)$ that
    \begin{align*}
      \lambda_1 \leq \lambda_2 \iff \xi_1 \leq \xi_2 \text{ and } \omega_1 \leq \omega_2
    \end{align*}
    We define the net
    \begin{align*}
      (x_{\lambda})_{\lambda \in \Lambda} = ((y_{\lambda}, z_{\lambda}))_{\lambda \in \Lambda}.
    \end{align*}
    Then, $(x_{\lambda})_{\lambda \in \Lambda}$ is entirely in $B \times C$, and $x_{\lambda} \to (y,z)$, so that $(y,z) \in \overline{A \times B}$.
    \begin{align*}
      (x_{\lambda}, y_{\lambda}) \to (x,y)
    \end{align*}

    For the other direction if $(x,y) \in \overline{B \times C}$, then, agin by theorem 144, there exists some sequence
    $(x_{\lambda}, y_{\lambda})_{\lambda \in \Lambda}$ such that $(x_{\lambda}, y_{\lambda}) \to (x,y)$. Well, we have $(x_{\lambda})_{\lambda \in \Lambda}$ a sequence entirely in $B$ that converges to $x \in B$,
    so that $x \in \overline{B}$, and the same holds for $(y_{\lambda})_{\lambda \in \Lambda}$ and $y \in C$, so that $y \in \overline{C}$. Hence, again by theorem 144, $(x,y) \in \overline{B} \times \overline{C}$.
  \item
    Let $(x_{\lambda})_{\lambda \in \Lambda}$ be a cauchy net. Then for each $n$ we can find some $\lambda_n \in \Lambda$ such that for all $\lambda, \lambda' \geq \lambda_n$
    \begin{align*}
      \rho(x_{\lambda}, x_{\lambda'}) \leq \cfrac{1}{2^n}
    \end{align*}
    We define the sequence $(y_n)_{n \in \N} = (x_{\lambda_n})_{n \in \N}$. Then $(y_n)_{n \in \N}$ is a cauchy sequence in $M$, and since
    this space is complete we get that it converges to some value $x$. Let $\epsilon>0$, then there exists $n_0 \in \N$ such that for all $n \geq n_0$
    \begin{align*}
      \rho(y_n, y) < \epsilon/2.
    \end{align*}
    Also, since this is a cauchy net, there exists $\lambda^{*}$ such that for all $\lambda, \lambda' \geq \lambda^{*}$ we have
    \begin{align*}
      \rho(x_{\lambda}, x_{\lambda'}) < \epsilon/2.
    \end{align*}
    Then, let $\lambda_0 := \max(\lambda^{*}, \lambda_{n_0})$. We have, then for all $\lambda \geq \lambda_0$
    \begin{align*}
      \rho(x_{\lambda}, x) &\leq \rho(x_{\lambda}, x_{\lambda_n}) + \rho(x_{\lambda_n}, x_{\lambda})  \\
      & < \cfrac{\epsilon}{2} + \cfrac{\epsilon}{2} = \epsilon
    \end{align*}
  \end{enumerate}
\end{proof}
\begin{exercise}
  \begin{enumerate}
  \item
    A topological space is called completely normal if every subspace of $X$ is normal. Prove TFCAE for a topological space $X$:
    \begin{enumerate}
    \item X is completely normal
    \item Every open subspace of $X$ is normal
    \item For any $A,B \subseteq X$, if $A \cap \overline{B} = \overline{A} \cap B = \emptyset$, then there exist disjoint open sets $U,V \subseteq X$ with $A \subseteq U$ and $B \subseteq V$.
    \item
      Prove or disprove: the Sorgenfrey line $\mathbb{E}$ is completely normal. 
    \end{enumerate}
  \item
    Prove that every pseudometric space is completely normal. 
  \end{enumerate}
\end{exercise}
\begin{proof}
  \begin{enumerate}
  \item
    \begin{itemize}
    \item $1) \implies 2)$:

      Well, an open subspace is a subspace, so it's normal, by $1)$.
    \item $2) \implies 3)$:

      Let $A, B \subseteq X$ be such that $\overline{A} \cap B = A \cap \overline{B} = \emptyset$. Consider the open subspace $Y := (X \setminus \overline{A}) \cup (X \setminus \overline{B})$. We have then $A,B \subseteq Y$, and
      \begin{align*}
        \overline{A} \cap Y &= \overline{A} \cap ((X \setminus \overline{A}) \cup (X \setminus \overline{B})) \\ 
        &= (\overline{A} \cap (X \setminus \overline{A})) \cup (\overline{A} \cap (X \setminus \overline{B})) \\
        &= \emptyset \cup (\overline{A} \cap (X \setminus \overline{B})) \\
        &= (A \cup Fr(A)) \cap (X \setminus \overline{B}) \\
        &= A \cap (X \setminus \overline{B}) \cup (Fr(A) \cap (X \setminus \overline{B})) \\
        &= A \cup (Fr(A) \setminus Fr(B))
      \end{align*}
      so that $A^{*} := A \cup (Fr(A) \setminus Fr(B))$ is closed in $Y$ (as $\overline{A}$ is closed in $X$), and similarly one shows that $B^{*} := B \cup (Fr(B) \setminus Fr(A))$ is closed in $Y$.

      Next, we check that the two sets are disjoint. We have
      \begin{align*}
        A^{*} \cap B^{*} &= \\
        &= (A \cap B) \cup (A \cap Fr(B) \setminus Fr(A)) \cup ((Fr(A) \setminus Fr(B)) \cap B) \\
        &\cup ((Fr(A) \setminus Fr(B)) \cap (Fr(B) \setminus Fr(A))).
      \end{align*}
      As $A \cap \overline{B} = \emptyset$, $A \cap B = \emptyset$ and $A \cap Fr(B) = \emptyset$; and as $\overline{A} \cap B = \emptyset$, $Fr(A) \cap B = \emptyset$ too; also,
      $(Fr(A) \setminus Fr(B)) \cap (Fr(B) \setminus Fr(A)) = \emptyset$. Hence we get,
      \begin{align*}
        A^{*} \cap B^{*} = \emptyset
      \end{align*}
      Thus, since $Y$ is an open subspace of $X$, it's normal, and there exist disjoint open (with respect to $Y$) sets $U,V$ such that
      $A^{*} \subseteq U$ and $B^{*} \subseteq V$. Now as $Y$ is open, $U, V$ are open with respect to $X$; and $A \subseteq U$, $B \subseteq V$ (as $A \subseteq A^{*}$ and $B \subseteq B^{*}$).

    \item $3) \implies 1)$:
      
      Let $Y \subseteq X$ be a subspace of $X$. Let $V_1, V_2 \subseteq Y$ be disjoint closed sets. Then there exist closed (in $X$) sets $W_1, W_2$ such that
      $V_1 = W_1 \cap Y$ and $V_2 = W_2 \cap Y$. We have $\overline{V_i} = \overline{W_i} \cap Y = W_i \cap Y = V_i $, since $W_i$ are closed in $X$, for $i=1,2$.
      Hence, $\emptyset = V_1 \cap V_2 = \overline{V_1} \cap V_2 = V_1 \overline{V_2}$, and so by 3) there exist $U_1, U_2$ disjoint open (in $X$) sets such that
      $V_i \subseteq U_i$, for $i=1,2$. Now, of course this implies $V_i \subseteq U_i \cap Y$, and $U_i \cap Y$ are open in $Y$, so that $Y$ is a normal subspace. 
    \end{itemize}
  \item
    Let $Y \subseteq X$ be an open subset of the pseudometric space $X$. Suppose that we have disjoint, closed subsets $A,B \subseteq Y$.
    Then as $A \cap B = \emptyset$, we have $A \subseteq Y \setminus B$, and so as $Y \setminus B$ is open we can choose for any $x \in A$ some $\delta_x>0$ such that $\mathcal{U}(x, \delta_x) \cap Y \subseteq Y \setminus B$.
    We do the same thing for $B \subseteq Y \setminus A$, i.e. we choose a $\epsilon_y >0$ for each $y \in B$ such that $\mathcal{U}(y, \epsilon_y) \cap Y \subseteq Y \setminus A$. Define
    \begin{align*}
      U := \bigcup\limits_{x \in A} (\mathcal{U}_{\rho}(x, \delta_x/2) \cap Y), \quad V := \bigcup\limits_{y \in B} (\mathcal{U}_{\rho}(y, \epsilon_y/2) \cap Y).
    \end{align*}
    Then $A \subseteq U$ and $B \subseteq V$. Suppose that $z \in U \cap V$, then we have
    \begin{align*}
      \rho(x,y) \leq \rho(x,z) + \rho(z,y) < \delta_x/2 + \epsilon_y/2.
    \end{align*}
    If $\delta_x \leq \epsilon_y$, then 
    \begin{align*}
      \rho(x,y) \leq \rho(x,z) + \rho(z,y) < \delta_x/2 + \epsilon_y/2 \leq \epsilon_y,
    \end{align*}
    so that $x \in \mathcal{U}(y, \epsilon_y) \cap Y$, which is a contradiction.

    Similarly if $ \epsilon_y \leq \delta_x$, we have that $y \in \mathcal{U}(x, \delta_x) \cap Y$. Hence, no such $z$ can exist and $U \cap V = \emptyset$. Thus, condition 2 of the first part of this question is satisfied and $X$ is completely normal.
  \item
    We let $A := (- \infty, a)$ and $B = (a, \infty)$. Then $\overline{A} = (- \infty, a) $ and $\overline{B} = [a, \infty)$, so we get 
    \begin{align*}
      \overline{A} \cap B = (- \infty, a) \cap (a, \infty) = \emptyset
    \end{align*}
    and 
    \begin{align*}
      A \cap \overline{B} = (- \infty, a) \cap [a, \infty) = \emptyset
    \end{align*}
    but $A, B$ are not separable by disjoint open sets, so that condition 3 of the first part of this question fails, and $\mathbb{E}$ is not completely normal. 
  \end{enumerate}
\end{proof}
\begin{exercise}
  Let $(X, \tau)$ be a topological space. Denote by $\mathcal{U}_x$ the nhood system at $x \in X$ and by $\overline{\mathcal{U}}_x$,
  the collection of closed nhoods of $x$. Prove the following:
  \begin{enumerate}
  \item
    $X$ is $T_0$ $\iff$ for all $x, y \in X, x \neq y$ implies $\overline{\set{x}} \neq \overline{\set{y}}$.
  \item
    $X$ is $T_1$ $\iff$ for every $x \in X$, $\set{x} = \bigcap \mathcal{U}_x$.
  \item
    $X$ is $T_2$ $\iff$ for every $x \in X$, $\set{x} = \bigcap \overline{\mathcal{U}_x}$.
  \end{enumerate}
\end{exercise}
\begin{proof}
  \begin{enumerate}
  \item 
    \begin{itemize}
    \item
      $\implies$: Say $x,y \in X$ such that $x \neq y$, without loss of generality then there exists some open $U \subseteq X$ such that $y \in U$ but $x \not \in U$.
      We have then $x \in X \setminus U$ and $y \not \in X \setminus U$, which is closed, so that $y \not \in \overline{\set{x}}$ and $\overline{\set{x}} \neq \overline{\set{y}}$.
    \item 
      $\impliedby$: Suppose $x, y \in X$ are such that $x \neq y$. Then $\overline{\set{x}} \neq \overline{\set{y}}$, i.e. there exists some closed $F$ such that
      $\overline{\set{x}} \subseteq F$ but $\overline{\set{y}} \not \subseteq F$, i.e. $y \in F^c$, but $x \not \in F^c$. Well, $F^c$ is open, so $X$ is $T_0$.
    \end{itemize}
  \item
    \begin{itemize}
    \item
      $\implies$: Let $y \in X$ be different from $x$. Then there exists some $U \in \tau$ such that $x \in U$ but $y \not \in U$; but $U \in \mathcal{U}_x$, so that
      $y \not \in \bigcap\mathcal{U}_u$. As $y$ was arbitrary, we conclude that $\set{x} = \bigcap\mathcal{U}_u$.
    \item
      $\impliedby$: Let $x,y \in X$ such that $x \neq y$. Then, as $ \set{x} = \bigcap\limits \mathcal{U}_x$, there exists some $U \in \mathcal{U}_x$ such that $y \not \in U$;
      but then $x \in U^o$ and $y \not \in U^o$, so that $X$ is $T_1$.
    \end{itemize}
  \item
    \begin{itemize}
    \item $\implies$: Let $x,y \in X$ such that $x \neq y$. Then, there exist disjoint open sets $U, V \subseteq X$ such that $x \in U$ and $y \in V$.
      We have $\overline{U} \setminus V \in \overline{\mathcal{U}_x}$, and $y \not \in \overline{U} \setminus V$, so that $y \not \in \bigcap \overline{\mathcal{U}_x}$.
      As $y$ was arbitrary, we conclude $\set{x} = \bigcap \overline{\mathcal{U}_x}$.
    \item $\impliedby$: As $\set{x} = \bigcap \overline{\mathcal{U}_x}$, there exists some $U \in \overline{\mathcal{U}_u}$ such that $y \not \in U$;
      similarly, as $\set{y} = \bigcap \overline{\mathcal{U}_y}$ there exists some $V \in \overline{\mathcal{U}_y}$ such that $x \not \in V$.
      Let $U^{*} := U^o \setminus U \cap V$ and $V^{*} := V^o \setminus U \cap V$. Then $U^*, V^* \in \tau$ are disjoint; and $x \in U^*, y \not \in U^*$, and $x \not \in V^*$, $y \in V^*$.
      Hence, $X$ is $T_2$. 
    \end{itemize}
  \end{enumerate}
\end{proof}

\begin{exercise}
  For any topological space $X$, define $\sim$ by $x \sim y \iff \overline{\set{x}} = \overline{\set{y}}$
  \begin{itemize}
  \item
    (2): The resulting quotient space $X/\sim$ $= \widetilde{X}$ is $T_0$.
  \item
    (3): The procedure above, when applied to a pseudometric space $(S, \rho)$ yields the metric identification $S^*$ of $S$ described in 2C. 
  \end{itemize}
\end{exercise}

\begin{exercise}
  \begin{enumerate}
  \item
    Let $\mathcal{F}$ be a family of pairwise disjoint subsets of a second countable topological space $X$, such that $F^o \neq \emptyset$ for every $F \in \mathcal{F}$.
    Prove that $\mathcal{F}$ is countable.
  \item
    Show that the Sorgenfrey line $\mathbb{E}$ is a Lindelof space.
  \end{enumerate}
\end{exercise}
\begin{proof}
  \begin{enumerate}
  \item
    Let $\mathcal{B}$ be a countable basis for $X$. Then as the $F \in \mathcal{F}$ are disjoint, it follows that if we define $\mathcal{B}_F$ such that
    $F^o = \bigcup \mathcal{B}_F$, that the $\mathcal{B}_F$ will also be pairwise disjoint. Hence, we get
    \begin{align*}
      \abs{\mathcal{F}} \leq \abs{\bigcup\limits_{F \in \mathcal{F}} \mathcal{B}_F} \leq \abs{\mathcal{B}} \leq \omega .
    \end{align*}
  \item
    Let $\mathcal{U}$ be an open cover of $\mathbb{E}$. We define
    \begin{align*}
      \mathcal{P} = \set{\mathcal{F} \subseteq \mathcal{U}: (\forall F_1, F_2 \in \mathcal{F},
        F_1 \neq F_2 \implies F_1 \cap F_2 = \emptyset) \wedge
        (\forall F \in \mathcal{F}, Int_{\R}(F) \neq \emptyset)}
    \end{align*}
    Let $\mathcal{C} \subseteq \mathcal{P}$ be a chain. Then, I claim $\bigcup \mathcal{C} \in \mathcal{P}$. Let $F_1, F_2 \in \bigcup \mathcal{C}$.
    Then, there exists some $\mathcal{F}_1 \in \mathcal{C}$ such that $F_1 \in \mathcal{F}_1$ and some $\mathcal{F}_2 \in \mathcal{C}$ such that $F_2 \in \mathcal{F}_2$. Now,
    either $\mathcal{F}_1 \subseteq \mathcal{F}_2$ or $\mathcal{F}_2 \subseteq \mathcal{F}_1$. Suppose, without loss of generality, that $\mathcal{F}_1 \subseteq \mathcal{F}_2$.
    Then $F_1, F_2 \in \mathcal{F}_2 \in \mathcal{P}$, and by definition the two sets are disjoint. If $F \in \bigcup \mathcal{C}$, then there exists some $\mathcal{F} \in \mathcal{C}$ such that $F \in \mathcal{F}$, and so by definition of these families $F^o \neq \emptyset$. Hence, $\bigcup \mathcal{C} \in \mathcal{P}$.

    By Zorn's Lemma, $\mathcal{P}$ has a maximal element, which we call $\mathcal{F}$.

    It remains to check that $\mathcal{F}$ covers $\mathbb{E}$. Suppose not. Then $\mathbb{E} \setminus \bigcup \mathcal{F} \neq \emptyset$. Hence there exists...

    
  \end{enumerate}
\end{proof}
\end{document}



