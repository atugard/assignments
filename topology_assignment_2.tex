\documentclass[12pt]{extarticle}
%Some packages I commonly use.
\usepackage[english]{babel}
\usepackage{graphicx}
\usepackage{framed}
\usepackage[normalem]{ulem}
\usepackage{amsmath}
\usepackage{amsthm}
\usepackage{amssymb}
\usepackage{amsfonts}
\usepackage{enumerate}
\usepackage[utf8]{inputenc}
\usepackage[top=1 in,bottom=1in, left=1 in, right=1 in]{geometry}

%A bunch of definitions that make my life easier
\newcommand{\matlab}{{\sc Matlab} }
\newcommand{\abs}[1]{|#1|}
\newcommand{\set}[1]{\{#1\}}
\newcommand{\cvec}[1]{{\mathbf #1}}
\newcommand{\rvec}[1]{\vec{\mathbf #1}}
\newcommand{\ihat}{\hat{\textbf{\i}}}
\newcommand{\jhat}{\hat{\textbf{\j}}}
\newcommand{\khat}{\hat{\textbf{k}}}
\newcommand{\minor}{{\rm minor}}
\newcommand{\trace}{{\rm trace}}
\newcommand{\spn}{{\rm Span}}
\newcommand{\rem}{{\rm rem}}
\newcommand{\ran}{{\rm range}}
\newcommand{\range}{{\rm range}}
\newcommand{\mdiv}{{\rm div}}
\newcommand{\proj}{{\rm proj}}
\newcommand{\R}{\mathbb{R}}
\newcommand{\N}{\mathbb{N}}
\newcommand{\Q}{\mathbb{Q}}
\newcommand{\Z}{\mathbb{Z}}
\newcommand{\<}{\langle}
\renewcommand{\>}{\rangle}
\renewcommand{\emptyset}{\varnothing}
\newcommand{\attn}[1]{\textbf{#1}}
\theoremstyle{definition}
\newtheorem{theorem}{Theorem}
\newtheorem{prob}{Problem}
\newtheorem{corollary}{Corollary}
\newtheorem*{definition}{Definition}
\newtheorem*{example}{Example}
\newtheorem*{note}{Note}
\newtheorem{exercise}{Exercise}
\newcommand{\bproof}{\bigskip {\bf Proof. }}
\newcommand{\eproof}{\hfill\qedsymbol}
\newcommand{\Disp}{\displaystyle}
\newcommand{\qe}{\hfill\(\bigtriangledown\)}
\setlength{\columnseprule}{1 pt}


\title{ Math 5205 -- Topology Assignment 1}
\author{David Draguta}
\date{2021-09-25

\begin{document}

\maketitle

\begin{exercise}
  Consider the following functions from $\R$ into $\R$: $f(x)=-x$, $g(x) = x^2 $, $h(x) = x^3 $,
  $k(x) =
  \begin{cases}
    0 \quad &\text{if} \, x \leq 0 \\
    1 \quad &\text{if} \, x > 0 \\
  \end{cases}$,
  and
  $l(x) =
  \begin{cases}
    0 \quad &\text{if} \, x < 0 \\
    1 \quad &\text{if} \, x \geq 0 \\
  \end{cases}$.
  
  Which of these functions is continuous when considered:

  \begin{enumerate}
  \item
    As a function from $\mathbb{E}$ into $\mathbb{R}$?
  \item
    As a function from $\mathbb{R}$ into $\mathbb{E}$?
  \item
    As a function from $\mathbb{E}$ into $\mathbb{E}$?
  \end{enumerate}

  $\mathbb{E}$ denotes the Sorgenfrey line from #6 (Assignment 1).
\end{exercise}

\begin{proof}
  We use the bases $\mathcal{B}_{\mathbb{E}} = \set{[a,b): a<b}$ and $\mathcal{B}_{\mathbb{R}} = \set{(x-\epsilon,x+\epsilon): x \in \R, \epsilon>0}$ for $(\mathbb{E}, \tau_\mathbb{E})$ and $(\mathbb{R}, \tau_\mathbb{R})$ respectively. Let $a<b$, and choose an $N > 1/b-a$. Then, $(a,b) = \bigcup\limits_{n \geq N}[a+1/n,b)$, and so $\mathcal{B}_{\mathbb{R}} \subseteq \mathcal{B}_{\mathbb{E}}$, which gives us that $\tau_\mathbb{R} \subseteq \tau_\mathbb{E}$. Hence any function that's continuous from $\R \to \R$ will be continuous from $\mathbb{E} \to \mathbb{R}$.
      
      We have $k^{-1}[a,b) =
        \begin{cases}
          \R \quad & \, 0,1 \in [a,b) \\
            (-\infty,0] & \, 0 \in [a,b), 1 \not \in [a,b) \\
              (0,\infty) & \, 0 \not \in [a,b), 1 \in [a,b) \\
                  \emptyset \, &\text{otherwise} 
        \end{cases}$,
        
        and it's the same thing for the preimage of $(a,b)$.

        We have
        $l^{-1}[a,b) =
        \begin{cases}
          \R \quad & \, 0,1 \in [a,b) \\
            (-\infty,0) & \, 0 \in [a,b), 1 \not \in [a,b) \\
              [0,\infty) & \, 0 \not \in [a,b), 1 \in [a,b) \\
                  \emptyset \, &\text{otherwise} 
        \end{cases}$,

        and the same for the preimage of $(a,b)$.
  \begin{enumerate}
  \item
    $f,g,h: \R \to \R$ are all continuous, so $f,g,h: \mathbb{E} \to \R$ are too. $l: \mathbb{E} \to \mathbb{R}$ is continuous, but $k: \mathbb{E} \to \mathbb{R}$ is not.
  \item
    All of $f,g,h,k,l: \R \to \mathbb{E}$ are not continuous.
  \item
    $f,g,h,l: \mathbb{E} \to \mathbb{E}$ are continuous, but $k: \mathbb{E} \to \mathbb{E}$ is not.

\end{proof}

\end{document}
