\documentclass[12pt]{extarticle}
%Some packages I commonly use.
\usepackage[english]{babel}
\usepackage{graphicx}
\usepackage{framed}
\usepackage[normalem]{ulem}
\usepackage{amsmath}
\usepackage{amsthm}
\usepackage{amssymb}
\usepackage{amsfonts}
\usepackage{enumerate}
\usepackage[utf8]{inputenc}
\usepackage[top=1 in,bottom=1in, left=1 in, right=1 in]{geometry}

%A bunch of definitions that make my life easier
\newcommand{\matlab}{{\sc Matlab} }
\newcommand{\abs}[1]{|#1|}
\newcommand{\set}[1]{\{#1\}}
\newcommand{\cvec}[1]{{\mathbf #1}}
\newcommand{\rvec}[1]{\vec{\mathbf #1}}
\newcommand{\ihat}{\hat{\textbf{\i}}}
\newcommand{\jhat}{\hat{\textbf{\j}}}
\newcommand{\khat}{\hat{\textbf{k}}}
\newcommand{\minor}{{\rm minor}}
\newcommand{\trace}{{\rm trace}}
\newcommand{\spn}{{\rm Span}}
\newcommand{\rem}{{\rm rem}}
\newcommand{\ran}{{\rm range}}
\newcommand{\range}{{\rm range}}
\newcommand{\mdiv}{{\rm div}}
\newcommand{\proj}{{\rm proj}}
\newcommand{\R}{\mathbb{R}}
\newcommand{\N}{\mathbb{N}}
\newcommand{\Q}{\mathbb{Q}}
\newcommand{\Z}{\mathbb{Z}}
\newcommand{\<}{\langle}
\renewcommand{\>}{\rangle}
\renewcommand{\emptyset}{\varnothing}
\newcommand{\attn}[1]{\textbf{#1}}
\theoremstyle{definition}
\newtheorem{theorem}{Theorem}
\newtheorem{prob}{Problem}
\newtheorem{corollary}{Corollary}
\newtheorem*{definition}{Definition}
\newtheorem*{example}{Example}
\newtheorem*{note}{Note}
\newtheorem{exercise}{Exercise}
\newcommand{\bproof}{\bigskip {\bf Proof. }}
\newcommand{\eproof}{\hfill\qedsymbol}
\newcommand{\Disp}{\displaystyle}
\newcommand{\qe}{\hfill\(\bigtriangledown\)}
\setlength{\columnseprule}{1 pt}


\title{ Math 5205 -- Topology Assignment 1}
\author{David Draguta}
\date{2021-09-25

\begin{document}

\maketitle

\begin{exercise}
  Consider the following functions from $\R$ into $\R$: $f(x)=-x$, $g(x) = x^2 $, $h(x) = x^3 $,
  $k(x) =
  \begin{cases}
    0 \quad &\text{if} \, x \leq 0 \\
    1 \quad &\text{if} \, x > 0 \\
  \end{cases}$,
  and
  $l(x) =
  \begin{cases}
    0 \quad &\text{if} \, x < 0 \\
    1 \quad &\text{if} \, x \geq 0 \\
  \end{cases}$.
  
  Which of these functions is continuous when considered:

  \begin{enumerate}
  \item
    As a function from $\mathbb{E}$ into $\mathbb{R}$?
  \item
    As a function from $\mathbb{R}$ into $\mathbb{E}$?
  \item
    As a function from $\mathbb{E}$ into $\mathbb{E}$?
  \end{enumerate}

  $\mathbb{E}$ denotes the Sorgenfrey line from \#6 (Assignment 1).
\end{exercise}

\begin{proof}
  We use the bases $\mathcal{B}_{\mathbb{E}} = \set{[a,b): a<b}$ and $\mathcal{B}_{\mathbb{R}} = \set{(x-\epsilon,x+\epsilon): x \in \R, \epsilon>0}$ for $(\mathbb{E}, \tau_\mathbb{E})$ and $(\mathbb{R}, \tau_\mathbb{R})$ respectively.
    Let $a<b$, and choose an $N > 1/b-a$. Then, $(a,b) = \bigcup\limits_{n \geq N}[a+1/n,b)$, and so $\mathcal{B}_{\mathbb{R}} \subseteq \mathcal{B}_{\mathbb{E}}$, which gives us that $\tau_\mathbb{R} \subseteq \tau_\mathbb{E}$. Hence any function that's continuous from $\R \to \R$ will be continuous from $\mathbb{E} \to \mathbb{R}$.
      
      \begin{align*}
        k^{-1}[a,b) =
          \begin{cases}
            \R \quad & \, 0,1 \in [a,b) \\
              (-\infty,0] & \, 0 \in [a,b), 1 \not \in [a,b) \\
                (0,\infty) & \, 0 \not \in [a,b), 1 \in [a,b) \\
                    \emptyset \, &\text{otherwise} 
          \end{cases},
      \end{align*}
      \begin{align*}
        l^{-1}[a,b) =
          \begin{cases}
            \R \quad & \, 0,1 \in [a,b) \\
              (-\infty,0) & \, 0 \in [a,b), 1 \not \in [a,b) \\
                  [0,\infty) & \, 0 \not \in [a,b), 1 \in [a,b) \\
                        \emptyset \, &\text{otherwise} 
          \end{cases},
      \end{align*}
      The preimage of $(a,b)$ is the same.

      \begin{enumerate}
      \item
        $f,g,h: \R \to \R$ are all continuous, so $f,g,h: \mathbb{E} \to \R$ are too. $l: \mathbb{E} \to \mathbb{R}$ is continuous, but $k: \mathbb{E} \to \mathbb{R}$ is not.
      \item
        None of $f,g,h,k,l: \R \to \mathbb{E}$ are continuous.
      \item
        $f,g,h,l: \mathbb{E} \to \mathbb{E}$ are continuous, but $k: \mathbb{E} \to \mathbb{E}$ is not.
      \end{enumerate}
\end{proof}

\begin{exercise}
  \begin{enumerate}
  \item
    Do Part 1 of Problem 4D on p.36
  \item
    Compare the topology of the looped line with the usual topology on $\R$.
  \item
    Do Part 2 of Problem 4D on p.36: express the closure in the looped line in terms of the closure with respect to the usual topology of $\R$.
    Hint: For bounded subsets of $\R$, the two closures coincide (why?).
  \end{enumerate}
\end{exercise}
\begin{proof}
  \begin{enumerate}
  \item
    We check that the basic neighbourhoods satisfy V1), V2), and V3) from Theorem 41 of the course notes, because then we know a unique topology exists with these as the basic neighbourhoods.
    Let $x \in \R$.
    \begin{itemize}
    \item
      V1): Let $V \in \mathcal{B}_x$. If $x$ is zero then there exists some $\epsilon>0$ such that $x \in (-\epsilon,\epsilon) \subseteq V$. If $x$ is not zero, then there exists some $\epsilon>0$ such that $x\in V=(x-\epsilon,x+\epsilon) \subseteq V$, and in either case V1) holds.
    \item
      V2): Let $V_1, V_2 \in \mathcal{B}_x$. If $x =0$ there exist $n_1,n_2 \in \N$ and $\epsilon_1, \epsilon_2 >0$ such that
      \begin{align*}
        V_i = (-\epsilon_i,\epsilon_i) \cup (-\infty, -n_i) \cup (n_i, \infty), \text{ for } i=1,2.
      \end{align*}
      Then
      \begin{align*}
        V_1 \cap V_2 = (-\min(\epsilon_1, \epsilon_2), \min(\epsilon_1, \epsilon_2)) \cup (-\infty, -\max(n_1,n_2)) \cup (\max(n_1,n_2), \infty)
      \end{align*}
      which is again a neighbourhood of $0$, and so $W:=V_1 \cap V_2 \in \mathcal{B}_x$ satisfies $W \subseteq V_1 \cap V_2$.
      
      Now, if $x \neq 0$, we let $V_i = (a_i,b_i)$ and we get
      \begin{align*}
        V_1 \cap V_2 = (a_1, b_1) \cap (a_2,b_2) = (\max(a_1,a_2), \min(b_1,b_2)).
      \end{align*}
      This also is a basic neighbourhood of $x$, so $W:=V_1 \cap V_2 \in \mathcal{B}_x$ satisfies  $W \subseteq V_1 \cap V_2$.
    \item
      V3): Let $V \in \mathcal{B}_x$. If $x = 0$, then for some $\epsilon>0$ and $n_0 \in \N$:
      \begin{align*}
        V = (-\epsilon, \epsilon) \cup (-\infty, -n_0) \cup (n_0, \infty).
      \end{align*}
      We let $V_0 := V$. Then for $y \in V_0$ if $y=0$ we let $W:=V$, so that $W \subseteq V$. If $y \neq 0$, then if $y \in (-\epsilon,\epsilon)$, let 
      \begin{align*}
        d := \cfrac{\min(\abs{\epsilon-y}, \abs{\epsilon+y})}{2}
      \end{align*}
      and 
      \begin{align*}
        W:= (y-d,y+d).
      \end{align*}
      If $y \in (\infty, -n_0)$, let
      \begin{align*}
        d:= \cfrac{-n_0 - y}{2},
      \end{align*}
      and
      \begin{align*}
        W:= (y-d, y+d)
      \end{align*}.

      If $y \in (n_0, \infty)$, we put
      \begin{align*}
        d := \cfrac{y-n_0}{2}
      \end{align*}
      and
      \begin{align*}
        W := (y-d, y+d).
      \end{align*}
      In all cases, $W \in \mathcal{B}_y$ and $W \subseteq V$.

      If $x \neq 0$, and $0 \not \in V$, we have $V = (x-\epsilon,x+\epsilon)$, and we take $V = V_0$. Then for any $y \in V$ we can take

      \begin{align*}
        d:= \cfrac{\min(\abs{x-y+\epsilon}, \abs{y-x+\epsilon})}{2}
      \end{align*}
      and
      \begin{align*}
        W:= (y-d, y+d)
      \end{align*}

      If $0 \in V$, then we find a smaller subinterval which doesn't contain the origin. Either $0 \in (x-\epsilon,x)$, in which case we take $V_0 = (x, x+\epsilon)$,
      or $0 \in (x, x+\epsilon)$ and we take $V_0 = (x-\epsilon, x)$. Then, we're back in the previous case, where $0 \not \in V_0$, and we just showed we can find an appropriate $W$.
    \end{itemize}
  \item
    For any base neighbourhood of the origin in the loop topology we can find an open interval contained in it; and for any base neighbourhood
    away from the origin, that neighbourhood is an interval which is a base neighbourhood in the usual topology on the real line; so that,
    by the Hausdorff Criterion, we have, if we let $\tau_1$ be the looped line topology and $\tau_2$ be the usual topology on the real line, that $\tau_2 \subseteq \tau_1$. The opposite direction isn't true, since for any open interval centered at the origin we can't find a base neighbourhood in the looped line topology which is contained in it... Therefore, $\tau_2 \subset \tau_1$ is strictly contained, so that the looped line topology is strictly finer than the usual topology.
  \item
    We have
    \begin{align*}
      Cl_L(E) =
      \begin{cases}
        Cl_{\R}(E) & \text{if } E \text{ is bounded.} \\
        Cl_{\R}(E) \cup \set{0} & \text{ otherwise.}
      \end{cases}
    \end{align*}
  \end{enumerate}
\end{proof}
\begin{exercise}
  \begin{enumerate}
  \item
    Let $X$ be an infinite set equipped with the cofinite topology. Prove or disprove: if $f: X \to \R$ is continuous at $x_0 \in X$,
    then $f$ is constant on $X$ ($\R$ has the usual topology).
  \item
    Show that the function $f:\R \to \R$, given by $f(x)=x^2$ is closed but not open.
  \item
    Show that the function $f:\R \to \R$, given by $f(x)=e^x$ is open but not closed. 
  \end{enumerate}
\end{exercise}
\begin{proof}
  \begin{enumerate}
  \item
    Let
    \begin{align*}
      A_{\epsilon} &:= f^{-1}([f(x_0) - \epsilon, f(x_0) + \epsilon]), \\
      B_{\epsilon} &:= f^{-1}((f(x_0) - \epsilon, f(x_0) + \epsilon)).
    \end{align*}
    Since $B_{\epsilon} \subseteq A_{\epsilon}$, if $A_{\epsilon}$ is finite, then so is $B_{\epsilon}$. We have $X = B_{\epsilon} \cup B_{\epsilon}^c$. Since $B_{\epsilon}$ has finite complement, by definition of cofinite topology, we have that $X$ is finite, which is a contradiction. Hence, as $A_{\epsilon}$ is closed, it must be that $A_{\epsilon} = X$, and so for all $x \in X, f(x) = f(x_0)$, i.e. $f$ is constant.
  \item
    We have for $a \neq 0$ that $f(-a,a) = [0,a^2)$, so $f$ is not open. $f[a,b] = [0, max(a^2,b^2)]$, so $f$ is closed. 
    \item
      $f(a,b) = (e^a, e^b)$, so $f$ is open; and $f(X) = (0, \infty)$, so $f$ is not closed.
  \end{enumerate}
\end{proof}
\begin{exercise}
  Let $\tau = \set{(a, \infty): a \in \R} \cup \set{\emptyset, \R}$.
  \begin{enumerate}
  \item
    Show that $\tau$ is a topology on $\R$.
  \item
    In the topological space $(\R, \tau)$ find a set $S$, such that the derived set $S'$ (Df.47, lecture notes) is not closed.
  \item
    Let $(X, \sigma)$ be a topological space in which every singleton is closed. Prove that for every $S \subseteq X$, the derived set $S'$ is $\sigma$-closed. 
  \end{enumerate}
\end{exercise}
\begin{proof}
  \begin{enumerate}
  \item
    \begin{itemize}
    \item
      $ 0, \R \in \tau.$
    \item
      Let $\mathcal{F} \subseteq \tau$. If $\bigcup \mathcal{F}$ is empty or the whole set, then there's nothing to check, as both are open. Suppose the union is neither the empty set nor the real line. Let $V = \set{a: (a, \infty) \in \mathcal{F}}$, and $a^* = \inf(V)$. Then $\bigcup \mathcal{F} = (a^*, \infty) \in \tau$. 
    \item
      If $(a, \infty), (b, \infty) \in \mathcal{F}$, we have $(a, \infty) \cap (b, \infty) = (\max(a,b), \infty) \in \tau$.
    \end{itemize}
  \item
    Take $S = \set{x}$. Then $S' = (-\infty, x)$. 
  \item
    Let $x \in X \setminus S'$. Then there exists some $U \in \mathcal{U}_x$ such that $U \cap (S \setminus \set{x}) = \emptyset$. I claim that $U \subseteq X \setminus S'$. Suppose not. Let $y \in U \cap S'$. Then, as $y \in U$, $U \in \mathcal{U}_y$. Since singletons are closed, $U \setminus \set{x} \in \mathcal{U}_y$ and $U \setminus \set{y} \in \mathcal{U}_x$.
    
    Since $y \in S'$,
    \begin{align*}
      (U \setminus \set{y}) \cap (S \setminus \set{x}) = (U \setminus \set{x}) \cap (S \setminus \set{y}) \neq \emptyset.
    \end{align*}

    Then as $U \setminus \set{y} \subseteq U$, we conclude $U \cap (S \setminus \set{x}) \neq \emptyset$, which is a contradiction. Hence, $X \setminus S'$ is open and $S'$ is closed. 

  \end{enumerate}
\end{proof}
\begin{exercise}
  Justify statements 1,2,3 of \#8E on p.58 in the text.
\end{exercise}
\begin{proof}
  \begin{enumerate}
  \item
    By Theorem 107 in the course notes, we know that $\pi_{\alpha}$ is open and continuous for each $\alpha \in A$, so that $\pi_{\alpha}(V)$ is open in $X_{\alpha}$ for each $\alpha \in A$, and $\pi_{\alpha}^{-1}(\pi_{\alpha}(V))$ is open in $X$ for each $\alpha \in A$; and so $V = \bigcap\limits_{\alpha \in A} \pi_{\alpha}^{-1}(\pi_{\alpha}(V))$ is open for all non-empty topological spaces $(X_{\alpha})_{\alpha \in A}$, which implies that the product has finitely many non-trivial terms, so that for all but finitely many terms $\pi_{\alpha}^{-1}(\pi_{\alpha}(V)) = X$, i.e. $\pi_{\alpha}(V) = X_\alpha$.
  \item
    We have that the projection map $\pi_{\alpha_0}: X_{\alpha_0}' \to X_{\alpha_0}$ is continuous with continuous inverse $\gamma: X_{\alpha_0} \to X_{\alpha_0}', \gamma(x) = (c_{\alpha})$, where 
    \begin{align*}
      c_{\alpha} =
      \begin{cases}
        x & \text{ if } \alpha = \alpha_0 \\
        b_{\alpha} & \text{ otherwise } 
      \end{cases}
    \end{align*}
  \item
    Let
    \begin{align*}
      Y = \set{x \in X: x_{\alpha} = b_{\alpha} \text{ except for finitely many $\alpha \in A$}}.
    \end{align*}
    Suppose $x \not \in Y$. Let $U \in \mathcal{U}_x$. Then $U^o$ is open in $X$ and so is written as a union of sets of the form
    \begin{align*}
      \pi_{\alpha_1}^{-1}(V_{\alpha_1}) \cap \dots \cap \pi_{\alpha_n}^{-1}(V_{\alpha_n}),
    \end{align*}
    for $V_{\alpha_i} \in \mathcal{B}_{\alpha_i}$ a subbase for $X_{\alpha_i}$. Pick any $c_{\alpha_i} \in\pi_{\alpha_i}^{-1}(V_{\alpha_i})$ and then for the remaining choices set $c_{\alpha} = b_{\alpha}$.
    Then $c:=(c_{\alpha})_{\alpha \in A} \in Y$ and $c \in U^o \subseteq U$, so $U \cap Y \neq \emptyset$. Hence, if $x \not \in Y$ then $x \in Y'$,
    and we have $\overline{Y} = X$.
  \end{enumerate}
\end{proof}

\begin{exercise}
  Justifty statements 1,2,3,4 of \#8D on p.58 in the text.

  Hint: Parts 1 and 2 are special cases of Part 3, so you may solve 1,2 and 3 at the same time, if you wish. In the second part of part 3, a counterexample is required: do question 5 above before attempting this. 
\end{exercise}
\begin{proof}
  \begin{itemize}
  \item $(\prod\limits_{\alpha \in A} A_{\alpha})^o = \prod\limits_{\alpha \in A} A_{\alpha}^o$, for $A$ finite:
    
    Then if $x \in (\prod\limits_{\alpha \in A} A_{\alpha})^o$, there exists some open $U \subseteq \prod\limits_{\alpha \in A} A_{\alpha} $ with $x \in U$.
    For each $\alpha \in A$, $x_{\alpha} \in U_{\alpha} := \pi_{\alpha}(U)$ and since the projection maps are open the $U_{\alpha} \subseteq A_{\alpha}$ are open.
    Hence, $x \in \prod\limits_{\alpha \in A} A_{\alpha}^o$.

    For the other direction suppose that $x \in \prod\limits_{\alpha \in A} A_{\alpha}^o$.
    Then for each $\alpha \in A$ there exists some open $U_\alpha \subseteq A_{\alpha}$ such that $x \in U_{\alpha}$.
    Define $U:= \bigcap\limits_{\alpha \in A} U_{\alpha}$. Then $U \subseteq \prod\limits_{\alpha \in A} A_{\alpha}$ is open and $x \in U$, so that $x \in (\prod\limits_{\alpha \in A} A_{\alpha})^o$.

    Suppose $A$ is infinite. Choose $A_{\alpha} \subset X_{\alpha}$ strictly contained. $(\prod\limits_{\alpha \in A} A_{\alpha})^o$ is open. Hence, by the first part of the previous question
    $\pi_{\alpha}((\prod\limits_{\alpha \in A} A_{\alpha})^o) = X_{\alpha} \supset A_{\alpha}$ for all but finitely many $\alpha \in A$, so that $(\prod\limits_{\alpha \in A} A_{\alpha})^o \neq \prod\limits_{\alpha \in A} A_{\alpha}^o$.
    
  \item $\overline{\prod\limits_{\alpha \in A} A_{\alpha}} = \prod\limits_{\alpha \in A} \overline{A_{\alpha}}$:
    
    Let $x \in \overline{\prod\limits_{\alpha \in A} A_{\alpha}}$. Then, for $F_\alpha \supseteq A_\alpha$ closed, since $ \bigcap\limits_{\alpha \in A} \pi_{\alpha}^{-1}(F_{\alpha})$ is closed in $X$
    and $\bigcap\limits_{\alpha \in A} \pi_{\alpha}^{-1}(F_{\alpha}) \supseteq \prod\limits_{\alpha \in A} A_{\alpha}$, we have that 
    $x \in \bigcap\limits_{\alpha \in A} \pi_{\alpha}^{-1}(F_{\alpha})$, so that $x_{\alpha} \in F_{\alpha}$ for each $\alpha \in A$. Thus, $x \in \prod\limits_{\alpha \in A} \overline{A_{\alpha}}$.

    Let $x \in \prod\limits_{\alpha \in A} \overline{A_{\alpha}}$, and $F \supseteq \prod\limits_{\alpha \in A} A_{\alpha}$ closed. Then $\pi_{\alpha}(F^c)^c$ is closed in $A_{\alpha}$ and $\pi_{\alpha}(F^c)^c \supseteq A_{\alpha}$,
    so that $x_\alpha \in \pi_{\alpha}(F^c)^c$ for each $\alpha \in A$. Hence $x \in \bigcap\limits_{\alpha \in A} \pi_{\alpha}^{-1}(\pi_{\alpha}(F^c)^c) \subseteq F$, i.e. $x \in F$.
    As $F$ was arbitrary we have $x \in \overline{\prod\limits_{\alpha \in A} A_{\alpha}}$.

  \item
    The last two bullets take care of \#1, \#2, and \#3 in the textbook, and the following bullet is \#4.
  \item
    \begin{align*}
      Fr(A \times B)
      &= \overline{A \times B} \setminus (A \times B)^o = (\overline{A} \times \overline{B}) \setminus (A^o \times B^o) \\
      &= (\overline{A} \setminus A^o \times \overline{B}) \cup (\overline{A} \times \overline{B} \setminus B^o) \\
      &= (Fr(A) \times \overline{B}) \cup (\overline{A} \times Fr(B))
    \end{align*}
  \end{itemize}
\end{proof}
\end{document}
